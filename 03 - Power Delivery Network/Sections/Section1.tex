%!TEX root = ../main.tex 

\section{Comment protéger une alimentation?}

\subsection{Protection antistatique}

\begin{frame}{Décharge Électrostatique (ESD)}
    \begin{columns}
        \begin{column}{0.66\textwidth}
            \begin{itemize}
                \item Norme IEC-61000-4-2
                \begin{itemize}
                    \item Types de décharges
                    \item Méthodologies de tests \& certification
                    \item 4 catégories de produits
                    \item Jusqu'à $\SI{\pm 8}{\kilo\volt}$ / $\SI{\pm 15}{\kilo\volt}$
                \end{itemize}
                \item Deux types de chocs statiques
                \begin{itemize}
                    \item \textbf{Contact Discharge} - Toucher directement chaque pin avec un ESD gun
                    \item \textbf{Air Discharge} - ESD gun proche du DUT jusqu'à décharge
                \end{itemize}
            \end{itemize}
        \end{column}

        \begin{column}{0.33\textwidth}
            \begin{figure}
                \centering
                \includegraphics[width=\textwidth]{pictures/ESD-discharge-finger.png}
            \end{figure}
            \begin{figure}
                \centering
                \includegraphics[width=0.5\textwidth]{pictures/ESD-logo.png}
            \end{figure}
        \end{column}
    \end{columns}
\end{frame}

\begin{frame}{Décharge Électrostatique - Waveform}
    \begin{columns}
        \begin{column}{0.5\textwidth}
        \end{column}
        \begin{column}{0.5\textwidth}
            \begin{itemize}
                \item Pic de courant initial
                \begin{itemize}
                    \item Rise time $\lesssim \SI{1}{\nano\second}$
                \end{itemize}
                \item $2^e$ pic
                \item Chute graduelle
            \end{itemize}
        \end{column}
    \end{columns}

    \vspace{-66pt}

    \begin{figure}
        \centering
        \includegraphics[width=\textwidth]{pictures/ESD-discharge-waveform.png}
    \end{figure}
\end{frame}

\begin{frame}{Circuit protégé antistatiquement}
    \begin{center}
    \resizebox{\textwidth}{!}{
    \begin{circuitikz}[american voltages]
        \draw [thick]
        (0,0) to [short, *-] (10,0)
        to [european resistor, l_=${LOAD}$] (10,5)
        (0,0) to [open, v<=$V$] (0,5)
        to [short, *- ,i=$i$] (2,5)
        to [short] (10,5)
        ;
    \end{circuitikz}
    }
    \end{center}
\end{frame}

\subsection{Protection de tension inverse}
\subsection{Protection de court-circuit}
\subsection{Protection de inrush current}
\subsection{GFCI \& Grounding}