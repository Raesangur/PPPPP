%!TEX root = ../presentation.tex 

\section[Level 7]{Dielectric Depths [20min-2h27]}

\subsection[5min-Pascal]{Ground Planes \rom{2}}
\pascalbackground
\begin{frame}{Plan}
    \begin{makelist}[\small][1.5]
        \icon[red]{\faTimes} & Ground Bounce\\
        \icon[red]{\faTimes} & Vias pour changer de layers\\
        \icon[red]{\faTimes} & item 3
    \end{makelist}
\end{frame}


\subsection[10min-Pascal]{Stackup \rom{1}}
\pascalbackground
\begin{frame}{Plan}
    \begin{makelist}[\small][1.5]
                \icon[red]{\faTimes} & Conversion\\
        \icon[red]{\faTimes} & Lecture datasheet diélectrique\\
        \icon[red]{\faTimes} & Optimisation Stackup\\
        \icon[red]{\faTimes} & Propriété FR4, copper, or, plomb
    \end{makelist}
\end{frame}


%Transmission line 3?

\subsection[10min-Max]{Matériaux \rom{3}}
\maxbackground

\begin{frame}{Rappel}
    \begin{twocolumns}[0.4]
        \leftcol
            Electric Permittivity [F/m]:
            \begin{equation*}
                \varepsilon =\varepsilon_r \varepsilon_0
            \end{equation*}
            Refractive Index:
            \begin{equation*}
                n =\frac{\sqrt{\varepsilon \mu}}{\sqrt{\varepsilon_0 \mu_0}} = \sqrt{\varepsilon_r \mu_r}
            \end{equation*}
            Permittivitée Complexe:
            \begin{equation*}
                \varepsilon = \epsilon_1 + j \epsilon_2 = (n+jk)^2
            \end{equation*}
        \rightcol
            Dans les sections précédentes, nous avons vu pourquoi les propriété des matériaux sont définies avec \textbf{nombres complexe} $\mathbb{C}$.
    \end{twocolumns}
\end{frame}

\begin{frame}{Rappel}
    \centering
    \begin{twocolumns}[0.4]
        \leftcol
            \centering
            \underline{Intensitée du Champ Électrique}
            \begin{equation*}
                \efield \rightarrow [\SI{}{\volt\per\meter}]
            \end{equation*}
            \underline{Densitée de Flux Électrique}
            \begin{equation*}
                \eflux \rightarrow [\SI{}{\coulomb\per\meter\squared}]
            \end{equation*}
            \underline{Relation}
            \begin{equation*}
                \eflux = \varepsilon \efield
            \end{equation*}
        \rightcol
            \centering
            \underline{Intensitée de Champ Magnétique}
            \begin{equation*}
                \mfield \rightarrow [\SI{}{\ampere\per\meter}]
            \end{equation*}
            \underline{Densitée de Flux Magnétique}
            \begin{equation*}
                \mflux \rightarrow [\SI{}{\weber\per\meter\squared} \;\; ou \;\; \SI{}{\tesla}]
            \end{equation*}
            \underline{Relation}
            \begin{equation*}
                \mflux = \mu \mfield
            \end{equation*}
    \end{twocolumns}
\end{frame}

\begin{frame}{Polarization Density}
    \begin{twocolumns}[0.6]
        \leftcol
        La densitée de polarisation représente la \textbf{contribution des dipôles électriques} du matériau qui \textbf{s'alignent} en réponse à un champs $\efield$
        \begin{equation*}
            \eflux = \varepsilon \efield \hspace{1em} \rightarrow \hspace{1em} \eflux = \varepsilon_0 \efield + \pola
        \end{equation*}
        \vspace{10pt}
        \begin{equation*}
            \begin{aligned}
                \pola &=  \left(\varepsilon - \varepsilon_0 \right) \efield\\
                &= \varepsilon_0 \suse \efield
            \end{aligned}
        \end{equation*}
        \rightcol
    \makefigure[1.0][0.6][]{level-7/polarization}
    \end{twocolumns}
    \centering
    $\pola$ représente également l'énergie électrique que le matériau accumule.
\end{frame}

\begin{frame}{Microwave oven}
    \begin{twocolumns}[0.6]
        \leftcol
            \makefigure[1.0][0.6][]{level-7/microwave-oven}
        \rightcol
            \makefigure[1.0][0.6][]{level-7/polarization}
    \end{twocolumns}
    \centering
    \vspace{-20pt}
    Un micro-onde réchauffe en faisant osciller la $\pola$ des molécules d'eau. A cause d'un délais entre $\pola$ et la position de la molécule, un effet de friction survient, créant de la chaleur. 
\end{frame}

\begin{frame}{Polarization relation to $\varepsilon$}
        \begin{twocolumns}[0.4]
        \leftcol
            Permittivitée Complexe:
            \begin{equation*}
                \varepsilon = \epsilon_1 + j \epsilon_2
            \end{equation*}
            \vspace{5pt}
            \begin{equation*}
                \begin{aligned}
                    \varepsilon_1 &: \text{Ability to store electric energy}\\
                    \varepsilon_2 &: \text{Dielectric damping loss}\\
                \end{aligned}
            \end{equation*}
        \rightcol
            %https://www.allaboutcircuits.com/technical-articles/introduction-to-dielectric-loss-in-transmission-lines/
            \makefigure[0.9][0.9][]{level-7/polarization-graph}
        \end{twocolumns}
\end{frame}


\begin{frame}{Effective Complex Permittivity}
    \begin{twocolumns}[0.5]
     \leftcol
         Conversion to phasor form:
        \begin{equation*}
            \begin{aligned}
                \nabv \times \mfield &= \ddt{\eflux} + \ecurrent \\
                &= j\omega \eflux + \sigma \efield\\
                &= j\omega \varepsilon \efield + \sigma \efield\\
            \end{aligned}
        \end{equation*}
            Developpement:
        \begin{equation*}
            \begin{aligned}
                j\omega \varepsilon \efield + \sigma \efield &= j\omega \left(\epsilon_1 - j \epsilon_2\right)\efield + \sigma \efield \\
                &= j\omega \epsilon_1 \efield +  \left( \sigma + \omega \epsilon_2 \right)\efield \\
                &= j\omega \left(\epsilon_1 - j \epsilon_2 - j \frac{\sigma}{\omega}\right)\efield\\
            \end{aligned}
        \end{equation*}
    \rightcol
        \begin{equation*}
            \begin{aligned}
                \varepsilon_1 &: \text{Ability to store electric energy}\\
                \varepsilon_2 &: \text{Dielectric damping loss}\\
                \sigma &: \text{Conductivity loss}
            \end{aligned}
        \end{equation*}
    \end{twocolumns}
\end{frame}

\begin{frame}{Effective Complex Permittivity}
    \centering
    \underline{Effective Complex Relative Permittivity}
    \begin{equation}
        \varepsilon_e = \left(\epsilon_1 - j\epsilon_2 - j\frac{\sigma}{\omega}\right)\frac{1}{\varepsilon_0}
    \end{equation}
    This term describe the properties of \textbf{non-homogeneous} materials. It can be seen as an average of all the individual relative permittivities.

\end{frame}

\begin{frame}{Effective refractive index}
    Indice de refraction effectif?
\end{frame}

