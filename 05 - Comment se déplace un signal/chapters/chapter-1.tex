%!TEX root = ../presentation.tex 

\section[Level 1]{Surface Ripple [20min]}
\begin{frame}{Introduction}
 Introduction des mathematiques et équation fondamentales à l'electromagnetisme
\end{frame}


\subsection[10min - Max]{EM Fields \rom{1}}
\begin{frame}{Plan}
    \begin{makelist}[\small][1.5]
        \icon{\faCheck} & Champ Vectoriel\\
        \icon[red]{\faTimes} & Divergente, Rotationnelle\\
        \icon[red]{\faTimes} & Regle de la main droite \\
        \icon[red]{\faTimes} & Equation de Maxwell
    \end{makelist}
\end{frame}

\begin{frame}{Champ Vectoriel}
\end{frame}

\begin{frame}{Divergente}

\end{frame}

\begin{frame}{Gauss's Law}
    \centering{\underline{Gauss's Law}}
    \vspace{-8pt}
    \begin{equation}
        \nabv \cdot \efield = \frac{\rho}{\epsilon_0}
        \hspace{1em} \Longleftrightarrow \hspace{1.0em}
        \nabv \cdot \eflux = \rho v
    \end{equation}
    \centering{\underline{Gauss's Law For Magnetism}}
    \vspace{-8pt}
    \begin{equation}
        \nabv \cdot \mflux = 0
    \end{equation}
\end{frame}

\begin{frame}{Curl}
\begin{twocolumns}[0.5]
    \leftcol
        \makefigure[1.0][0.8][Cyclone Analogy]{level-1/cyclone}
    \rightcol
        \makefigure[1.0][0.6][Rotating Ball Analogy]{level-1/curl-ball}
 \end{twocolumns}
\end{frame}

\begin{frame}{Faraday's Law of Induction}
    \centering{\underline{Faraday's Law of Induction}}
    \vspace{-10pt}
    \begin{equation}
        \nabv \times \efield = -\ddt{\mflux}
    \end{equation}
\end{frame}

\begin{frame}{Ampère-Maxwell's Law}
    \centering{\underline{Ampère-Maxwell's Law}}
    \vspace{-10pt}
    \begin{equation}
        \nabv \times \mfield = \ddt{\eflux} + \ecurrent
        \hspace{1em} \Longleftrightarrow \hspace{1.0em}
        \nabv \times \mflux = \mu_0 \epsilon_0 \ddt{\efield} + \mu_0 \ecurrent
    \end{equation}
\end{frame}

\begin{frame}{Maxwell Equation}
    \begin{twocolumns}[0.3]
    \leftcol
        \makefigure[1.0][0.8][James Clark Maxwell]{level-1/james-maxwell}
    \rightcol
        \centering{\underline{Gauss's Law}}
        \vspace{-8pt}
        \begin{equation}
                \nabv \cdot \efield = \frac{\rho}{\epsilon_0}
                \hspace{1em} \Longleftrightarrow \hspace{1.0em}
                \nabv \cdot \eflux = \rho v
        \end{equation}
        \centering{\underline{Gauss's Law For Magnetism}}
        \vspace{-8pt}
        \begin{equation}
            \nabv \cdot \mflux = 0
        \end{equation}
        \centering{\underline{Faraday's Law of Induction}}
        \vspace{-10pt}
        \begin{equation}
            \nabv \times \efield = -\ddt{\mflux}
        \end{equation}
        \centering{\underline{Ampère-Maxwell's Law}}
        \vspace{-10pt}
        \begin{equation}
            \nabv \times \mfield = \ddt{\eflux} + \ecurrent
            \hspace{1em} \Longleftrightarrow \hspace{1.0em}
            \nabv \times \mflux = \mu_0 \epsilon_0 \ddt{\efield} + \mu_0 \ecurrent
        \end{equation}
 \end{twocolumns}
\end{frame}

\subsection[1min - Max]{Superposition \rom{1}}
\begin{frame}{Plan}
    \begin{makelist}[\small][1.5]
        \icon[red]{\faTimes} & Équation Linéaire\\
        \icon[red]{\faTimes} & Addition de Signaux
    \end{makelist}
\end{frame}

\begin{frame}{Green Theorem}
    \begin{twocolumns}[0.5]
        \leftcol
            \makefigure[1.0][0.7][]{level-1/green-theorem-meme}
        \rightcol
            \makefigure[1.0][0.7][Inductor Magnetic field]{level-1/inductor-magnetic-field}
    \end{twocolumns}
    \vspace{-15pt}
    Quiz: Dans quel sens vont les electrons dans la bobine?
\end{frame}

\subsection[4min - Max]{Charge Movement}
\begin{frame}{Plan}
    \begin{makelist}[\small][1.5]
        \icon[red]{\faTimes} & Comment les Electrons bougent\\
        \icon[red]{\faTimes} & Propriété materiaux
    \end{makelist}
\end{frame}


\subsection[3min - Max]{Passive Components \rom{1}}
\begin{frame}{Plan}
    \begin{makelist}[\small][1.5]
        \icon[red]{\faTimes} & Resistance\\
        \icon[red]{\faTimes} & Condensateur \\
        \icon[red]{\faTimes} & Inducteur
    \end{makelist}
\end{frame}