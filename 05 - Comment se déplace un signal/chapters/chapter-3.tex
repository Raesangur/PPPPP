%!TEX root = ../presentation.tex

\section[Level 3]{Impedance \& Reflection [20min - 1h10]}

\introbackground
\begin{frame}[plain, label=intro-level-3]
    \centering
    \Large
    \textcolor{white}{\textbf{Sujets Abordés dans la Section:}}\\
    \vspace{24pt}
    \begin{tabular}{c l}
        \textcolor{UDSgreenFierte}{\faEye}
            & \textcolor{white}{\textbf{Impedance de Source} \& \textbf{GPIO}}\\
            [0.3em]
        \textcolor{UDSgreenFierte}{\faEye}
            & \textcolor{white}{Suite sur les propriété de \textbf{Matériaux} (paramètres complexes)}\\
            [0.3em]
        \textcolor{UDSgreenFierte}{\faEye}
            & \textcolor{white}{Introduction sur les \textbf{Impedance \& Lignes de transmission}}\\
            [0.3em]
        \textcolor{UDSgreenFierte}{\faHubspot}
            & \textcolor{white}{Introduction sur la \textbf{Réflection \& Transmission}}\\
            [0.3em]
    \end{tabular}
\end{frame}
\defaultbackground


\subsection[5min-Pascal]{Signal Source \rom{2}}
\pascalbackground
\begin{frame}{Source vs Sink}
    \begin{center}
        \textit{Où va le courant?}
    \end{center}

    \begin{twocolumns}
        \leftcol
        \begin{center}
            \textbf{Current Source}
        \end{center}
        \rightcol
        \begin{center}
            \textbf{Current Sink}
        \end{center}
    \end{twocolumns}
    \vfill
    \begin{twocolumns}
        \leftcol
        \ctikzset{bipoles/resistor/height=0.2}
        \ctikzset{bipoles/resistor/width=0.5}
        \ctikzset{bipoles/diode/height=0.35}
        \ctikzset{bipoles/diode/width=0.2}
        \begin{maketikzfigure}[1][0.45]
            \draw (0, 0) node [dipchip,
                num pins = 12,
                hide numbers,
                external pins width=0.3,
                external pad fraction=4,
                ](IC){};
            ]
            \only<1>{
                \node [left, font=\tiny] at (IC.bpin 9) {OUT};
            }

            \only<1>{
                \draw (IC.pin 9) to 
                    [short] ($(IC.pin 9) + (0.5, 0)$) to
                    [R] ($(IC.pin 9) + (1.5, 0)$) to
                    [empty led] ($(IC.pin 9) + (2.5, 0)$) to
                    [short] ($(IC.pin 7) + (2.5, 0)$) node [ground]{};
            }
            \only<2->{
                \draw (IC.pin 9) to 
                    [short] ($(IC.pin 9) + (0.5, 0)$) to
                    [R] ($(IC.pin 9) + (1.5, 0)$) to
                    [full led, color=accent] ($(IC.pin 9) + (2.5, 0)$) to
                    [short] ($(IC.pin 7) + (2.5, 0)$) node [ground]{};

                \draw [->, thick, accent]
                    ($(IC.pin 8) + (0.5, 0)$) to [in=90, out=0]
                    ($(IC.pin 7) + (2, 0)$);
            }
        \end{maketikzfigure}

        \rightcol
        \ctikzset{bipoles/resistor/height=0.2}
        \ctikzset{bipoles/resistor/width=0.5}
        \ctikzset{bipoles/diode/height=0.35}
        \ctikzset{bipoles/diode/width=0.2}
        \begin{maketikzfigure}[1][0.45]
            \draw (0, 0) node [dipchip,
                num pins = 12,
                hide numbers,
                external pins width=0.3,
                external pad fraction=4,
                ](IC){};
            ]
            \only<1>{
                \node [left, font=\tiny] at (IC.bpin 9) {OUT};
            }

            \only<1>{
                \draw (IC.pin 9) to 
                    [short] ($(IC.pin 9) + (0.5, 0)$) to
                    [R] ($(IC.pin 9) + (1.5, 0)$) to
                    [empty led, invert] ($(IC.pin 9) + (2.5, 0)$) to
                    [short] ($(IC.pin 12) + (2.5, 0)$) node [vcc]{};
            }
            \only<2->{
                \draw (IC.pin 9) to 
                    [short] ($(IC.pin 9) + (0.5, 0)$) to
                    [R] ($(IC.pin 9) + (1.5, 0)$) to
                    [full led, invert, color=accent] ($(IC.pin 9) + (2.5, 0)$) to
                    [short] ($(IC.pin 12) + (2.5, 0)$) node [vcc]{};

                \draw [->, thick, accent]
                    ($(IC.pin 12) + (2, 0)$) to [in=0, out=-90]
                    ($(IC.pin 10) + (0.5, 0)$);
            }
        \end{maketikzfigure}
    \end{twocolumns}
\end{frame}

\begin{frame}{Fonctionnement d'un GPIO}
    \begin{twocolumns}
        \leftcol
        \begin{center}
            \textbf{Push-Pull}
        \end{center}
        \rightcol
        \begin{center}
            \textbf{Open-Drain}
        \end{center}
    \end{twocolumns}
    \vfill
    \begin{twocolumns}
        \leftcol
        \begin{maketikzfigure}[1][0.66]
            \only<1, 4->{
                \draw [thick]
                    (0, 0) node (in)  {IN}
                    (3.5, 0) node (out) {OUT};
            }
            \only<2>{
                \draw [thick]
                    (0, 0) node[color=accent]  (in)  {IN}
                    (3.5, 0) node[color=accent2] (out) {OUT};
            }
            \only<3>{
                \draw [thick]
                    (0, 0) node[color=accent2] (in)  {IN}
                    (3.5, 0) node[color=accent]  (out) {OUT};
            }

            \only<1, 4->{
                \draw [thick]
                    (2, 1.5) node[pigfete, bodydiode] (pmos){};
                \draw [thick]
                    (2, -1.5) node[nigfete, bodydiode] (nmos){};
            }
            \only<2>{
                \draw [thick]
                    (2, 1.5) node[pigfete, bodydiode, color=accent2] (pmos){};
                \draw [thick]
                    (2, -1.5) node[nigfete, bodydiode, color=accent] (nmos){};
            }
            \only<3>{
                \draw [thick]
                    (2, 1.5) node[pigfete, bodydiode, color=accent] (pmos){};
                \draw [thick]
                    (2, -1.5) node[nigfete, bodydiode, color=accent2] (nmos){};
            }

            \draw [thick]
                (in.east) to [short] (0.5, 0) to [short] (0.5, 0 -| pmos.G) to (pmos.G);
            \draw [thick]
                (in.east) to [short] (0.5, 0) to [short] (0.5, 0 -| nmos.G) to (nmos.G);

            \draw [thick]
                (pmos.S) to [short] ($(pmos.S) + (0, 0.5)$) node[vcc]{};
            \draw [thick]
                (nmos.S) to [short] ($(nmos.S) - (0, 0.5)$) node[ground]{};

            \draw [thick]
                (pmos.D) to [short] (nmos.D);
            \draw [thick]
                (0, 0 -| pmos.D) to [short] (out.west);

            \only<2> {
                \draw[->, very thick, accent2] (out.south) to
                [in=90, out=180] (nmos.S -| 2.5, 0);
            }
            \only<3> {
                \draw[->, very thick, accent] (pmos.S -| 2.5, 0) to
                [in=180, out=-90] (out.north);
            }
        \end{maketikzfigure}
        \rightcol
        \ctikzset{bipoles/resistor/height=0.2}
        \ctikzset{bipoles/resistor/width=0.5}
        \begin{maketikzfigure}[1][0.66]
            \only<1-4>{
                \draw [thick]
                    (2, 0) node[nigfete, bodydiode] (nmos){};
            }
            \only<5>{
                \draw [thick]
                    (2, 0) node[nigfete, bodydiode, color=accent] (nmos){};
            }
            \only<6>{
                \draw [thick]
                    (2, 0) node[nigfete, bodydiode, color=accent2] (nmos){};
            }

            \only<1-4>{
                \draw [thick]
                    (nmos.G -| 0, 0) node (in)  {IN}
                    (5, 1) node (out) {OUT};
            }
            \only<5>{
                \draw [thick]
                    (nmos.G -| 0, 0) node[color=accent]  (in)  {IN}
                    (5, 1) node[color=accent2] (out) {OUT};
            }
            \only<6>{
                \draw [thick]
                    (nmos.G -| 0, 0) node[color=accent2] (in)  {IN}
                    (5, 1) node[color=accent]  (out) {OUT};
            }

            \draw [thick]
                (2.6, 3) node[vcc]{} to [R] (2.6, 1);

            \draw [thick]
                (in.east) to (nmos.G);

            \draw [thick]
                (nmos.S) to [short] ($(nmos.S) - (0, 0.5)$) node[ground]{};

            \draw [thick]
                (nmos.D) to [short] (0, 1 -| nmos.D) to [short] (out.west);

            \only<5> {
                \draw[-, very thick, accent2] (out.north) to
                    [in=90, out=180] (3, 0.9)
                    [in=90, out=-90] (nmos.S -| 2.5, 0);

                \draw[->, very thick, accent2] (3, 3) to
                [in=90, out=-90] (3, 0.9) to
                [in=90, out=-90] ($(nmos.S -| 2.5, 0) - (0, 0.5)$);
            }
            \only<6> {
                \draw[->, very thick, accent] (3, 3) to
                [in=180, out=-90] (out.north);
            }
        \end{maketikzfigure}
    \end{twocolumns}
\end{frame}

\subsection[5min-Max]{Materiaux \rom{2}}
\maxbackground
\begin{comment}
\begin{frame}{Plan}
    \begin{makelist}[\small][1.5]
        \icon[red]{\faTimes} & Complex $\epsilon$ \& $\mu$ \\
        \icon[red]{\faTimes} & Complex Refractive index\\
        \icon[red]{\faTimes} & Losses in dB\\
    \end{makelist}
\end{frame}
\end{comment}
\begin{frame}{Rappel}
    \begin{twocolumns}[0.4]
        \leftcol
            Electric Permittivity [F/m]:
            \begin{equation*}
                \varepsilon =\varepsilon_r \varepsilon_0
            \end{equation*}
            Magnetic Permeabililty [H/m]:
            \begin{equation*}
                \mu =\mu_r \mu_0
            \end{equation*}
            Refractive Index:
            \begin{equation*}
                n =\frac{\sqrt{\varepsilon \mu}}{\sqrt{\varepsilon_0 \mu_0}} = \sqrt{\varepsilon_r \mu_r}
            \end{equation*}
        \rightcol
            Nous avons initialement considéré ces valeurs comme étant des nombre réel $\mathbb{R}$.\\
            \vspace{10pt}
            Cela fonctionne tres bien pour definir des matériaux \textbf{sans pertes}.\\
            \vspace{10pt}
            Par contre, dans la vrai vie, ce sont des nombres complexe $\mathbb{C}$. Cela permet de définir des matériaux \textbf{avec pertes}
    \end{twocolumns}
\end{frame}

\begin{frame}{Matériaux Complexes}
        \begin{twocolumns}[0.6]
        \leftcol
            Indice de Refraction Complexe:
            \begin{equation*}
                \tilde{n} = n + jk
            \end{equation*}
            Permittivitée Complexe:
            \begin{equation*}
                \varepsilon = \epsilon_1 + j \epsilon_2 = (n+jk)^2
            \end{equation*}
            \vspace{5pt}
            \begin{equation*}
                \epsilon_1 = n^2 - k^2
            \end{equation*}
            \begin{equation*}
                \epsilon_2 = 2nk
            \end{equation*}
        \rightcol
           Indice de Refraction:
            \begin{equation*}
                n = \sqrt{\frac{\sqrt{\epsilon_1^2 + \epsilon_2^2}+\epsilon_1}{2}}
            \end{equation*}
           Coefficient d'Extinction:
            \begin{equation*}
                k = \sqrt{\frac{\sqrt{\epsilon_1^2 + \epsilon_2^2}-\epsilon_1}{2}}
            \end{equation*}
        \end{twocolumns}
\end{frame}

\begin{frame}{Coefficient d'Extinction}
    mettre graphique sinusoidale ammortie
    \begin{equation}
        T = e^{-kl}
    \end{equation}
    a valider
\end{frame}


\subsection[5min-Pascal]{Impédances \rom{1}}
\pascalbackground
\begin{frame}{Plan}
    \begin{makelist}[\small][1.5]
        \icon[red]{\faTimes} & PPPPP2\\
        \icon[red]{\faTimes} & Impedance dans le plan complexe\\
        \icon[red]{\faTimes} & Rappel qu'on ignore la conductance G.\\ %On va la resortir plus tard dans la presentation
        \icon[red]{\faTimes} & Propagation Constant (p.56)
    \end{makelist}
\end{frame}

\begin{frame}{Qu'est-ce qu'une impédance?}
    \begin{twocolumns}
        \leftcol
        \begin{itemize}
            \item \textit{J'ai un PPPPP complet là-dessus!}
            \item Rappel
            \bigskip
            \item Résistance complexe, dénoté $Z$, en $\SI{}{\ohm}$
            \item Résistance électrique à une certaine fréquence
            \item Composé de:
            \begin{itemize}
                \item Résistance \textcolor{red}{($R$)}
                \item Réactance Inductive \textcolor{orange}{($X_L$)}
                \item Réactance Capacitive \textcolor{blue}{($X_C$)}
            \end{itemize}
        \end{itemize}

        \rightcol
        \begin{center}
            \Large
            $V = ZI$\\
            \vspace{12pt}
            \textcolor{orange}{
                $X_L = 2\pi f L$\\
            }
            \vspace{12pt}
            \textcolor{blue}{
                $X_C = \dfrac{1}{2\pi f C}$
            }
        \end{center}
    \end{twocolumns}
\end{frame}

\begin{frame}{Graphique d'impédance}
    \begin{twocolumns}
        \leftcol
        \begin{center}
            \only<2->{
                $f = \SI{2}{\mega\hertz}$\\
            }
        \end{center}
        \begin{columns}
            \begin{column}{0.5\textwidth}
                \only<2->{
                \begin{center}
                \textcolor{orange}{
                    $X_L = 2 \pi f L = \SI{126}{\ohm}$\\
                }
                \end{center}
                }
            \end{column}
            \begin{column}{0.5\textwidth}
                \only<3->{
                \begin{center}
                \textcolor{blue}{
                $X_C = \dfrac{1}{2 \pi f C} = \SI{80}{\ohm}$\\
                }
                \end{center}
                }
            \end{column}
        \end{columns}

        \vspace{24pt}

        \begin{maketikzfigure}[1][0.4][american voltages, american currents, cute inductors]
            \draw[thick] (0, 2) to [sV=$V$, l_=$\SI{2}{\mega\hertz}$] (0, -2);

            \only<1>{
                \draw[thick] (0, 2) to [short] (6, 2);
            }
            \only<2->{
                \draw[thick] (0, 2) to [short] (2, 2)
                    to [L, l_=$\SI{10}{\micro\henry}$, color=orange] (4, 2)
                    to [short] (6, 2);
            }
            \only<3>{
                \draw[thick] (4.5, 2)
                    to [C, l_=$\SI{100}{\nano\farad}$, color=blue] (4.5, -2);
            }


            \draw[thick] (6, 2) to
                [R, l=$\SI{220}{\ohm}$, color=red] (6, -2)
                to [short] (0, -2);

        \end{maketikzfigure}

        \rightcol

        \begin{center}
            \only<2->{
                $Z = \sqrt{X^2 + R^2}$\\
                \vspace{12pt}
            }
            \only<2>{
                $Z = \sqrt{126^2 + 220^2} = \SI{250}{\ohm}$\\
            }
            \only<3->{
                $Z = \sqrt{(126 - 80)^2 + 220^2} = \SI{225}{\ohm}$\\
            }
        \end{center}

        \begin{adjustbox}{width=\textwidth, height=0.8\textheight, keepaspectratio}%
            \begin{tikzpicture}
            \only<1>{
                \begin{axis}[
                    axis x line=middle,
                    axis y line=none,
                    xmin=0, xmax=300,
                    ymin=-200, ymax=200,
                    xlabel = \(R (\SI{}{\ohm})\),
                    xtick={0, 100, 200, 300},
                    ytick=\empty,
                    enlargelimits=false,
                    clip=false
                ]
            }
            \only<2->{
                \begin{axis}[
                    axis x line=middle,
                    axis y line=left,
                    xmin=0, xmax=300,
                    ymin=-200, ymax=200,
                    xlabel = \(R (\SI{}{\ohm})\),
                    ylabel = \(X (\SI{}{\ohm})\),
                    xtick={0, 100, 200, 300},
                    ytick={-200, -100, 0, 100, 200},
                    enlargelimits=false,
                    clip=false
                ]
            }
            \draw[->, very thick, red]    (axis cs:0,0) -- (axis cs:220,0);
            \only<2->{
            \draw[->, very thick, orange] (axis cs:0,0) -- (axis cs:0,126);
            }

            \only<3->{
            \draw[->, very thick, blue]   (axis cs:0,0) -- (axis cs:0,-80);
            }


            \only<2>{
            \draw[-, accent, dashed] (axis cs:0,126) -- (axis cs:220,126);
            \draw[-, accent, dashed] (axis cs:220,0) -- (axis cs:220,126);

            \draw[->, very thick, accent] (axis cs:0,0) -- (axis cs:220,126) node[midway, above, sloped] () {$Z = \SI{250}{\ohm}$};
            }

            \only<3>{
            \draw[-, accent, dashed] (axis cs:220,0) -- (axis cs:220,46);

            \draw[->, very thick, accent] (axis cs:0,0) -- (axis cs:220, 46) node[midway, above, sloped] () {$Z = \SI{225}{\ohm}$};
            }
            \end{axis}
            \end{tikzpicture}
        \end{adjustbox}
    \end{twocolumns}
\end{frame}

\begin{frame}{Quote}
    \begin{makelist}[\small][1.5]
        \icon[red]{\faTimes} & L'Électricité prend le chemin avec la plus petite \textbf{Resistance}\\
        \icon[green]{\faCheck} & L'Électricité prend le chemin avec la plus petite \textbf{Impedance}\\
    \end{makelist}
\end{frame}

\subsection[5min-Pascal]{Réflection}
\pascalbackground
\begin{frame}{Plan}
    \begin{makelist}[\small][1.5]
        \icon[red]{\faTimes} & Bounce Diagram\\
        \icon[red]{\faTimes} & Impedance Mismatch\\
        \icon[red]{\faTimes} & Item 3
    \end{makelist}
\end{frame}
%Bounce Diagrams

\subsection[5min-Pascal]{Current loops \rom{2}}
\pascalbackground
\begin{frame}{Plan}
    \begin{makelist}[\small][1.5]
        \icon[red]{\faTimes} & GND Loop avec cable(Ou on place ca apres la section noise?)\\
        \icon[red]{\faTimes} & Frequency dependant loop\\
        \icon[red]{\faTimes} & Item 3
    \end{makelist}
\end{frame}

\subsection[5min-Pascal]{Transmission Line \rom{1}}
\pascalbackground
\begin{frame}{Plan}
    \begin{makelist}[\small][1.5]
        \icon[red]{\faTimes} & Equation de base\\
        \icon[red]{\faTimes} & Pertes en dB (exponential decay)\\
    \end{makelist}
\end{frame}
