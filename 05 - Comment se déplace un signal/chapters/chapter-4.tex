%!TEX root = ../presentation.tex

\section[Level 4]{Noise [27min - 1h37]}

\introbackground
\begin{frame}[plain, label=intro-level-4]
    \centering
    \Large
    \textcolor{white}{\textbf{Sujets Abordés dans la Section:}}\\
    \vspace{24pt}
    \begin{tabular}{c l}
        \textcolor{UDSgreenFierte}{\faEye}
            & \textcolor{white}{Introduction sur les \textbf{Decibels} \& \textbf{GPIO}}\\
            [0.3em]
        \textcolor{UDSgreenFierte}{\faEye}
            & \textcolor{white}{Explication de différent type de \textbf{Bruit Parasites} dans les circuits}\\
            [0.3em]
        \textcolor{UDSgreenFierte}{\faEye}
            & \textcolor{white}{Quantifier l'impact des \textbf{Effets Parasites}}\\
            [0.3em]
        \textcolor{UDSgreenFierte}{\faHubspot}
            & \textcolor{white}{Comment lire un \textbf{Eye Diagram}}\\
            [0.3em]
    \end{tabular}
\end{frame}
\defaultbackground

\subsection[5min-Max]{Decibel Review}
\maxbackground
\begin{frame}{Plan}
    \begin{makelist}[\small][1.5]
        \icon[red]{\faTimes} & Pourquoi c'est important\\
        \icon[red]{\faTimes} & Analogie des dB avec le stock market\\
        \icon[red]{\faTimes} & Item 3
    \end{makelist}
\end{frame}

\begin{frame}{Signal Power}
    \centering{\underline{Signal Power in dB}}
    \begin{equation}
            P_{Signal|\unit{dB}}=10 \log{\frac{P_{\text{Signal}|\unit{W}}}{1\unit{W}}}
    \end{equation}
    \vspace{10pt}\\
    \centering{\underline{Signal Power in dBm}}
    \begin{equation}
            P_{Signal|\unit{dBm}}=10 \log{\frac{P_{\text{Signal}|\unit{mW}}}{1\unit{mW}}}
    \end{equation}
\end{frame}

\begin{frame}{System Gain}
    \centering{\underline{Voltage Gain}}
    \begin{equation}
            A_{V|\unit{dB}}=20 \log{\frac{V_{out}}{V_{in}}}
            \hspace{2.0em} \Longleftrightarrow \hspace{2.0em}
            \frac{V_{out}}{V_{in}} = 10^{\frac{A_{v}}{20}} 
    \end{equation}
    \vspace{10pt}\\
    \centering{\underline{Power Gain}}
    \begin{equation}
            A_{V|\unit{dB}}=10 \log{\frac{P_{out}}{P_{in}}}
            \hspace{2.0em} \Longleftrightarrow \hspace{2.0em}
            \frac{P_{out}}{P_{in}} = 10^{\frac{A_{P}}{10}} 
    \end{equation}
\end{frame}


\subsection[4min-Max]{Signal Source \rom{3}}
\maxbackground
\begin{frame}{Plan}
    \begin{makelist}[\small][1.5]
        \icon[red]{\faTimes} & Random Noise Source\\
        \icon[red]{\faTimes} & Noise Power\\
        \icon[red]{\faTimes} & Source of noise in a circuit
    \end{makelist}
\end{frame}

\begin{frame}{Noise Source}
    \begin{makelist}[\small][1.5]
        \icon[red]{\faTimes} & Thermal Noise\\
        \icon[red]{\faTimes} & Environmental Noise \\
        \icon[red]{\faTimes} & Flicker Noise
    \end{makelist}
\end{frame}

\begin{frame}{Bruit Thermique - Resistance}
    \begin{twocolumns}[0.5]
        \leftcol
        Johnson-Nyquist Noise Equation:
        \begin{equation}
            V_{rms}= \sqrt{4k_BTR\Delta f}
        \end{equation}
        $k_b = 1.38\times 10^{-23}$: Boltzmann Constant [J]\\
        $T$: Temperature [K]\\
        $\Delta f$: Frequency Range\\
        $R$: Resistance [Ohm]\\
        \rightcol
        \makefigure[0.8][0.8][]{level-4/johnson-circuit}
    \end{twocolumns}
\end{frame}

\begin{frame}{Bruit Thermique - Thermique}
    %\vspace{20pt}
    \centering
    \underline{Valeurs pour 300K:}
    \begin{equation}
        V_{rms}= 0.13\cdot\sqrt{R\Delta f} \hspace{2.0em} [\SI{}{\nano\volt}]
    \end{equation}
    \makefigure[1.0][1.0]{level-4/johnson-nyquist-band}
\end{frame}

\begin{frame}{Bruit Thermique - Resistance}
    \makefigure[0.8][0.8][]{level-4/RC-noise_simpler}
\end{frame}

\begin{frame}{Bruit Thermique - Autres}
    \begin{twocolumns}[0.5]
        \leftcol
        \centering
        \underline{Bruit Condensateur:}
        \begin{equation}
            V_{rms}= \sqrt{\frac{k_BT}{C}}
        \end{equation}
        \rightcol
        \centering
        \underline{Bruit Inducteur:}
        \begin{equation}
            \overline{I_{n}}= \sqrt{\frac{k_BT}{L}}
        \end{equation}
    \end{twocolumns}
\end{frame}

\begin{frame}{Bruit Thermique - Applications}
    \maketable{thermal-noise}
\end{frame}

%Radio Noise
%https://www.itu.int/rec/R-REC-P.372/en
\begin{frame}{Bruits Environnementaux}
    \begin{makelist}[\small][1.5]
        \icon{\faBolt} & \textbf{Atmospheriques:} Éclairs, Effets Corona, etc... ($\approx -120 \text{ to} -80 \SI{}{\dbm\per\hertz}$, $<20\SI{}{\mega\hertz}$)\\
        \icon{\faHouzz} & \textbf{Industriel:} Lignes HV, Lampes Fluorescentes, etc... ($\approx -110 \text{ to} -80 \SI{}{\dbm\per\hertz}$, $20-300\SI{}{\mega\hertz}$)\\
        \icon{\faSun} & \textbf{Solaire:} Vents solaires\\
        \icon{\faMeteor} & \textbf{Cosmique:} Effets colectif des étoiles (8MHz-1.5GHz), CMB, Quasar, Pulsar, Chaos Gods ($\approx -204 \text{ to} -164 \SI{}{\dbm\per\hertz}$, $20-120\SI{}{\mega\hertz}$)\\
        \icon{\faFrown} & \textbf{Psychologique:} Toi qui fait pas tes lectures\\
    \end{makelist}
\end{frame}

\begin{frame}{Autres source de bruit sur un PCB}
    \maketable{pcb-noise}
\end{frame}

\subsection[2min-Max]{Noise Spectrum}
\maxbackground
\begin{frame}{Plan}
    \begin{makelist}[\small][1.5]
        \icon[red]{\faTimes} & Frequency dependant noise power\\
        \icon[red]{\faTimes} & Demo avec type de bruit (red, white, brown, etc..)
    \end{makelist}
\end{frame}

\begin{frame}{Average Power of a Signal}
    \begin{equation}
        P_{n}=\lim_{T \rightarrow \infty}\frac{1}{T}\int_{0}^{T}n^{2}(t) \,dt
   \end{equation}
    \vspace{10pt}\\
    \begin{equation}
        \int_{0}^{\infty}S_{x}(f) \, df = \lim_{T \rightarrow \infty}\frac{1}{T}\int_{0}^{T}x^{2}(t) \, dt
   \end{equation}
\end{frame}

\subsection[3min-Max]{Harmonics \rom{2}}
\maxbackground
\begin{frame}{Plan}
    \begin{makelist}[\small][1.5]
        \icon[red]{\faTimes} & Gauss representation in frequency domain of a sine wave\\
        \icon[red]{\faTimes} & Sinc function\\
        \icon[red]{\faTimes} & Item 3
    \end{makelist}
\end{frame}
%
%Sinc function

\subsection[5min-Max]{Signal to Noise Ratio (SNR)}
\maxbackground
\begin{frame}{Plan}
    \begin{makelist}[\small][1.5]
        \icon[red]{\faTimes} & Why it matters\\
        \icon[red]{\faTimes} & How can you tell the SNR you need\\
        \icon[red]{\faTimes} & Shannon-Hartley Theorem\\
        \icon[red]{\faTimes} & Application: DAC,ADC\\
        \icon[red]{\faTimes} & Application: Example for Voyager 1 Detection \href{https://www.seti.org/detecting-voyager-1-ata}{Link}
    \end{makelist}
\end{frame}

\begin{frame}{Definition}
    \begin{equation}
        SNR = \frac{P_{\text{Signal}|\unit{W}}}{P_{\text{Noise}|\unit{W}}}=P_{\text{Signal}|\unit{dB}} - P_{\text{Noise}|\unit{dB}}
    \end{equation}
\end{frame}

\begin{frame}{Shannon-Hartley}
    \begin{equation}
        C = BW \cdot \log_{2}\left( 1+\frac{P_{\text{Signal}|\unit{W}}}{P_{\text{Noise}|\unit{W}}} \right)
    \end{equation}
    \begin{makelist}[\small][1.5]
        \icon[blue]{\faCloudversify} & C: Canal Capacity [Bit/s]\\
        \icon[blue]{\faChartArea} & BW: Canal Bandwidth [Hz]
    \end{makelist}
\end{frame}
% 
% 
\begin{frame}{Determining SNR}
    Mettre tableau avec resolution et SNR
\end{frame}

\begin{frame}{Common PCB Protocols}
    \maketable{protocol-snr}
\end{frame}

\begin{frame}{(BER) Bit Error Rate}
    \maketable{ber}
\end{frame}

\subsection[5min-Pascal]{Jitter}
\pascalbackground

\begin{frame}{Jitter}
    \begin{twocolumns}
        \leftcol
        \begin{itemize}
            \item Le Jitter est la variation du timing des fronts d'un signal
            \bigskip
            \item Par rapport au signal lui-même
            \item Par rapport à d'autres signaux (clock vs data)
            \bigskip
            \item Période à période
            \item À l'intérieur d'une période
        \end{itemize}
        \rightcol
        \begin{center}
        \begin{adjustbox}{width=\textwidth, height=0.7\textheight, keepaspectratio}%
            \begin{tikzpicture}
                \begin{axis}[
                    axis lines = left,
                    xlabel = \(t\),
                    ylabel = \(V\),
                    ymax = 1.75,
                    xmax = 7.5,
                ]
                \addplot [
                    domain=0:7.5,
                    thick,
                    color=accent,
                ]
                coordinates
                {(0, 0) (1, 0) (1.5, 1) (5, 1) (5.5, 0) (10, 0)};
                \addplot [
                    domain=0:7.5,
                    color=accent,
                    dashed
                ]
                coordinates
                {(0, 0) (0.75, 0) (1.25, 1) (4.75, 1) (5.25, 0) (10, 0)};
                \addplot [
                    domain=0:7.5,
                    color=accent,
                    dashed
                ]
                coordinates
                {(0, 0) (1.05, 0) (1.55, 1) (5.05, 1) (5.55, 0) (10, 0)};
                \addplot [
                    domain=0:7.5,
                    color=accent,
                    dashed
                ]
                coordinates
                {(0, 0) (1.33, 0) (1.83, 1) (5.33, 1) (5.83, 0) (10, 0)};

                \draw[<->, very thick]   (axis cs:1.5, 1.15) -- (axis cs:5, 1.15) node[midway, above] () {$T \pm jitter$};
                \end{axis}
            \end{tikzpicture}
        \end{adjustbox}
        \end{center}
    \end{twocolumns}
\end{frame}

\begin{frame}{Sources de jitter}
    \begin{twocolumns}
        \leftcol
            \begin{itemize}
                \item[] \itemicon{\faThermometerHalf} \textbf{Clock Skew}
                \begin{itemize}
                    \item Signaux qui n'arrive pas en même temps
                \end{itemize}
                \item[] \itemicon{\faWaveSquare} \textbf{Réflexions}
                \begin{itemize}
                    \item Mismatch d'impédance cause un rebondissement
                \end{itemize}
                \item[] \itemicon{\faPlug} \textbf{Power Supply Noise}
                \begin{itemize}
                    \item Affecte le comportement des IC
                \end{itemize}
                \item[] \itemicon{\faWifi} \textbf{Interférence Électromagnétique}
                \begin{itemize}
                    \item Peut causer des mauvaises lectures
                \end{itemize}
                \item[] \itemicon{\faThermometerFull} \textbf{Effets thermiques}
                \begin{itemize}
                    \item Modifie le comportement des IC et les valeurs des composantes
                \end{itemize}
                \item[] \itemicon{\faLevelUp*} \textbf{Time Walk}
                \begin{itemize}
                    \item Temps de montées différents
                \end{itemize}
            \end{itemize}
        \rightcol
            \begin{maketikzfigure}[1][0.25]
                \draw (0, 0) node [dipchip,
                num pins = 4,
                hide numbers,
                external pins width=0.3,
                external pad fraction=4,
                ](IC1){};

                \draw (8, 0) node [dipchip,
                num pins = 4,
                hide numbers,
                external pins width=0.3,
                external pad fraction=4,
                ](IC2){};


                \node [left, font=\tiny] at (IC1.bpin 3) {DATA};
                \node [left, font=\tiny] at (IC1.bpin 4) {CLK};

                \node [right, font=\tiny] at (IC2.bpin 1) {DATA};
                \node [right, font=\tiny] at (IC2.bpin 2) {CLK};

                \draw[thick] (IC1.pin 4) to [short] (IC2.pin 1);
                \draw[thick] (IC1.pin 3) to [short] ($(IC1.pin 3) + (1, 0)$)
                    to [short] ($(IC1.pin 3) + (1, -1)$)
                    to [short] ($(IC2.pin 2) - (1, 1)$)
                    to [short] ($(IC2.pin 2) - (1, 0)$)
                    to [short] (IC2.pin 2);
            \end{maketikzfigure}

            \vspace{-24pt}

            \begin{center}
            \begin{adjustbox}{width=\textwidth, height=0.33\textheight, keepaspectratio}%
                \begin{tikzpicture}
                    \pgfplotsset{
                        width= 360,
                        height= 170
                        }
                    \begin{axis}[
                        axis lines = left,
                        xlabel = \(t\),
                        ylabel = \(V\),
                        ymax = 1.75,
                        xmax = 12,
                    ]
                    \addplot [
                        domain=0:12,
                        thick,
                        color=accent,
                    ]
                    coordinates
                    {(0, 0) (1, 0) (1.5, 1) (5, 1) (5.5, 0) (7.5, 0) (8, 0.5) (10, 0.5) (10.5, 0) (12, 0)};
                    \end{axis}
                \end{tikzpicture}
            \end{adjustbox}
            \end{center}

            \vspace{-24pt}

            \begin{center}
            \begin{adjustbox}{width=\textwidth, height=0.33\textheight, keepaspectratio}%
                \begin{tikzpicture}
                    \pgfplotsset{
                        width= 360,
                        height= 170
                        }
                    \begin{axis}[
                        axis lines = left,
                        xlabel = \(t\),
                        ylabel = \(V\),
                        ymax = 1.75,
                        xmax = 12,
                    ]
                    \addplot [
                        domain=0:12,
                        thick,
                        color=blue,
                    ]
                    coordinates
                    {(0, 0) (1, 0) (1.5, 1) (5, 1) (5.5, 0) (7.5, 0)};

                    \addplot [
                        domain=0:12,
                        thick,
                        color=red,
                    ]
                    coordinates
                    {(0, 0) (1, 0) (2, 1) (5, 1) (6, 0) (7.5, 0)};

                    \draw[thick, dashed, accent] (axis cs:1.5, 0) -- (axis cs:1.5, 1);
                    \draw[thick, dashed, accent] (axis cs:2, 0) -- (axis cs:2, 1);
                    \end{axis}
                \end{tikzpicture}
            \end{adjustbox}
            \end{center}
            \vfill
    \end{twocolumns}
\end{frame}

\begin{frame}{Pourquoi le jitter?}
    \begin{twocolumns}[0.4]
        \leftcol
            \begin{itemize}
                \item Erreurs de lecture de données
                    \begin{itemize}
                        \item BER
                    \end{itemize}
                \item Pertes de clock
                \item Limite des performances
                \item Spéficication des protocoles
            \end{itemize}
        \rightcol
            \begin{center}
            \begin{adjustbox}{width=\textwidth, height=0.7\textheight, keepaspectratio}%
                \begin{tikzpicture}
                    \begin{groupplot}[
                    group style={
                        group size=1 by 2,
                        xlabels at=edge bottom,
                        xticklabels at=edge bottom,
                        vertical sep=10pt
                    },
                    width=8cm,
                    height=5cm,
                    xlabel=t,
                    xmin=0, xmax=12,
                    ymin=0, ymax=2,
                    xtick={0, 2, ..., 12},
                    tickpos=left,
                    ytick align=outside,
                    xtick align=outside,
                    samples=300
                ]
                \nextgroupplot[ylabel=DATA]
                \addplot [color=red] coordinates{(1, 0) (1.5, 1) (4.5, 1) (5, 0) (8, 0) (8.5, 1) (11, 1) (11.5, 0)};
                \coordinate (data1) at (axis cs:1.5, 0.9);
                \coordinate (data2) at (axis cs:8, -0.1);

                \nextgroupplot[ylabel=CLK]
                \addplot [color=blue] coordinates{(1, 0) (1.5, 1) (4.5, 1) (5, 0) (7.5, 0) (8, 1) (10.5, 1) (11, 0)};
                \coordinate (clk1) at (axis cs:1.5, 1.1);
                \coordinate (clk2) at (axis cs:8, 1.1);
                \end{groupplot}

                \draw[dashed, color=accent, thick] (data1) -- (clk1);
                \draw[dashed, color=accent, thick] (data2) -- (clk2);
                \end{tikzpicture}
            \end{adjustbox}
            \end{center}
    \end{twocolumns}
\end{frame}


\subsection[5min-Pascal]{Eye diagram}
\pascalbackground

\begin{frame}{Comment mesurer le jitter?}
    \begin{center}
        % 000
        \begin{adjustbox}{width=0.25\textwidth, height=0.2\textheight, keepaspectratio}%
            \begin{tikzpicture}
                \pgfplotsset{
                    width= 400,
                    height= 200
                    }
                \begin{axis}[
                    axis lines = left,
                    xlabel = \(t\),
                    ylabel = \(V\),
                    ymax = 3.3,
                    xmax = 7,
                ]
                \addplot [
                    domain=0:7,
                    ultra thick,
                    color=accent,
                ]
                coordinates
                {(0, 0) (7, 0)};

                \node[] at (axis cs: 1,   1.65) {\Huge \textbf{0}};
                \node[] at (axis cs: 3.5, 1.65) {\Huge \textbf{0}};
                \node[] at (axis cs: 6,   1.65) {\Huge \textbf{0}};
                \end{axis}
            \end{tikzpicture}
        \end{adjustbox}
        % 001
        \begin{adjustbox}{width=0.25\textwidth, height=0.2\textheight, keepaspectratio}%
            \begin{tikzpicture}
                \pgfplotsset{
                    width= 400,
                    height= 200
                    }
                \begin{axis}[
                    axis lines = left,
                    xlabel = \(t\),
                    ylabel = \(V\),
                    ymax = 3.3,
                    xmax = 7,
                ]
                \addplot [
                    domain=0:7,
                    ultra thick,
                    color=accent,
                ]
                coordinates
                {(0, 0) (4.5, 0) (5, 3.3) (7, 3.3)};

                \only<2->{
                    \addplot [
                        domain=0:7,
                        very thick,
                        dashed,
                        color=accent,
                    ]
                    coordinates
                    {(0, 0) (4.25, 0) (4.75, 3.3) (7, 3.3)};
                    \addplot [
                        domain=0:7,
                        very thick,
                        dashed,
                        color=accent,
                    ]
                    coordinates
                    {(0, 0) (4.6, 0) (5.1, 3.3) (7, 3.3)};
                    \addplot [
                        domain=0:7,
                        very thick,
                        dashed,
                        color=accent,
                    ]
                    coordinates
                    {(0, 0) (4.33, 0) (4.83, 3.3) (7, 3.3)};
                }


                \node[] at (axis cs: 1, 1.65)   {\Huge \textbf{0}};
                \node[] at (axis cs: 3.5, 1.65) {\Huge \textbf{0}};
                \node[] at (axis cs: 6, 1.65)   {\Huge \textbf{1}};
                \end{axis}
            \end{tikzpicture}
        \end{adjustbox}
        % 010
        \begin{adjustbox}{width=0.25\textwidth, height=0.2\textheight, keepaspectratio}%
            \begin{tikzpicture}
                \pgfplotsset{
                    width= 400,
                    height= 200
                    }
                \begin{axis}[
                    axis lines = left,
                    xlabel = \(t\),
                    ylabel = \(V\),
                    ymax = 3.3,
                    xmax = 7,
                ]
                \addplot [
                    domain=0:7,
                    ultra thick,
                    color=accent,
                ]
                coordinates
                {(0, 0) (2, 0) (2.5, 3.3) (4.5, 3.3) (5, 0) (7, 0)};

                \only<2->{
                    \addplot [
                        domain=0:7,
                        very thick,
                        dashed,
                        color=accent,
                    ]
                    coordinates
                    {(0, 0) (1.75, 0) (2.25, 3.3) (4.25, 3.3) (4.75, 0) (7, 0)};
                    \addplot [
                        domain=0:7,
                        very thick,
                        dashed,
                        color=accent,
                    ]
                    coordinates
                    {(0, 0) (2.1, 0) (2.6, 3.3) (4.6, 3.3) (5.1, 0) (7, 0)};
                    \addplot [
                        domain=0:7,
                        very thick,
                        dashed,
                        color=accent,
                    ]
                    coordinates
                    {(0, 0) (1.83, 0) (2.33, 3.3) (4.33, 3.3) (4.83, 0) (7, 0)};
                }

                \node[] at (axis cs: 1, 1.65)   {\Huge \textbf{0}};
                \node[] at (axis cs: 3.5, 1.65) {\Huge \textbf{1}};
                \node[] at (axis cs: 6, 1.65)   {\Huge \textbf{0}};
                \end{axis}
            \end{tikzpicture}
        \end{adjustbox}
        % 011
        \begin{adjustbox}{width=0.25\textwidth, height=0.2\textheight, keepaspectratio}%
            \begin{tikzpicture}
                \pgfplotsset{
                    width= 400,
                    height= 200
                    }
                \begin{axis}[
                    axis lines = left,
                    xlabel = \(t\),
                    ylabel = \(V\),
                    ymax = 3.3,
                    xmax = 7,
                ]
                \addplot [
                    domain=0:7,
                    ultra thick,
                    color=accent,
                ]
                coordinates
                {(0, 0) (2, 0) (2.5, 3.3) (4.5, 3.3) (7, 3.3)};

                \only<2->{
                    \addplot [
                        domain=0:7,
                        very thick,
                        dashed,
                        color=accent,
                    ]
                    coordinates
                    {(0, 0) (1.75, 0) (2.25, 3.3) (7, 3.3)};
                    \addplot [
                        domain=0:7,
                        very thick,
                        dashed,
                        color=accent,
                    ]
                    coordinates
                    {(0, 0) (2.1, 0) (2.6, 3.3) (7, 3.3)};
                    \addplot [
                        domain=0:7,
                        very thick,
                        dashed,
                        color=accent,
                    ]
                    coordinates
                    {(0, 0) (1.83, 0) (2.33, 3.3) (7, 3.3)};
                }

                \node[] at (axis cs: 1, 1.65)   {\Huge \textbf{0}};
                \node[] at (axis cs: 3.5, 1.65) {\Huge \textbf{1}};
                \node[] at (axis cs: 6, 1.65)   {\Huge \textbf{1}};
                \end{axis}
            \end{tikzpicture}
        \end{adjustbox}
        % 100            
        \begin{adjustbox}{width=0.25\textwidth, height=0.2\textheight, keepaspectratio}%
            \begin{tikzpicture}
                \pgfplotsset{
                    width= 400,
                    height= 200
                    }
                \begin{axis}[
                    axis lines = left,
                    xlabel = \(t\),
                    ylabel = \(V\),
                    ymax = 3.3,
                    xmax = 7,
                ]
                \addplot [
                    domain=0:7,
                    ultra thick,
                    color=accent,
                ]
                coordinates
                {(0, 3.3) (2, 3.3) (2.5, 0) (7, 0)};

                \only<2->{
                    \addplot [
                        domain=0:7,
                        very thick,
                        dashed,
                        color=accent,
                    ]
                    coordinates
                    {(0, 3.3) (1.75, 3.3) (2.25, 0) (7, 0)};
                    \addplot [
                        domain=0:7,
                        very thick,
                        dashed,
                        color=accent,
                    ]
                    coordinates
                    {(0, 3.3) (2.1, 3.3) (2.6, 0) (7, 0)};
                    \addplot [
                        domain=0:7,
                        very thick,
                        dashed,
                        color=accent,
                    ]
                    coordinates
                    {(0, 3.3) (1.83, 3.3) (2.33, 0) (7, 0)};
                }

                \node[] at (axis cs: 1, 1.65)   {\Huge \textbf{1}};
                \node[] at (axis cs: 3.5, 1.65) {\Huge \textbf{0}};
                \node[] at (axis cs: 6, 1.65)   {\Huge \textbf{0}};
                \end{axis}
            \end{tikzpicture}
        \end{adjustbox}
        % 101
        \begin{adjustbox}{width=0.25\textwidth, height=0.2\textheight, keepaspectratio}%
            \begin{tikzpicture}
                \pgfplotsset{
                    width= 400,
                    height= 200
                    }
                \begin{axis}[
                    axis lines = left,
                    xlabel = \(t\),
                    ylabel = \(V\),
                    ymax = 3.3,
                    xmax = 7,
                ]
                \addplot [
                    domain=0:7,
                    ultra thick,
                    color=accent,
                ]
                coordinates
                {(0, 3.3) (2, 3.3) (2.5, 0) (4.5, 0) (5, 3.3) (7, 3.3)};

                \only<2->{
                    \addplot [
                        domain=0:7,
                        very thick,
                        dashed,
                        color=accent,
                    ]
                    coordinates
                    {(0, 3.3) (1.75, 3.3) (2.25, 0) (4.25, 0) (4.75, 3.3) (7, 3.3)};
                    \addplot [
                        domain=0:7,
                        very thick,
                        dashed,
                        color=accent,
                    ]
                    coordinates
                    {(0, 3.3) (2.1, 3.3) (2.6, 0) (4.6, 0) (5.1, 3.3) (7, 3.3)};
                    \addplot [
                        domain=0:7,
                        very thick,
                        dashed,
                        color=accent,
                    ]
                    coordinates
                    {(0, 3.3) (1.83, 3.3) (2.33, 0) (4.33, 0) (4.83, 3.3) (7, 3.3)};
                }


                \node[] at (axis cs: 1, 1.65)   {\Huge \textbf{1}};
                \node[] at (axis cs: 3.5, 1.65) {\Huge \textbf{0}};
                \node[] at (axis cs: 6, 1.65)   {\Huge \textbf{1}};
                \end{axis}
            \end{tikzpicture}
        \end{adjustbox}
        % 110
        \begin{adjustbox}{width=0.25\textwidth, height=0.2\textheight, keepaspectratio}%
            \begin{tikzpicture}
                \pgfplotsset{
                    width= 400,
                    height= 200
                    }
                \begin{axis}[
                    axis lines = left,
                    xlabel = \(t\),
                    ylabel = \(V\),
                    ymax = 3.3,
                    xmax = 7,
                ]
                \addplot [
                    domain=0:7,
                    ultra thick,
                    color=accent,
                ]
                coordinates
                {(0, 3.3) (4.5, 3.3) (5, 0) (7, 0)};

                \only<2->{
                    \addplot [
                        domain=0:7,
                        very thick,
                        dashed,
                        color=accent,
                    ]
                    coordinates
                    {(0, 3.3)(4.25, 3.3) (4.75, 0) (7, 0)};
                    \addplot [
                        domain=0:7,
                        very thick,
                        dashed,
                        color=accent,
                    ]
                    coordinates
                    {(0, 3.3) (4.6, 3.3) (5.1, 0) (7, 0)};
                    \addplot [
                        domain=0:7,
                        very thick,
                        dashed,
                        color=accent,
                    ]
                    coordinates
                    {(0, 3.3) (4.33, 3.3) (4.83, 0) (7, 0)};
                }

                \node[] at (axis cs: 1, 1.65)   {\Huge \textbf{1}};
                \node[] at (axis cs: 3.5, 1.65) {\Huge \textbf{1}};
                \node[] at (axis cs: 6, 1.65)   {\Huge \textbf{0}};
                \end{axis}
            \end{tikzpicture}
        \end{adjustbox}
        % 111
        \begin{adjustbox}{width=0.25\textwidth, height=0.2\textheight, keepaspectratio}%
            \begin{tikzpicture}
                \pgfplotsset{
                    width= 400,
                    height= 200
                    }
                \begin{axis}[
                    axis lines = left,
                    xlabel = \(t\),
                    ylabel = \(V\),
                    ymax = 3.3,
                    xmax = 7,
                ]
                \addplot [
                    domain=0:7,
                    ultra thick,
                    color=accent,
                ]
                coordinates
                {(0, 0) (0.01, 3.3) (7, 3.3)};

                \node[] at (axis cs: 1, 1.65)   {\Huge \textbf{1}};
                \node[] at (axis cs: 3.5, 1.65) {\Huge \textbf{1}};
                \node[] at (axis cs: 6, 1.65)   {\Huge \textbf{1}};
                \end{axis}
            \end{tikzpicture}
        \end{adjustbox}
    \end{center}
\end{frame}

\begin{frame}{Eye Diagram}
    \begin{center}
    \begin{adjustbox}{width=\textwidth, height=0.7\textheight, keepaspectratio}%
        \begin{tikzpicture}
            \pgfplotsset{
                width= 400,
                height= 200
                }
            \begin{axis}[
                axis lines = left,
                xlabel = \(t\),
                ylabel = \(V\),
                ymax = 3.3,
                xmax = 7,
            ]
            \addplot [
                domain=0:7,
                ultra thick,
                color=accent,
            ]
            coordinates
            {(0, 3.3) (2, 3.3) (2.5, 0) (4.5, 0) (5, 3.3) (7, 3.3)};

            \addplot [
                domain=0:7,
                ultra thick,
                color=accent,
            ]
            coordinates
            {(0, 0) (2, 0) (2.5, 3.3) (4.5, 3.3) (5, 0) (7, 0)};

            \addplot [
                domain=0:7,
                very thick,
                dashed,
                color=accent,
            ]
            coordinates
            {(0, 0) (1.75, 0) (2.25, 3.3) (4.25, 3.3) (4.75, 0) (7, 0)};
            \addplot [
                domain=0:7,
                very thick,
                dashed,
                color=accent,
            ]
            coordinates
            {(0, 0) (2.1, 0) (2.6, 3.3) (4.6, 3.3) (5.1, 0) (7, 0)};
            \addplot [
                domain=0:7,
                very thick,
                dashed,
                color=accent,
            ]
            coordinates
            {(0, 0) (1.83, 0) (2.33, 3.3) (4.33, 3.3) (4.83, 0) (7, 0)};


            \addplot [
                domain=0:7,
                very thick,
                dashed,
                color=accent,
            ]
            coordinates
            {(0, 3.3) (1.75, 3.3) (2.25, 0) (4.25, 0) (4.75, 3.3) (7, 3.3)};
            \addplot [
                domain=0:7,
                very thick,
                dashed,
                color=accent,
            ]
            coordinates
            {(0, 3.3) (2.1, 3.3) (2.6, 0) (4.6, 0) (5.1, 3.3) (7, 3.3)};
            \addplot [
                domain=0:7,
                very thick,
                dashed,
                color=accent,
            ]
            coordinates
            {(0, 3.3) (1.83, 3.3) (2.33, 0) (4.33, 0) (4.83, 3.3) (7, 3.3)};
            \end{axis}
        \end{tikzpicture}
    \end{adjustbox}
    \end{center}
\end{frame}

\begin{frame}{Anatomie de l'oeil}
    \begin{center}
        \makefigure{level-4/eye-diagram1}
    \end{center}
\end{frame}

\begin{frame}{Exemple}
    \begin{center}
        \makefigure{level-4/axon-500mhz-69ghz}
    \end{center}
\end{frame}

\begin{frame}{Eye Diagram du USB}
    \begin{center}
        \makefigure{level-4/eye-diagram-usb}
    \end{center}
\end{frame}
