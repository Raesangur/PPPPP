%!TEX root = ../presentation.tex 

\section{Bonnes pratiques générales}

\subsection{Définition des besoins}

\begin{frame}{Mise en contexte --- Haptic Board}
    \begin{twocolumns}
        \leftcol
        \begin{itemize}
            \item Dernier board que j'ai design
            \begin{itemize}
                \item A24, pour PMC
            \end{itemize}
            \item Placé au dos de la main au-dessus d'un autre board
            \item Contrôle des éléments d'haptique
            \item Dernière partie d'une intégration de 10 PCBs sur le bras
        \end{itemize}
        \rightcol
        \makefigure[0.8]{haptic-board-layout}
        \makefigure[0.8]{haptic-board-3d}
    \end{twocolumns}
\end{frame}

\begin{frame}{Définition des besoins}
    \begin{twocolumns}
        \leftcol
        \begin{itemize}
            \item     Dresser une liste des fonctionalités
            \item<2-> Dresser des requis techniques quantifiables
            \bigskip
            \only<3->{
            \item Combien en as-tu besoin?
            \item A quel point ils doivent êtres fiables
            \item Comment tu vas les tester?
            \begin{itemize}
                \item Dresser un plan de test!
            \end{itemize}
            \item Envisager la complexité dès le début
            }
        \end{itemize}

        \rightcol
        \begin{itemize}
            \item Activation de 4 solénoïdes
            \only<2->{
            \begin{itemize}
                \item $\SI{5}{\volt}$ @ $\SI{500}{\milli\ampere}$ chaque
            \end{itemize}
            }
            \item Activation de 4 piézo
            \only<2->{
            \begin{itemize}
                \item $\SI{60}{\volt}$ @ $\SI{200}{\hertz}$ AC
            \end{itemize}
            }
            \item Petit
            \only<2->{
            \begin{itemize}
                \item $\SI{25.5}{\milli\meter} \times \SI{45}{\milli\meter}$ 
            \end{itemize}
            }
            \item Ne chauffe pas
            \only<2->{
            \begin{itemize}
                \item $\Delta T_{max} = \SI{40}{\celsius}$
            \end{itemize}
            }
            \item Alimenté $\SI{5}{\volt}$ et/ou $\SI{3.3}{\volt}$
            \item Contrôlé par ${I^2}C$ \& ${I^2}S$
            \item Contraintes de bruit électronique
        \end{itemize}
    \end{twocolumns}
\end{frame}

% TODO : Refaire le diagramme avec TiKz
\begin{frame}{Schéma-Blocs}
    \vspace{-8pt}
    \begin{twocolumns}
        \leftcol
        \begin{itemize}
            \item Faire un schéma-bloc des différentes portions du projet
            \item À inclure dans le schéma final 
        \end{itemize}

        \rightcol
        \begin{itemize}
            \item Général
            \item Power Delivery Network
            \item MCU/CPU/FPGA
            \item Communications
            \item Séquences
        \end{itemize}
    \end{twocolumns}
    \vfill
    \makefigure[1][0.5]{haptic-board-block-diagram}
\end{frame}

% TODO : Refaire le diagramme avec TiKz
\begin{frame}{Principe de Pareto --- Règle du 80-20}
    \begin{itemize}
        \item Principe simple:
        \begin{itemize}
            \item 80\% de tes résultats viennent de 20\% des efforts
            \item Pour obtenir le dernier 20\% des résultats, il faut mettre 80\% des efforts
        \end{itemize}
    \end{itemize}
    \vfill
    \only<1> {
        \makefigure[1][0.5][Source: \cite{pareto}]{pareto}
    }

    \only<2-> {
    \begin{twocolumns}[0.6]
        \leftcol
        \begin{itemize}
            \item 80\% des coûts vient de 20\% des pièces
            \item 80\% de la complexité vient de 20\% du design
            \item 80\% du power consommé par 20\% des pièces
            \item 80\% du temps de debug sur 20\% des problèmes
        \end{itemize}
        \rightcol
        \makefigure[1][0.5][Source: \cite{pareto}]{pareto}
    \end{twocolumns}
    }
\end{frame}


\subsection{Debugging}

\begin{frame}[t]{Outils de debugging}
    \vspace{-8pt}
    \begin{columns}[T]
        \begin{column}{0.25\textwidth}
            \vspace{-12pt}
            \begin{center}
                \textbf{Multimètre}
            \end{center}
            \begin{itemize}
                \item Mesures DC
                \bigskip
                \item Mesures de l'alimentation
                \item Vérifier des shorts
            \end{itemize}
        \end{column}
        \begin{column}{0.001\textwidth}
            \rule{0.1mm}{0.85\textheight}
        \end{column}

        \begin{column}{0.25\textwidth}
            \vspace{-12pt}
            \begin{center}
                \textbf{Oscilloscope}
            \end{center}
            \begin{itemize}
                \item Temporel
                \item Meilleur outil
                \bigskip
                \item Bruit
                \item Communication
            \end{itemize}
        \end{column}
        \begin{column}{0.001\textwidth}
            \rule{0.1mm}{0.85\textheight}
        \end{column}

        \begin{column}{0.25\textwidth}
            \vspace{-12pt}
            \begin{center}
                \textbf{Analyseur Logique}
            \end{center}
            \begin{itemize}
                \item Protocole
                \bigskip
                \item Décodage protocole
                \item Validation communication
            \end{itemize}
        \end{column}
        \begin{column}{0.001\textwidth}
            \rule{0.1mm}{0.85\textheight}
        \end{column}

        \begin{column}{0.25\textwidth}
            \vspace{-12pt}
            \begin{center}
                \textbf{Caméra Thermique}
            \end{center}
            \begin{itemize}
                \item Température
                \bigskip
                \item Trouver pièce brisée
                \item Valider requis thermiques
            \end{itemize}
        \end{column}
    \end{columns}
    \vspace{-0.25\textwidth}
    \begin{columns}
        \begin{column}{0.25\textwidth}
            \makefigure[1][0.4]{multimeter}
        \end{column}
        \begin{column}{0.25\textwidth}
            \makefigure[1][0.4]{oscilloscope}
        \end{column}
        \begin{column}{0.25\textwidth}
            \makefigure[1][0.4]{logic-analyzer}
        \end{column}
        \begin{column}{0.25\textwidth}
            \makefigure[1][0.4]{thermal-camera}
        \end{column}
    \end{columns}
\end{frame}

\begin{frame}[t]{Outils de debugging}
    \vspace{-8pt}
    \begin{columns}[T]
        \begin{column}{0.33\textwidth}
            \vspace{-12pt}
            \begin{center}
                \textbf{Current Clamp}
            \end{center}
            \begin{itemize}
                \item Courant
                \bigskip
                \item Mesures de l'alimentation
                \item Non-intrusif
            \end{itemize}
        \end{column}
        \begin{column}{0.001\textwidth}
            \rule{0.1mm}{0.85\textheight}
        \end{column}

        \begin{column}{0.33\textwidth}
            \vspace{-12pt}
            \begin{center}
                \textbf{Power Analyzer --- SMU}
            \end{center}
            \begin{itemize}
                \item Mesure power DC
                \bigskip
                \item Précision
                \item Logging
                \item Source
            \end{itemize}
        \end{column}
        \begin{column}{0.001\textwidth}
            \rule{0.1mm}{0.85\textheight}
        \end{column}

        \begin{column}{0.33\textwidth}
            \vspace{-12pt}
            \begin{center}
                \textbf{LCR Meter}
            \end{center}
            \begin{itemize}
                \item Réactance
                \bigskip
                \item Mesure de composants passives
                \item Impédance
                \item Quality Factor
            \end{itemize}
        \end{column}
    \end{columns}
    \vspace{-0.25\textwidth}
    \begin{columns}
        \begin{column}{0.33\textwidth}
            \makefigure[1][0.4]{current-probe}
        \end{column}
        \begin{column}{0.33\textwidth}
            \makefigure[1][0.4]{power-analyzer}
        \end{column}
        \begin{column}{0.33\textwidth}
            \makefigure[1][0.4]{lcr-meter}
        \end{column}
    \end{columns}
\end{frame}

\begin{frame}[t]{Outils de debugging}
    \vspace{-8pt}
    \begin{columns}[T]
        \begin{column}{0.33\textwidth}
            \vspace{-12pt}
            \begin{center}
                \textbf{Vector Network Analyzer}
            \end{center}
            \begin{itemize}
                \item Caractéristiques électriques
                \item Mesure signal et retour
                \bigskip
                \item Mesure Impédance
                \item S-Parameter
            \end{itemize}
        \end{column}
        \begin{column}{0.001\textwidth}
            \rule{0.1mm}{0.85\textheight}
        \end{column}

        \begin{column}{0.33\textwidth}
            \vspace{-12pt}
            \begin{center}
                \textbf{Spectrum Analyzer}
            \end{center}
            \begin{itemize}
                \item Oscilloscope sur stéroïdes
                \item Fourier
                \bigskip
                \item Mesure signal
                \item Mesure du bruit
            \end{itemize}
        \end{column}
        \begin{column}{0.001\textwidth}
            \rule{0.1mm}{0.85\textheight}
        \end{column}

        \begin{column}{0.33\textwidth}
            \vspace{-12pt}
            \begin{center}
                \textbf{Near-Field Probe}
            \end{center}
            \begin{itemize}
                \item EMI
                \bigskip
                \item Mesure bruit électromagnétique
                \item Fréquence précise
                \item EMC
            \end{itemize}
        \end{column}
    \end{columns}
    \vspace{-0.25\textwidth}
    \begin{columns}
        \begin{column}{0.33\textwidth}
            \makefigure[1][0.4]{vector-network-analyzer}
        \end{column}
        \begin{column}{0.33\textwidth}
            \makefigure[1][0.4]{spectrum-analyzer}
        \end{column}
        \begin{column}{0.33\textwidth}
            \makefigure[1][0.4]{near-field-probe}
        \end{column}
    \end{columns}
\end{frame}

\begin{frame}{Debugging \& Testing}
    \begin{itemize}
        \item Avoir plusieurs méthodes de debug
        \item Design pour pouvoir être debug
        \item Être conscient des outils de debugging à ta disposition
        \item Prévoir comment débugger et tester toutes les fonctionalités
        \bigskip
        \item Rajouter plus de testpoint que nécessaire
    \end{itemize}
\end{frame}

\begin{frame}{Où vont les testpoints?}
    \begin{itemize}
        \item GND GND GND
        \item Power
        \item Lignes de communication
        \item Toute la chaîne analogique
        \item Clocks et signaux de contrôle
        \item Et plus!
    \end{itemize}
\end{frame}

\subsection{Simulation}