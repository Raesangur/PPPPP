%!TEX root = ../main.tex 

\section{Bonnes pratiques de schéma}

\subsection{Clareté}
% Éviter tous les croisements, pas de GND dans les airs, aligner
% Utiliser la grille
% Utiliser des net names
% Quand utiliser des net locaux, globaux, power
% Inputs à gauche, output à droite
% Garder les sections modulaires, bien indiquer les sections, prend de l'espace
% Tout devrait être dans le schematics
%  - Mettre plus que pas assez
%  - Notes pour le programmeur, pour le layouteur, pour l'assembleur
%  - Tu ne devrais pas avoir à réouvrir une datasheet
% Mettre des couleurs sur les nets
% Utiliser des net classes
% Éviter les longues traces qui passent au travers du schéma
% Regrouper des nets en bus

\subsection{Notes}
% Mettre un schéma-bloc et un arbre d'alimentation
%     Draw.io
%     Travailler en hiérarchique
% Calculer le power pour chaque bloc et page
% Mettre des notes avec tous les calculs
% Mettre des notes avec pages de datasheets
% Mettre les courbes d'efficacité etc.
% Cartouche
% Indiquer les tailles et tolérances de composantes sur le schematic
% Notes pour les pins
% - Qu'est-ce qu'elles font, comment configurer
% Essayer le plus possible de garder tout le texte horizontal

\subsection{Testpoints et Debugging}
% Valider les boot sequences & power-sequence
% Se servir des pins flottantes
%  - Ajouter IO, LED, Testpoint, UART
% Mettre des systèmes de mesure de courant
% Met toujours plus de testpoints qu'on pense
%  - Différents types de testpoints

\subsection{Outils}
% Les 0R sont tes amis
%  - Peut remplacer une ferrite ou une shunt
%  - Pins EN et CFG
%  - Solder Bridge Pads
% Pull-ups / Pull-downs pour des pins de configuration
% Utiliser plusieurs pages / Faire un schéma hiérarchique
% Utiliser les Net Classes

\subsection{Autre}
% Toujours mettre des protections
%  - ESD
%  - Power
% Une connection I2C inter-board; mettre des pull-ups sur les deux boards
% Export ton schematic en PDF pour l'ouvrir sans logiciel
% 0.1µF

\subsection{Design Review}
% Faire réviser par quelqu'un d'autre après
% Passer le DRC / ERC
% Pour chaque composante:
%  - Valider le footprint
%  - Valider le symbole
%    - Numéros de pins selon la pièce choisie dans le BOM
%  - Valider chacune des connections en retournant vérifier dans la datasheet
%    - Est-ce que je suis au bon niveau de tension?
%    - Est-ce que j'ai besoin de pull-ups / pull-downs / autres passives
%  - Valider le découplage selon les fréquences
%  - Est-ce que le schéma est facilement lisible?
% Faire une review du BOM
%  - Est-ce que j'ai des erreurs de copier-coller?
%  - Est-ce que j'ai le bon modèle de pièce dans le BOM?
%    - Version SOIC vs version QFN
%    - Version pour un autre niveau de tension
%    - Version qui vient dans un paquet de 1000 au minimum
%  - Est-ce que j'ai toutes mes pièces mécaniques dans le BOM (standoffs, vis, washer, nut, heatsinks)
% Valider le debugging
%  - Est-ce que j'ai assez de testpoints (+ de GND!)
%  - Est-ce que j'ai des ports de debug sur mon MCU/FPGA
%  - Est-ce que j'ai des 0R pour ma configuration?
% Revalider tous les calculs
%  - Trucs changent pendant le design et ne sont pas toujours mis à jour
%  - Diviseurs de tension
%  - Dimentionnement des pièces
