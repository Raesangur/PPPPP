%!TEX root = ../main.tex 

\section{Bonnes pratiques de Layout}

\subsection{Routing}
% Toujours des plans continus sous les traces
%  - Ne jamais séparer les ground planes
% Un plan de GND par plan de signal
% - Alterner tes GND avec tes couches de signaux
% Faire des polygones pour le power
% Faire le stackup en fonction du fanout
% Teardrops
% Ordre de routing
% Faire attention au couplage
% Angle (30° acid trap)
% Règle du 3H
% Ne pas connecter une trace plus large que le pad auquel il se connecte
% Garder les traces le plus court possible, éviter des vias lorsque possible de router autrement
% Éviter de router sous des composantes

\subsection{Placement}
% Met toi plus de place que tu penses
% Placer les passives de l'autre bord
% Découplage
% Garder de l'espace sur les côtés du board
% Garder de l'espace autour des trous de vis
% Garder de l'espace avec les grosses composantes
% ESD entre l'entrée de la pin et la chip
% Orientation des chips
% Vue 3D pour le placement
% Fiducials

\subsection{Silkscreen}
% Met du texte partout
%  - Tous les testpoints, toutes les LEDs, tous les connecteurs
%  - Mettre le silkscreen plus loin
%  - Toujours avoir le silkscreen dans les mêmes orientations
% Orientation du texte

\subsection{Outils}
% Faire toutes les design rules en premier
%  - Net classes & impédance
% Grille
% Utiliser les fonctions de locking et de grouping
% Utilise les fonctions de keepout
%  - Annectode Carte SD
% Passer le DRC
% Nommer les nets critiques

\subsection{Communication avec le manufacturier}
% Fabrication Notes
% Fiducials
% Coupon de layers
% Coupon
%  - Tous les requirements les plus difficiles à faire
%  - Si ça passe pas sur le coupon, ça passe pas sur le panel
%  - Aller voir ce que le monde mettent sur le coupon
% Draftsman
% Assembly Drawing

\subsection{Autre}
% Thermal Relief
% Comment router un USB-C 2.0
% Perfectionisme (savoir quand arrêter)

\subsection{Design Review}
% Review les dimensions
%  - Taille du panneau
%  - Position des trous de montage
%    - Est-ce qu'ils doivent êtres connectés au GND?
%  - Clearance verticale (plus hautes composantes)
%  - Est-ce que c'est possible d'insérer un connecteur / carte SD?
% Review toutes les design rules
%  - Passer le DRC
%  - Valider avec le manufacturier (back and forth)
% Placement des pièces
%  - Est-ce que toutes les pièces sont placées
%  - Est-ce que les pièces sont accessibles pour rework
%  - Bon côté du board (connecteurs, LEDs, EMI...)
% Tout le texte est lisible
%  - Pas d'overlap avec des composantes
%  - Pas d'overlap avec des vias non-tented
%  - Deux orientations maximales
%  - Texte pour les testpoints, connecteurs, LEDs et autre
%  - Date, auteur, projet, logo, nom du board etc. visible
% Toutes les pins #1 sont visibles!
% Panelisation et fiducials
% Valider les impédances avec un autre outil (SaturnPCB, calculatrice en ligne)
% Vérifier toutes les largeurs de traces (courant, impédance, largeur par rapport au pad)
% Valider toutes les couches une par une (gerber viewer efficace pour ça)
%  - Est-ce que j'ai des coupures dans mes plans de PWR/GND
%  - Est-ce que je peux savoir quelle layer est laquelle
%  - Est-ce que mon stencil couvre tout, ou est-ce qu'il est ouvert là où il ne devrait pas l'être
%  - Est-ce que mon stencil est segmenté sous mes gros pads?
%  - Est-ce que j'ai une couverture de soldermask entre tous mes pads?
%  - Est-ce que toutes les informations sont là sur mes couches d'assemblage et de fabrication
%  - Est-ce que j'ai des couches de dimensions?
% Valider le routing
%  - Est-ce que toutes mes traces sont connectées?
%  - Est-ce que mes traces de PWR/GND sont très courtes et larges vers un/des vias
%  - Est-ce que je peux éloigner certaines traces l'unes de l'autre
%  - Est-ce que j'ai des ilôts non-connectés de cuivre
