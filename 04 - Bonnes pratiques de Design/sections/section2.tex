%!TEX root = ../presentation.tex 


\section{Bonnes pratiques des composantes \& BOM}

\subsection{Footprints}
% Toujours valider les footprints
% Faire tes footprints toi-même
% Faire très attention aux transistors!
% Marqueurs de pin 1
% Modèle 3D pour toutes les pièces
% S'assurer de respecter les couches mécaniques

\subsection{Symboles}
% Faire les symboles toi-même
% Utiliser les informations de pin (Input / Output / Passive / Power)
% Input à gauche, output à droite

\subsection{Datasheets}
% Lis la datasheet au complet!
% Lire les schematics des evaluation boards
% Lire les application notes faites par les fabricants de pièces
%  - Application notes sur le board design
% Toujours valider les courbes de power
%  - Combien de courant est-ce que la chip va prendre?
% Manufacturier donne parfois des configurateurs
% Valider les plages de température, de tension
%  - Comment le chip va chauffer dans ton utilisation?

\subsection{Recherche de pièces}
% Comment browser Digikey, Mouser, LCSC
%  - Marketplace components sur Digikey
% Not recommended for new designs
%  - Aller sur le site du manufacturier
% Liste de fournisseurs
%  - Carte du monde
%  - Attention aux tarrifs
%  - Attention aux fausses pièces

\subsection{BOM}
% Faire valider le BOM par quelqu'un d'autre
% Consolidation du BOM
% Un assemblage va coûter autant que les pièces
% Prix de vente = Fab price * 2.5 (eevblog)
% Limiter la taille du BOM (assemblage)
% Toujours tout mettre dans le BOM
%  - Fusibles
%  - Vis & Mechanical
%  - Alimentations