%!TEX root = ../main.tex 

\section{Bonnes pratiques de schéma}

\subsection{Clareté}
% Éviter tous les croisements, pas de GND dans les airs, aligner
% Utiliser la grille
% Utiliser des net names
% Quand utiliser des net locaux, globaux, power
% Inputs à gauche, output à droite
% Garder les sections modulaires, bien indiquer les sections
% Tout devrait être dans le schematics
%  - Mettre plus que pas assez
%  - Notes pour le programmeur, pour le layouteur, pour l'assembleur
%  - Tu ne devrais pas avoir à réouvrir une datasheet
% Mettre des couleurs sur les nets
% Utiliser des net classes

\subsection{Notes}
% Mettre un schéma-bloc et un arbre d'alimentation
%     Draw.io
%     Travailler en hiérarchique
% Calculer le power pour chaque bloc et page
% Mettre des notes avec tous les calculs
% Mettre des notes avec pages de datasheets
% Mettre les courbes d'efficacité etc.
% Cartouche
% Indiquer les tailles et tolérances de composantes sur le schematic
% Notes pour les pins
% - Qu'est-ce qu'elles font, comment configurer

\subsection{Testpoints et Debugging}
% Valider les boot sequences & power-sequence
% Se servir des pins flottantes
%  - Ajouter IO, LED, Testpoint, UART
% Mettre des systèmes de mesure de courant
% Met toujours plus de testpoints qu'on pense
%  - Différents types de testpoints

\subsection{Outils}
% Passer le DRC
% Les 0R sont tes amis
%  - Peut remplacer une ferrite ou une shunt
%  - Pins EN et CFG
%  - Solder Bridge Pads

\subsection{Autre}
% Bien faire son découplage
% Toujours mettre des protections
%  - ESD
%  - Power
% Export ton schematic en PDF pour l'ouvrir sans logiciel
