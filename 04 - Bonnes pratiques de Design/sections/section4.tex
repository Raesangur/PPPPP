%!TEX root = ../main.tex 

\section{Comment conçevoir un arbre d'alimentation?}

\subsection{Déterminer les besoins}

\begin{frame}{Tensions d'opération}
    \begin{itemize}
        \item[] \textcolor{UDSgreenFierte}{\faToggleOn} ~Chaque puce a une tension d'opération
        \item[] \textcolor{UDSgreenFierte}{\faList} ~Parfois plusieurs tensions possible ($\SI{3.3}{\volt}$ jusqu'à $\SI{5}{\volt}$)
        \item[] \textcolor{UDSgreenFierte}{\faSignal} ~Parfois plusieurs tensions nécessaires!
        \bigskip
        \item[] \textcolor{UDSgreenFierte}{\faWaveSquare} ~Parfois les puces peuvent avoir des IO à des tensions différentes
    \end{itemize}
\end{frame}

\begin{frame}{Courant d'opération}
    \begin{columns}
        \only<2> {
            \begin{column}{0.55\textwidth}
        }
        \only<1, 3> {
            \begin{column}{0.66\textwidth}
        }
            \begin{itemize}
                \item[] \icon{\faListOl} ~Pour chaque puce, à ses tension d'opération, récupérer:
                \begin{itemize}
                    \item[] \itemicon{\faBed}
                        ~Le courant minimal (sleep)
                    \item[] \itemicon{\faWalking}
                        ~Le courant normal d'opération
                    \item[] \itemicon{\faRunning}
                        ~Le courant maximal d'opération
                    \item[] \itemicon{\faEquals}
                        ~Parfois tout le même
                \end{itemize}
                \only<2-> {
                    \item[] \itemicon[accent][-6pt]{\faArrowRight}
                        ~Conçevoir circuit pour pouvoir passer le courant maximal
                    \item[] \itemicon[accent][-6pt]{\faChartLine}
                        ~Choisir régulateurs pour avoir efficacité maximale au courant nominal
                }
                \only<3> {
                    \item[] \itemicon[accent][-6pt]{\faList} ~Aussi rassembler tous les courants autres
                    \begin{itemize}
                        \item[] \itemicon{\faSync}
                            ~Moteurs
                        \item[] \itemicon{\faLightbulb}
                            ~LEDs et affichage
                        \item[] \itemicon{\faEthernet}
                            ~Connecteurs et sous-cartes
                    \end{itemize}
                }
            \end{itemize}
        \end{column}
        
        \only<2> {
        \begin{column}{0.45\textwidth}
            \begin{figure}
                \includegraphics<2>[width=\textwidth, height=0.75\textheight, keepaspectratio]{pictures/switching-efficiency-curve.png}
            \end{figure}
        \end{column}
        }
        \only<1, 3> {
        \begin{column}{0.33\textwidth}
            \begin{figure}
                \includegraphics<3>[width=\textwidth, height=0.75\textheight, keepaspectratio]{pictures/switching-efficiency-curve.png}
            \end{figure}
        \end{column}
        }
    \end{columns}
\end{frame}

\begin{frame}{Trouver le courant d'opération}
    \begin{figure}
        \includegraphics[width=\textwidth, height=0.75\textheight, keepaspectratio]{pictures/ad4050-power.png}
    \end{figure}
    \href{https://www.analog.com/media/en/technical-documentation/data-sheets/ad4050-ad4056.pdf}{Analog Devices - AD4050 - 12-Bit 2MSPS SAR ADC}
\end{frame}

\begin{frame}{Besoins en stabilité}
    \begin{itemize}
        \item[] \textcolor{UDSgreenFierte}{\faWater} ~Déterminer les besoins de chaque puce en stabilité
        \item[] \textcolor{UDSgreenFierte}{\faBatteryQuarter} ~Quel est le $\Delta V$ maximum tolérable pour l'opération de la puce
        \item[] \textcolor{UDSgreenFierte}{\faTachometer*} ~Quelle est la précision nécessaire pour une puce analogique
        \item[] \textcolor{UDSgreenFierte}{\faClock} ~À quel point un $\Delta V$ peut introduire un $\Delta f$ pour la fréquence?
    \end{itemize}

    \begin{columns}
        \begin{column}{0.5\textwidth}
            \only<2-> {
            \begin{itemize}
                \item[] \textcolor{UDSgreenFierte}{\faChartBar} ~Toujours mieux de quantifier
                \item[] \textcolor{UDSgreenFierte}{\faList} ~\textbf{Permet de déterminer type de régulateur}
            \end{itemize}
            }
        \end{column}
        \begin{column}{0.5\textwidth}
            \begin{figure}
                \includegraphics[width=\textwidth, height=0.45\textheight, keepaspectratio]{pictures/eye-diagram.png}
            \end{figure}
        \end{column}
    \end{columns}
\end{frame}


\subsection{Bilan d'alimentation}

\begin{frame}{Bilan d'alimentation - Example - Alimentation 12V}
    \centering
    \Large
    \renewcommand{\arraystretch}{1.2}
    \vspace{-6pt}

    \begin{tabular}{>{\color{UDSgreenSolidarite}}c l | l | l | l | l | l}
    \rowcolor{UDSgreenSolidarite}
    & \textcolor{white}{\textbf{IC}}
    & \textcolor{white}{\textbf{Tension}}
    & \textcolor{white}{\textbf{I min}}
    & \textcolor{white}{\textbf{I nom}}
    & \textcolor{white}{\textbf{I max}} 
    & \textcolor{white}{\textbf{Type}}\\
    \hline
    \textcolor{UDSgreenFierte}{\faMicrochip} &
        \textbf{IC 1} &
        $\SI{3.3}{\volt}$ &
        $\SI{10}{\micro\ampere}$ &
        $\SI{17.5}{\milli\ampere}$ &
        $\SI{17.5}{\milli\ampere}$ &
        Digital\\
    \textcolor{UDSgreenFierte}{\faMicrochip} &
        \textbf{IC 2} &
        $\SI{3.3}{\volt}$ &
        $\SI{10}{\milli\ampere}$ &
        $\SI{10}{\milli\ampere}$ &
        $\SI{10}{\milli\ampere}$ &
        Digital\\
    & \hspace{16pt}\raisebox{10pt}{\rotatebox{-90}{\faLevelUp*}} &
        $\SI{3.3}{\volt}$ &
        $\SI{0}{\micro\ampere}$ &
        $\SI{10}{\milli\ampere}$ &
        $\SI{50}{\milli\ampere}$ &
        Analog\\
    \textcolor{UDSgreenFierte}{\faMicrochip} &
        \textbf{IC 3} &
        $\SI{2.5}{\volt}$ &
        $\SI{20}{\milli\ampere}$ &
        $\SI{20}{\milli\ampere}$ &
        $\SI{50}{\milli\ampere}$ &
        Analog\\
    \textcolor{UDSgreenFierte}{\faLightbulb} &
        \textbf{LEDs} &
        $\SI{5}{\volt}$ &
        $\SI{0}{\ampere}$ &
        $\SI{80}{\milli\ampere}$ &
        $\SI{80}{\milli\ampere}$ &
        Digital\\
    \textcolor{UDSgreenFierte}{\faUsb} &
        \textbf{USB} &
        $\SI{5}{\volt}$ &
        $\SI{0}{\ampere}$ &
        $\SI{100}{\milli\ampere}$ &
        $\SI{500}{\milli\ampere}$ &
        Digital\\
    \textcolor{UDSgreenFierte}{\faFan} &
        \textbf{Stepper} &
        $\SI{12}{\volt}$ &
        $\SI{0}{\ampere}$ &
        $\SI{300}{\milli\ampere}$ &
        $\SI{750}{\milli\ampere}$ &
        Digital\\
    \hline
    \textcolor{UDSgreenFierte}{\boldmath$\Sigma$} & \textbf{Total} & &
        \textbf{\boldmath$\SI{30.01}{\milli\ampere}$} &
        \textbf{\boldmath$\SI{537.5}{\milli\ampere}$} &
        \textbf{\boldmath$\SI{1457.5}{\milli\ampere}$} & \\
    \end{tabular}
\end{frame}

\begin{frame}{Bilan d'alimentation - Example - Alimentation 12V}
    \centering
    \Large
    \renewcommand{\arraystretch}{1.2}
    \vspace{-6pt}

    \begin{tabular}{>{\color{UDSgreenSolidarite}}c l | l | l | l}
    \rowcolor{UDSgreenSolidarite}
    & \textcolor{white}{\textbf{IC}}
    & \textcolor{white}{\textbf{Tension}}
    & \textcolor{white}{\textbf{I max}} 
    & \textcolor{white}{\textbf{Type}}\\
    \hline
    \textcolor{UDSgreenFierte}{\faMicrochip} &
        \textbf{IC 1} &
        $\SI{3.3}{\volt}$ &
        $\SI{17.5}{\milli\ampere}$ &
        Digital\\
    \textcolor{UDSgreenFierte}{\faMicrochip} &
        \textbf{IC 2} &
        $\SI{3.3}{\volt}$ &
        $\SI{10}{\milli\ampere}$ &
        Digital\\
    & \hspace{16pt}\raisebox{10pt}{\rotatebox{-90}{\faLevelUp*}} &
        $\SI{3.3}{\volt}$ &
        $\SI{50}{\milli\ampere}$ &
        Analog\\
    \textcolor{UDSgreenFierte}{\faMicrochip} &
        \textbf{IC 3} &
        $\SI{2.5}{\volt}$ &
        $\SI{50}{\milli\ampere}$ &
        Analog\\
    \textcolor{UDSgreenFierte}{\faLightbulb} &
        \textbf{LEDs} &
        $\SI{5}{\volt}$ &
        $\SI{80}{\milli\ampere}$ &
        Digital\\
    \textcolor{UDSgreenFierte}{\faUsb} &
        \textbf{USB} &
        $\SI{5}{\volt}$ &
        $\SI{500}{\milli\ampere}$ &
        Digital\\
    \textcolor{UDSgreenFierte}{\faFan} &
        \textbf{Stepper} &
        $\SI{12}{\volt}$ &
        $\SI{750}{\milli\ampere}$ &
        Digital\\
    \hline
    \textcolor{UDSgreenFierte}{\boldmath$\Sigma$} & \textbf{Total} & &
        \textbf{\boldmath$\SI{1457.5}{\milli\ampere}$} & \\
    \end{tabular}
\end{frame}

\begin{frame}{Diagramme d'alimentation}
    \centering
    \resizebox{\textwidth}{!}{
    \begin{tikzpicture}[
        block/.style = {rectangle, draw,
        minimum height=0.75cm,
        minimum width=2cm,
        align=center,
        very thick,
        rounded corners=0.1cm},
        node distance=0.8cm and 0.6cm,
        >={Stealth[round]},
        x = 3cm,
        y = 1cm
    ]

    \node(inlabel) at (0, 0) {};

    \node[block] (prot) at (0.5, 0)
        {\textcolor{UDSgreenFierte}{\faShield*} ~\textcolor{black}{Protections}};
    \node[block] (step) at (5,   0)
        {\textcolor{UDSgreenFierte}{\faFan} ~\textcolor{black}{Stepper}};
    \node[block] (5v)   at (2,  -1)
        {\textcolor{UDSgreenFierte}{\faWaveSquare} ~\textcolor{black}{\SI{5}{\volt}}};
    \node[block] (led)  at (5,  -1)
        {\textcolor{UDSgreenFierte}{\faLightbulb} ~\textcolor{black}{LED}};
    \node[block] (usb)  at (5,  -2)
        {\textcolor{UDSgreenFierte}{\faUsb} ~\textcolor{black}{USB}};
    \node[block] (3v3)  at (2,  -3)
        {\textcolor{UDSgreenFierte}{\faWaveSquare} ~\textcolor{black}{\SI{3.3}{\volt}}};

    \node[block] (3v3a) at (3.33,  -5)
        {\textcolor{UDSgreenFierte}{\faSync} ~\textcolor{black}{Ferrite}};
    \node[block] (3v3ic1) at (3.33,  -3)
        {\textcolor{UDSgreenFierte}{\faSync} ~\textcolor{black}{Ferrite}};
    \node[block] (3v3ic2) at (3.33,  -4)
        {\textcolor{UDSgreenFierte}{\faSync} ~\textcolor{black}{Ferrite}};

    \node[block] (ic1)  at (5,  -3)
        {\textcolor{UDSgreenFierte}{\faMicrochip} ~\textcolor{black}{IC 1}};
    \node[block] (ic2)  at (5,  -4)
        {\textcolor{UDSgreenFierte}{\faMicrochip} ~\textcolor{black}{IC 2}};
    \node[block] (ic2a) at (5,  -5)
        {\textcolor{UDSgreenFierte}{\faMicrochip} ~\textcolor{black}{IC 2A}};
    \node[block] (2v5)  at (3.33,  -6)
        {\textcolor{UDSgreenFierte}{\faLongArrowAltRight} ~\textcolor{black}{\SI{2.5}{\volt}}};
    \node[block] (ic3)  at (5,  -6)
        {\textcolor{UDSgreenFierte}{\faMicrochip} ~\textcolor{black}{IC 3}};

    \draw[->, ultra thick, color=red]
         ($(inlabel)+(-1.25, 0)$) -- (prot.west)
         node[midway, above]
         {\textcolor{black}{Input $\SI{12}{\volt}$}};

    \draw[->, ultra thick, color=red]
         (prot) -- (step)
         node[midway, above]
         {\textcolor{black}{$\SI{750}{\milli\ampere}$}};
    \draw[->, ultra thick, color=red]
         (prot) |- (5v)
         node[pos=0.75, above]
         {\textcolor{black}{$?\SI{}{\milli\ampere}$}};
    \draw[->, ultra thick, color=red]
         (prot) |- (3v3)
         node[pos=0.75, above]
         {\textcolor{black}{$?\SI{}{\milli\ampere}$}};

    \draw[->, ultra thick, color=magenta]
         (5v) -- (led)
         node[midway, above]
         {\textcolor{black}{$\SI{80}{\milli\ampere}$}};
    \draw[->, ultra thick, color=magenta]
         (5v) |- (usb)
         node[pos=0.79, above]
         {\textcolor{black}{$\SI{500}{\milli\ampere}$}};

    \draw[->, ultra thick, color=blue]
         (3v3) -- (3v3ic1)
         node[midway, above]
         {\textcolor{black}{$\SI{17.5}{\milli\ampere}$}};
    \draw[->, ultra thick, color=blue]
         (3v3ic1) -- (ic1)
         node[midway, above]
         {\textcolor{black}{$\SI{17.5}{\milli\ampere}$}};
    \draw[->, ultra thick, color=blue]
         (3v3) |- (3v3ic2)
         node[pos=0.825, above]
         {\textcolor{black}{$\SI{10}{\milli\ampere}$}};
    \draw[->, ultra thick, color=blue]
         (3v3ic2) -- (ic2)
         node[midway, above]
         {\textcolor{black}{$\SI{10}{\milli\ampere}$}};
    \draw[->, ultra thick, color=blue]
         (3v3) |- (3v3a)
         node[pos=0.825, above]
         {\textcolor{black}{$\SI{50}{\milli\ampere}$}};
    \draw[->, ultra thick, color=blue]
         (3v3) |- (2v5)
         node[pos=0.825, above]
         {\textcolor{black}{?$\SI{}{\milli\ampere}$}};

    \draw[->, ultra thick, color=green]
         (3v3a) -- (ic2a)
         node[midway, above]
         {\textcolor{black}{$\SI{50}{\milli\ampere}$}};
    \draw[->, ultra thick, color=orange]
         (2v5)  -- (ic3)
         node[midway, above]
         {\textcolor{black}{$\SI{50}{\milli\ampere}$}};
    \end{tikzpicture}
    }
\end{frame}


\begin{frame}{Résolution des régulateurs - 5V}
    \begin{columns}
        \begin{column}{0.4\textwidth}
            \only<1> {
                \begin{itemize}
                    \item Remplir les informations de base
                    \begin{itemize}
                        \item[] \textcolor{UDSgreenFierte}{\faFileSignature}
                            ~Nom du régulateur
                        \item[] \textcolor{UDSgreenFierte}{\faTasks}
                            ~Séquence
                        \item[] \textcolor{UDSgreenFierte}{\faWaveSquare}
                            ~Type de régulateur
                        \item[] \textcolor{UDSgreenFierte}{\faArrowRight}
                            ~Courant maximum
                        \item[] \textcolor{UDSgreenFierte}{\faTemperatureHigh}
                            ~$R_{\theta JA}$
                    \end{itemize}
                    \item Remplir les informations de sortie
                    \begin{itemize}
                        \item[] \textcolor{UDSgreenFierte}{\faBatteryThreeQuarters}
                            ~Tension
                        \item[] \textcolor{UDSgreenFierte}{\faBed}
                            ~Courant min
                        \item[] \textcolor{UDSgreenFierte}{\faWalking}
                            ~Courant nominal
                        \item[] \textcolor{UDSgreenFierte}{\faRunning}
                            ~Courant max
                    \end{itemize}
                    \item \textit{Il manque le courant d'entrée}
                \end{itemize}
            }
            \only<2> {
                \begin{itemize}
                    \item Pour le $I_{in}$ il nous faut le $P_{in}$
                    \item Pour le $P_{in}$ il nous faut le $\eta$
                \end{itemize}
            }
            \only<3> {
                \begin{itemize}
                    \item Trouver le $\eta$ dans le graphique
                    \item Facilement $\pm 10\%$ entre $\eta_{nom}$ et $\eta_{max}$
                \end{itemize}
            }
            \only<4> {
                \begin{center}
                    $P_{in_{nom}} = \frac{P_{out_{nom}}}{\eta_{nom}}$\\
                    \vspace{6pt}
                    $P_{in_{max}} = \frac{P_{out_{max}}}{\eta_{max}}$
                \end{center}
            }
            \only<5> {
                \begin{center}
                    $P_{diss_{nom}} = P_{out_{nom}} - P_{in_{nom}}$\\
                    \vspace{6pt}
                    $P_{diss_{max}} = P_{out_{max}} - P_{in_{max}}$
                \end{center}
            }
            \only<6> {
                \begin{center}
                    $I_{in_{nom}} = \frac{P_{out_{nom}}}{V_{in}}$\\
                    \vspace{6pt}
                    $I_{in_{max}} = \frac{P_{out_{max}}}{V_{in}}$
                \end{center}
            }
            \only<7> {
                \begin{center}
                    $\Delta t_{_{nom}} = P_{diss_{nom}} \cdot R_{\theta JA}$\\
                    \vspace{6pt}
                    $\Delta t_{_{max}} = P_{diss_{max}} \cdot R_{\theta JA}$
                \end{center}
            }

            \begin{figure}
                \includegraphics<2> [width=\textwidth, height=0.75\textheight, keepaspectratio]{pictures/l6982-efficiency-curve-raw.png}
                \includegraphics<3->[width=\textwidth, height=0.75\textheight, keepaspectratio]{pictures/l6982-efficiency-curve-dual.png}
            \end{figure}
        \end{column}

        \begin{column}{0.6\textwidth}
            \centering
            \vspace{-6pt}

            \begin{tabular}{C{0.2\textwidth} | C{0.2\textwidth} | C{0.2\textwidth}}
                \rowcolor{UDSgreenSolidarite}
                \multicolumn{3}{c}{\textcolor{white}{\textbf{L6982}}}\\
                \hline

                Type         & $R_{\theta JA}$              & $I_{max}$\\
                \textbf{SWR} & $\SI{55}{\celsius\per\watt}$ & $\SI{2}{\ampere}$\\
                \hline

                $V_{in}$         & & $V_{out}$\\
                $\SI{12}{\volt}$ & & $\SI{5}{\volt}$\\
                \hline

                $I_{in_{nom}}$    & & $I_{out_{nom}}$\\
                \only<2-5> {
                    $?\SI{}{\ampere}$ & & $\SI{180}{\milli\ampere}$\\
                }
                \only<6-> {
                    $\SI[parse-numbers=false]{83.\overline{3}}{\milli\ampere}$ & & $\SI{180}{\milli\ampere}$\\
                    }
                $I_{in_{max}}$    & & $I_{out_{max}}$\\
                \only<2-5> {
                    $?\SI{}{\ampere}$ & & $\SI{580}{\milli\ampere}$
                }
                \only<6-> {
                    $\SI{260}{\milli\ampere}$ & & $\SI{580}{\milli\ampere}$
                }
                

                \only<2-> {
                    \\
                    \hline

                    \multicolumn{3}{c}{
                        \begin{tabular}{C{0.3\textwidth} | C{0.3\textwidth}}
                            \only<2> {
                                $\eta_{nom} =\ ?\%$ &
                                $\eta_{max} =\ ?\%$
                            }
                            \only<3-> {
                                $\eta_{nom} =\ 90\%$ &
                                $\eta_{max} =\ 93\%$
                            }
                        \end{tabular}
                    }
                }
                \only<7-> {
                    \\
                    \hline

                    \multicolumn{3}{c}{
                        \begin{tabular}{C{0.3\textwidth} | C{0.3\textwidth}}
                                $\Delta t_{_{nom}} =\ \SI{5.5}{\celsius}$ &
                                $\Delta t_{_{max}} =\ \SI{12.1}{\celsius}$
                        \end{tabular}
                    }
                }
                
                \only<4-> {
                \\
                    \hline

                    $P_{in_{nom}}$      & $P_{diss_{nom}}$        & $P_{out_{nom}}$\\
                    \only<4> {
                        $\SI{1}{\watt}$ & $?\SI{}{\watt}$         & $\SI{900}{\milli\watt}$\\
                    }
                    \only<5-> {
                        $\SI{1}{\watt}$ & $\SI{100}{\milli\watt}$ & $\SI{900}{\milli\watt}$\\
                    }

                    $P_{in_{max}}$         & $P_{diss_{max}}$        & $P_{out_{max}}$\\
                    \only<4> {
                        $\SI{3.12}{\watt}$ & $?\SI{}{\watt}$         & $\SI{2.9}{\watt}$
                    }
                    \only<5-> {
                        $\SI{3.12}{\watt}$ & $\SI{220}{\milli\watt}$ & $\SI{2.9}{\watt}$
                    }
                }
            \end{tabular}
        \end{column}
    \end{columns}
\end{frame}


\begin{frame}{Résolution des régulateurs}
    \small
    \begin{columns}
        \begin{column}{0.33\textwidth}
            \vspace{-6pt}
            \begin{tabular}{C{0.25\textwidth} | C{0.25\textwidth} | C{0.25\textwidth}}
                \rowcolor{UDSgreenSolidarite}
                \multicolumn{3}{c}{\textcolor{white}{\textbf{TPS79025}}}\\
                \hline

                Type         & $R_{\theta JA}$                & $I_{max}$\\
                \textbf{LDO} & $\SI{73.1}{\celsius\per\watt}$ & $\SI{2}{\ampere}$\\
                \hline

                $V_{in}$          & & $V_{out}$\\
                $\SI{3.3}{\volt}$ & & $\SI{2.5}{\volt}$\\
                \hline

                $I_{in_{nom}}$           & & $I_{out_{nom}}$\\
                $\SI{20}{\milli\ampere}$ & & $\SI{20}{\milli\ampere}$\\
                $I_{in_{max}}$           & & $I_{out_{max}}$\\
                $\SI{50}{\milli\ampere}$ & & $\SI{50}{\milli\ampere}$\\
                \hline

                \multicolumn{3}{c}{
                    $\eta =\ 75.\overline{75}\%$
                }\\
                \hline

                $P_{in_{nom}}$         & $P_{diss_{nom}}$       & $P_{out_{nom}}$\\
                $\SI{66}{\milli\watt}$ & $\SI{16}{\milli\watt}$ & $\SI{50}{\milli\watt}$\\

                $P_{in_{max}}$          & $P_{diss_{max}}$       & $P_{out_{max}}$\\
                $\SI{165}{\milli\watt}$ & $\SI{40}{\milli\watt}$ & $\SI{125}{\milli\watt}$
            \end{tabular}
        \end{column}
        \begin{column}{0.001\textwidth}
            \rule{0.1mm}{0.85\textheight}
        \end{column}
        \begin{column}{0.33\textwidth}
            \vspace{-6pt}
            \begin{tabular}{C{0.25\textwidth} | C{0.25\textwidth} | C{0.25\textwidth}}
                \rowcolor{UDSgreenSolidarite}
                \multicolumn{3}{c}{\textcolor{white}{\textbf{L6982}}}\\
                \hline

                Type         & $R_{\theta JA}$              & $I_{max}$\\
                \textbf{SWR} & $\SI{55}{\celsius\per\watt}$ & $\SI{2}{\ampere}$\\
                \hline

                $V_{in}$         & & $V_{out}$\\
                $\SI{12}{\volt}$ & & $\SI{3.3}{\volt}$\\
                \hline

                $I_{in_{nom}}$         & & $I_{out_{nom}}$\\
                $\SI{19}{\milli\ampere}$ & & $\SI{58}{\milli\ampere}$\\
                $I_{in_{max}}$         & & $I_{out_{max}}$\\
                $\SI{39}{\milli\ampere}$ & & $\SI{128}{\milli\ampere}$\\
                \hline

                \multicolumn{3}{c}{
                    \begin{tabular}{C{0.375\textwidth} | C{0.375\textwidth}}
                        $\eta_{nom} =\ 86\%$ &
                        $\eta_{max} =\ 90\%$
                    \end{tabular}
                }\\
                \hline

                $P_{in_{nom}}$  & $P_{diss_{nom}}$        & $P_{out_{nom}}$\\
                $\SI{221}{\milli\watt}$ & $\SI{31}{\milli\watt}$ & $\SI{190}{\milli\watt}$\\

                $P_{in_{max}}$     & $P_{diss_{max}}$        & $P_{out_{max}}$\\
                $\SI{468}{\milli\watt}$ & $\SI{47}{\milli\watt}$ & $\SI{420}{\milli\watt}$
            \end{tabular}
        \end{column}
        \begin{column}{0.001\textwidth}
            \rule{0.1mm}{0.85\textheight}
        \end{column}
        \begin{column}{0.33\textwidth}
            \vspace{-6pt}
            \begin{tabular}{C{0.25\textwidth} | C{0.25\textwidth} | C{0.25\textwidth}}
                \rowcolor{UDSgreenSolidarite}
                \multicolumn{3}{c}{\textcolor{white}{\textbf{L6982}}}\\
                \hline

                Type         & $R_{\theta JA}$              & $I_{max}$\\
                \textbf{SWR} & $\SI{55}{\celsius\per\watt}$ & $\SI{2}{\ampere}$\\
                \hline

                $V_{in}$         & & $V_{out}$\\
                $\SI{12}{\volt}$ & & $\SI{5}{\volt}$\\
                \hline

                $I_{in_{nom}}$            & & $I_{out_{nom}}$\\
                $\SI{83}{\milli\ampere}$  & & $\SI{180}{\milli\ampere}$\\
                $I_{in_{max}}$            & & $I_{out_{max}}$\\
                $\SI{260}{\milli\ampere}$ & & $\SI{580}{\milli\ampere}$\\
                \hline

                \multicolumn{3}{c}{
                    \begin{tabular}{C{0.375\textwidth} | C{0.375\textwidth}}
                        $\eta_{nom} =\ 90\%$ &
                        $\eta_{max} =\ 93\%$
                    \end{tabular}
                }\\
                \hline

                $P_{in_{nom}}$  & $P_{diss_{nom}}$        & $P_{out_{nom}}$\\
                $\SI{1}{\watt}$ & $\SI{100}{\milli\watt}$ & $\SI{900}{\milli\watt}$\\

                $P_{in_{max}}$    & $P_{diss_{max}}$        & $P_{out_{max}}$\\
                $\SI{3.1}{\watt}$ & $\SI{220}{\milli\watt}$ & $\SI{2.9}{\watt}$
            \end{tabular}
        \end{column}
    \end{columns}
\end{frame}

\begin{frame}{Diagramme d'alimentation}
    \centering
    \resizebox{\textwidth}{!}{
    \begin{tikzpicture}[
        block/.style = {rectangle, draw,
        minimum height=0.75cm,
        minimum width=2cm,
        align=center,
        very thick,
        rounded corners=0.1cm},
        node distance=0.8cm and 0.6cm,
        >={Stealth[round]},
        x = 3cm,
        y = 1cm
    ]

    \node(inlabel) at (0, 0) {};

    \node[block] (prot) at (0.5, 0)
        {\textcolor{UDSgreenFierte}{\faShield*} ~\textcolor{black}{Protections}};
    \node[block] (step) at (5,   0)
        {\textcolor{UDSgreenFierte}{\faFan} ~\textcolor{black}{Stepper}};
    \node[block] (5v)   at (2,  -1)
        {\textcolor{UDSgreenFierte}{\faWaveSquare} ~\textcolor{black}{\SI{5}{\volt}}};
    \node[block] (led)  at (5,  -1)
        {\textcolor{UDSgreenFierte}{\faLightbulb} ~\textcolor{black}{LED}};
    \node[block] (usb)  at (5,  -2)
        {\textcolor{UDSgreenFierte}{\faUsb} ~\textcolor{black}{USB}};
    \node[block] (3v3)  at (2,  -3)
        {\textcolor{UDSgreenFierte}{\faWaveSquare} ~\textcolor{black}{\SI{3.3}{\volt}}};

    \node[block] (3v3a) at (3.33,  -5)
        {\textcolor{UDSgreenFierte}{\faSync} ~\textcolor{black}{Ferrite}};
    \node[block] (3v3ic1) at (3.33,  -3)
        {\textcolor{UDSgreenFierte}{\faSync} ~\textcolor{black}{Ferrite}};
    \node[block] (3v3ic2) at (3.33,  -4)
        {\textcolor{UDSgreenFierte}{\faSync} ~\textcolor{black}{Ferrite}};

    \node[block] (ic1)  at (5,  -3)
        {\textcolor{UDSgreenFierte}{\faMicrochip} ~\textcolor{black}{IC 1}};
    \node[block] (ic2)  at (5,  -4)
        {\textcolor{UDSgreenFierte}{\faMicrochip} ~\textcolor{black}{IC 2}};
    \node[block] (ic2a) at (5,  -5)
        {\textcolor{UDSgreenFierte}{\faMicrochip} ~\textcolor{black}{IC 2A}};
    \node[block] (2v5)  at (3.33,  -6)
        {\textcolor{UDSgreenFierte}{\faLongArrowAltRight} ~\textcolor{black}{\SI{2.5}{\volt}}};
    \node[block] (ic3)  at (5,  -6)
        {\textcolor{UDSgreenFierte}{\faMicrochip} ~\textcolor{black}{IC 3}};

    \draw[->, ultra thick, color=red]
         ($(inlabel)+(-1.25, 0)$) -- (prot.west)
         node[midway, above]
         {\textcolor{black}{Input $\SI{12}{\volt}$}};

    \draw[->, ultra thick, color=red]
         (prot) -- (step)
         node[midway, above]
         {\textcolor{black}{$\SI{750}{\milli\ampere}$}};
    \draw[->, ultra thick, color=red]
         (prot) |- (5v)
         node[pos=0.75, above]
         {\textcolor{black}{$260\SI{}{\milli\ampere}$}};
    \draw[->, ultra thick, color=red]
         (prot) |- (3v3)
         node[pos=0.75, above]
         {\textcolor{black}{$40\SI{}{\milli\ampere}$}};

    \draw[->, ultra thick, color=magenta]
         (5v) -- (led)
         node[midway, above]
         {\textcolor{black}{$\SI{80}{\milli\ampere}$}};
    \draw[->, ultra thick, color=magenta]
         (5v) |- (usb)
         node[pos=0.79, above]
         {\textcolor{black}{$\SI{500}{\milli\ampere}$}};

    \draw[->, ultra thick, color=blue]
         (3v3) -- (3v3ic1)
         node[midway, above]
         {\textcolor{black}{$\SI{17.5}{\milli\ampere}$}};
    \draw[->, ultra thick, color=blue]
         (3v3ic1) -- (ic1)
         node[midway, above]
         {\textcolor{black}{$\SI{17.5}{\milli\ampere}$}};
    \draw[->, ultra thick, color=blue]
         (3v3) |- (3v3ic2)
         node[pos=0.825, above]
         {\textcolor{black}{$\SI{10}{\milli\ampere}$}};
    \draw[->, ultra thick, color=blue]
         (3v3ic2) -- (ic2)
         node[midway, above]
         {\textcolor{black}{$\SI{10}{\milli\ampere}$}};
    \draw[->, ultra thick, color=blue]
         (3v3) |- (3v3a)
         node[pos=0.825, above]
         {\textcolor{black}{$\SI{50}{\milli\ampere}$}};
    \draw[->, ultra thick, color=blue]
         (3v3) |- (2v5)
         node[pos=0.825, above]
         {\textcolor{black}{?$\SI{}{\milli\ampere}$}};

    \draw[->, ultra thick, color=green]
         (3v3a) -- (ic2a)
         node[midway, above]
         {\textcolor{black}{$\SI{50}{\milli\ampere}$}};
    \draw[->, ultra thick, color=orange]
         (2v5)  -- (ic3)
         node[midway, above]
         {\textcolor{black}{$\SI{50}{\milli\ampere}$}};

    \node[block] (power) at (0, -5.5) {
        $I_{in_{total}} = \SI{750}{\milli\ampere} + \SI{260}{\milli\ampere} + \SI{40}{\milli\ampere}$\\
        $I_{in_{total}} = \SI{1.05}{\ampere}$\\
        \vspace{12pt}
        $P_{in_{total}} = \SI{12.6}{\watt}$\\
        \vspace{12pt}
        $\eta_{total} = \dfrac{\SI{12.445}{\watt}}{\SI{12.6}{\watt}} = 97.7\%$
    };
    \end{tikzpicture}
    }
\end{frame}


\subsection{Séquençage}

\begin{frame}{Séquençage - Exemple}
    \begin{itemize}
        \item[] \textcolor{UDSgreenFierte}{\faList} ~Puces avec plusieurs alimentations ont parfois besoin de séquence
        \item[] \textcolor{UDSgreenFierte}{\faLevelDown*} ~Séquence de power-up, mais aussi de power-down
        \item[] \textcolor{red}{\faBurn} ~Mauvais comportement, dommages possibles
    \end{itemize}

    \vfill
    \begin{figure}
        \includegraphics[width=\textwidth, height=0.55\textheight, keepaspectratio]{pictures/ice40-power-supply-sequence.png}
    \end{figure}
    \href{https://www.latticesemi.com/view_document?document_id=49312}{Lattice - ICE40 - Ultra-low power non-volatile FPGA}
\end{frame}

\begin{frame}{Séquençage - Power Good}
    \begin{columns}
        \begin{column}{0.4\textwidth}
            \vspace{-20pt}
            \begin{itemize}
                \item[] \textcolor{UDSgreenFierte}{\faSignIn*}
                    ~Plusieurs régulateurs ont une sortie Power-Good
                \item[] \textcolor{UDSgreenFierte}{\faStopwatch}
                    ~S'active après que la sortie ait atteint un threshold
                \bigskip
                \item[] \textcolor{UDSgreenFierte}{\faLevelUp*}
                    ~Besoin d'une pull-up
                \bigskip
                \item[] \textcolor{UDSgreenFierte}{\faLightbulb}
                    ~Allumer des LEDs une fois la sortie active
                \item[] \textcolor{UDSgreenFierte}{\faProjectDiagram}
                    ~Enchaîner des régulateurs en séquence
                \item[] \textcolor{UDSgreenFierte}{\faArrowAltCircleRight}
                    ~PG dans EN en cascade
            \end{itemize}
        \end{column}

        \begin{column}{0.6\textwidth}
            \centering
            \footnotesize
            \vspace{-8pt}
            \begin{figure}
                \includegraphics[width=\textwidth, height=0.33\textheight, keepaspectratio]{pictures/power-good-sequence.png}
            \end{figure}
            \vspace{-8pt}
            \href{https://www.ti.com/lit/an/slyt598/slyt598.pdf}{TI - SLYT598 - Power-Supply Sequencing for FPGAs}
            \vfill
            \begin{figure}
                \includegraphics[width=\textwidth, height=0.33\textheight, keepaspectratio]{pictures/power-good.png}
            \end{figure}
            \vspace{-8pt}
            \href{https://www.latticesemi.com/view_document?document_id=49312}{ST - LDLN030 - Ultra-low noise LDO with PG and SS}
        \end{column}
    \end{columns}
\end{frame}

\begin{frame}{Séquençage - Multi-Output Reset IC Sequencer}
    \begin{itemize}
        \item Chips dédiés - Regardent les $V_{out}$
        \item Diviseurs de tension - UVLO
        \item Délais programmables avec condensateurs
    \end{itemize}
    \vfill
    \begin{figure}
        \includegraphics[width=\textwidth, height=0.6\textheight, keepaspectratio]{pictures/power-supply-sequencer.png}
    \end{figure}
    \vspace{-8pt}
    \href{https://www.ti.com/lit/an/slyt598/slyt598.pdf}{TI - SLYT598 - Power-Supply Sequencing for FPGAs}
\end{frame}

% Séquence brisée avec mauvaise dépendance

\begin{frame}
    \begin{itemize}
        \item Parfois la séquence ne fonctionne pas
        \item Reg $\SI{3.3}{\volt}$ doit s'allumer avant reg $\SI{5}{\volt}$
        \item Mais reg $\SI{3.3}{\volt}$ est alimenté par le reg $\SI{5}{\volt}$
        \bigskip
        \only<2-> {
            \item Utiliser une loadswitch!
        }
    \end{itemize}

    \vfill
    \centering
    \resizebox{\textwidth}{!}{
    \begin{tikzpicture}[
        block/.style = {rectangle, draw,
        minimum height=0.75cm,
        minimum width=2cm,
        align=center,
        very thick,
        rounded corners=0.1cm},
        node distance=0.8cm and 0.6cm,
        >={Stealth[round]},
        x = 3cm,
        y = 1cm
    ]

    \node(inlabel) at (0, 0) {};

    \node[block] (5v) at (0.5, 0)
        {\textcolor{UDSgreenFierte}{\faWaveSquare} ~\textcolor{black}{\SI{5}{\volt}}};
    \node[block] (fpga) at (4, -2)
        {\textcolor{UDSgreenFierte}{\faMicrochip} ~\textcolor{black}{FPGA}};
    \node[block] (3v3)   at (2,  -2)
        {\textcolor{UDSgreenFierte}{\faWaveSquare} ~\textcolor{black}{\SI{3.3}{\volt}}};

    \only<2-> {
        \node[block] (ls)  at (2,  0)
            {\textcolor{UDSgreenFierte}{\faLongArrowAltRight} ~\textcolor{black}{Loadswitch}};
    }


    \draw[->, ultra thick, color=black]
         ($(inlabel)+(-0.75, 0)$) -- (5v.west)
         node[midway, above]
         {\textcolor{black}{Input $\SI{12}{\volt}$}};

    \only<1> {
        \draw[->, ultra thick, color=red]
        (5v) -| (fpga);
    }
    \only<2-> {
        \draw[->, ultra thick, color=red]
        (5v) -- (ls);
        \draw[->, ultra thick, color=red]
        (ls) -| (fpga);
    }

    \only<2-> {
        \draw[->, ultra thick, color=cyan]
        (3v3.north) -- (ls.south)
        node[midway, right]
        {Power Good};
    }

    \draw[->, ultra thick, color=red]
         (5v) |- (3v3);

    \draw[->, ultra thick, color=blue]
         (3v3) -- (fpga);
    \end{tikzpicture}
    }
\end{frame}

\begin{frame}{Ramp-Up time}
    \begin{columns}
        \begin{column}{0.5\textwidth}
            \begin{itemize}
                \item FPGA ont parfois des requis de timing de power spéciaux
                \item En plus des requis de séquence
                \item Temps entre des alimentations
                \item Requis de slope min/max
                \bigskip
                \item Affecté par Soft-Start!
                \item Affecté par capacitance sur la ligne
            \end{itemize}
        \end{column}
        \begin{column}{0.5\textwidth}
            \begin{figure}
                \includegraphics[width=\textwidth, height=\textheight, keepaspectratio]{pictures/power-supply-ramp-up.png}
            \end{figure}
            \vspace{-8pt}
            \href{https://www.intel.com/content/www/us/en/docs/programmable/683889/current/fpga-power-supplies-ramp-time-requirement.html}{Intel - Power Supplies Ramp Time Requirement}
        \end{column}
    \end{columns}
\end{frame}

\subsection{Trucs et astuces}
\begin{frame}{Quoi mettre autour d'un régulateur}
    \begin{columns}
        \begin{column}{0.5\textwidth}
            \begin{itemize}
                \item Testpoints!
                \begin{itemize}
                    \item Paire Power-GND directement à la sortie
                    \item Sur le Power Good
                \end{itemize}

                \only<2-> {
                \item Résistances de $\SI{0}{\ohm}$
                \begin{itemize}
                    \item Aux pins que vous ne pensez pas utiliser (EN, configuration etc.)
                    \item Directement à la sortie du chip
                \end{itemize}
                }
            \end{itemize}
        \end{column}

        \begin{column}{0.5\textwidth}
            \only<3-> {
            \begin{itemize}
                \item Jumper à la sortie
                \begin{itemize}
                    \item Permet d'isoler la sortie pendant l'assemblage et les tests initiaux
                    \item Permet de faire une mesure du courant au besoin
                \end{itemize}

                \only<4-> {
                \item Shunt resistor
                \begin{itemize}
                    \item Directement à la sortie
                    \item À côté de la bobine
                \end{itemize}
                }
            \end{itemize}
            }
        \end{column}
    \end{columns}
    
    \vfill
    \begin{figure}
        \includegraphics<2->[width=0.8\textwidth, height=\textheight, keepaspectratio]{pictures/en-0r.png}
    \end{figure}

\end{frame}

\begin{frame}{Quoi optimiser}
    \begin{columns}
        \begin{column}{0.5\textwidth}
            \Large
            \centering
            \begin{tabular}{c l}
                \textcolor{UDSgreenFierte}{\faBatteryHalf} &
                    ~Efficacité Énergétique \\
                    [0.6em]
                \textcolor{UDSgreenFierte}{\faDollarSign} &
                    ~Coûts \\
                    [0.6em]
                \textcolor{UDSgreenFierte}{\faUserClock} &
                    ~Temps de conception et risque \\
                    [0.6em]
                \textcolor{UDSgreenFierte}{\faCompress*} &
                    ~Espace \\
                    [0.6em]
                \textcolor{UDSgreenFierte}{\faSignal} &
                    ~Performance et bruit\\
                    [0.6em]
            \end{tabular}
        \end{column}
        \begin{column}{0.5\textwidth}
            \begin{figure}
                \includegraphics[width=0.8\textwidth, height=0.33\textheight,
                keepaspectratio]{pictures/ptn0405aas.png}
            \end{figure}
            \vfill
            \begin{figure}
                \includegraphics[width=0.8\textwidth, height=0.33\textheight, keepaspectratio]{pictures/ire-q12.png}
            \end{figure}
        \end{column}
    \end{columns}
\end{frame}

\begin{frame}{À garder en tête}
    \begin{itemize}
        \item Plus le $\Delta V$ est petit, plus le régulateur sera efficace
        \item Ne pas trop surdimensionner les régulateurs
        \item Régulateur switching juste avant un régulateur linéaire est une très bonne option
        \begin{itemize}
            \item Efficacité d'un switching
            \item Bruit d'un régulateur linéaire
            \item $\SI{12}{\volt}$
                  ~\faLongArrowAltRight
                  ~$\SI{4}{\volt}$ 
                  ~\faLongArrowAltRight
                  ~$\SI{3.3}{\volt}$
            \item Plus coûteux
        \end{itemize}
        \item Les différents types de condensateurs ont des courbes d'impédance différentes
        \item Les IC contiennent de la capacitance dans le chip
        \item Des pads vide ça coûte rien d'autre que de l'espace
        \begin{itemize}
            \item Prévoyez plus de capacitance
            \item Prévoyez des filtre
            \item Prévoyez plus de protections antistatiques que nécessaire
            \item Les $\SI{0}{\ohm}$ sont vos amis
        \end{itemize}
    \end{itemize}
\end{frame}