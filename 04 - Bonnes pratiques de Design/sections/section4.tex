%!TEX root = ../main.tex 

\section{Bonnes pratiques de Layout}

\subsection{Routing}
% Toujours des plans continus sous les traces
%  - Ne jamais séparer les ground planes
% Un plan de GND par plan de signal
% Faire des polygones pour le power
% Faire le stackup en fonction du fanout
% Teardrops
% Ordre de routing
% Faire attention au couplage
% Angle (30° acid trap)
% Règle du 3H

\subsection{Placement}
% Met toi plus de place que tu penses
% Placer les passives de l'autre bord
% Découplage
% Garder de l'espace sur les côtés du board
% Garder de l'espace autour des trous de vis
% Garder de l'espace avec les grosses composantes
% ESD entre l'entrée de la pin et la chip
% Orientation des chips
% Vue 3D pour le placement
% Fiducials

\subsection{Silkscreen}
% Met du texte partout
%  - Tous les testpoints, toutes les LEDs, tous les connecteurs
%  - Mettre le silkscreen plus loin
%  - Toujours avoir le silkscreen dans les mêmes orientations
% Orientation du texte

\subsection{Outils}
% Faire toutes les design rules en premier
%  - Net classes & impédance
% Grille
% Utiliser les fonctions de locking et de grouping
% Passer le DRC

\subsection{Communication avec le manufacturier}
% Fabrication Notes
% Coupon de layers
% Coupon
%  - Tous les requirements les plus difficiles à faire
%  - Si ça passe pas sur le coupon, ça passe pas sur le panel
%  - Aller voir ce que le monde mettent sur le coupon

\subsection{Autre}
% Thermal Relief
% Comment router un USB-C 2.0
% Perfectionisme (savoir quand arrêter)
