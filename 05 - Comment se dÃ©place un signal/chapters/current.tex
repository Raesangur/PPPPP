%!TEX root = ../presentation.tex
\pascalbackground

\begin{frame}{Jitter}
    \begin{twocolumns}
        \leftcol
        \begin{itemize}
            \item Le Jitter est la variation du timing des fronts d'un signal
            \bigskip
            \item Par rapport au signal lui-même
            \item Par rapport à d'autres signaux (clock vs data)
            \bigskip
            \item Période à période
            \item À l'intérieur d'une période
        \end{itemize}
        \rightcol
        \begin{center}
        \begin{adjustbox}{width=\textwidth, height=0.7\textheight, keepaspectratio}%
            \begin{tikzpicture}
                \begin{axis}[
                    axis lines = left,
                    xlabel = \(t\),
                    ylabel = \(V\),
                    ymax = 1.75,
                    xmax = 7.5,
                ]
                \addplot [
                    domain=0:7.5,
                    thick,
                    color=accent,
                ]
                coordinates
                {(0, 0) (1, 0) (1.5, 1) (5, 1) (5.5, 0) (10, 0)};
                \addplot [
                    domain=0:7.5,
                    color=accent,
                    dashed
                ]
                coordinates
                {(0, 0) (0.75, 0) (1.25, 1) (4.75, 1) (5.25, 0) (10, 0)};
                \addplot [
                    domain=0:7.5,
                    color=accent,
                    dashed
                ]
                coordinates
                {(0, 0) (1.05, 0) (1.55, 1) (5.05, 1) (5.55, 0) (10, 0)};
                \addplot [
                    domain=0:7.5,
                    color=accent,
                    dashed
                ]
                coordinates
                {(0, 0) (1.33, 0) (1.83, 1) (5.33, 1) (5.83, 0) (10, 0)};

                \draw[<->, very thick]   (axis cs:1.5, 1.15) -- (axis cs:5, 1.15) node[midway, above] () {$T \pm jitter$};
                \end{axis}
            \end{tikzpicture}
        \end{adjustbox}
        \end{center}
    \end{twocolumns}
\end{frame}

\begin{frame}{Sources de jitter}
    \begin{twocolumns}
        \leftcol
            \begin{itemize}
                \item \textbf{Clock Skew}
                \begin{itemize}
                    \item Signaux qui n'arrive pas en même temps
                    \item Jitter constant
                \end{itemize}
                \item \textbf{Réflexions}
                \begin{itemize}
                    \item Mismatch d'impédance cause un rebondissement
                    \item $2^e$ voir $3^e$ signal vu à l'entrée
                \end{itemize}
                \item \textbf{Power Supply Noise}
                \begin{itemize}
                    \item Affecte le comportement des IC
                \end{itemize}
                \item \textbf{Interférence Électromagnétique}
                \begin{itemize}
                    \item Peut causer des mauvaises lectures
                \end{itemize}
                \item \textbf{Effets thermiques}
                \begin{itemize}
                    \item Modifie le comportement des IC et les valeurs des composantes
                \end{itemize}
            \end{itemize}
        \rightcol
            \begin{maketikzfigure}[1][0.4]
                \draw (0, 0) node [dipchip,
                num pins = 4,
                hide numbers,
                external pins width=0.3,
                external pad fraction=4,
                ](IC1){};

                \draw (8, 0) node [dipchip,
                num pins = 4,
                hide numbers,
                external pins width=0.3,
                external pad fraction=4,
                ](IC2){};


                \node [left, font=\tiny] at (IC1.bpin 3) {DATA};
                \node [left, font=\tiny] at (IC1.bpin 4) {CLK};

                \node [right, font=\tiny] at (IC2.bpin 1) {DATA};
                \node [right, font=\tiny] at (IC2.bpin 2) {CLK};

                \draw[thick] (IC1.pin 4) to [short] (IC2.pin 1);
                \draw[thick] (IC1.pin 3) to [short] ($(IC1.pin 3) + (1, 0)$)
                    to [short] ($(IC1.pin 3) + (1, -1)$)
                    to [short] ($(IC2.pin 2) - (1, 1)$)
                    to [short] ($(IC2.pin 2) - (1, 0)$)
                    to [short] (IC2.pin 2);
            \end{maketikzfigure}

            \begin{center}
            \begin{adjustbox}{width=\textwidth, height=0.4\textheight, keepaspectratio}%
                \begin{tikzpicture}
                    \pgfplotsset{
                        width= 360,
                        height= 170
                        }
                    \begin{axis}[
                        axis lines = left,
                        xlabel = \(t\),
                        ylabel = \(V\),
                        ymax = 1.75,
                        xmax = 12,
                    ]
                    \addplot [
                        domain=0:12,
                        thick,
                        color=accent,
                    ]
                    coordinates
                    {(0, 0) (1, 0) (1.5, 1) (5, 1) (5.5, 0) (7.5, 0) (8, 0.5) (10, 0.5) (10.5, 0) (12, 0)};
                    \end{axis}
                \end{tikzpicture}
            \end{adjustbox}
            \end{center}
    \end{twocolumns}
\end{frame}

\begin{frame}{Pourquoi le jitter?}
    \begin{twocolumns}[0.4]
        \leftcol
            \begin{itemize}
                \item Erreurs de lecture de données
                    \begin{itemize}
                        \item BER
                    \end{itemize}
                \item Pertes de clock
                \item Limite des performances
                \item Spéficication des protocoles
            \end{itemize}
        \rightcol
            \begin{center}
            \begin{adjustbox}{width=\textwidth, height=0.7\textheight, keepaspectratio}%
                \begin{tikzpicture}
                    \begin{groupplot}[
                    group style={
                        group size=1 by 2,
                        xlabels at=edge bottom,
                        xticklabels at=edge bottom,
                        vertical sep=10pt
                    },
                    width=8cm,
                    height=5cm,
                    xlabel=t,
                    xmin=0, xmax=12,
                    ymin=0, ymax=2,
                    xtick={0, 2, ..., 12},
                    tickpos=left,
                    ytick align=outside,
                    xtick align=outside,
                    samples=300
                ]
                \nextgroupplot[ylabel=DATA]
                \addplot [color=red] coordinates{(1, 0) (1.5, 1) (4.5, 1) (5, 0) (8, 0) (8.5, 1) (11, 1) (11.5, 0)};
                \coordinate (data1) at (axis cs:1.5, 0.9);
                \coordinate (data2) at (axis cs:8, -0.1);

                \nextgroupplot[ylabel=CLK]
                \addplot [color=blue] coordinates{(1, 0) (1.5, 1) (4.5, 1) (5, 0) (7.5, 0) (8, 1) (10.5, 1) (11, 0)};
                \coordinate (clk1) at (axis cs:1.5, 1.1);
                \coordinate (clk2) at (axis cs:8, 1.1);
                \end{groupplot}

                \draw[dashed, color=accent, thick] (data1) -- (clk1);
                \draw[dashed, color=accent, thick] (data2) -- (clk2);
                \end{tikzpicture}
            \end{adjustbox}
            \end{center}
    \end{twocolumns}
\end{frame}