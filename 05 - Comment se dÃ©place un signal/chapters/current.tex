%!TEX root = ../presentation.tex
\pascalbackground

\begin{frame}{Comment fonctionne un crystal}
    \begin{twocolumns}[0.66]
        \leftcol
            \begin{itemize}
                \item Oscillateur utilisant l'effet piézoélectrique
                \item Déformation mécanique d'une matrice cristalline
                \item Génère un champ électrique
                \item Effet inverse possible!
                \bigskip
                \item Fréquence de résonance
                \item Q factor beaucoup plus élevé que circuit RLC équivalent
            \end{itemize}
        \makefigure[1][0.35]{level-5/piezoelectric-effect}
        \rightcol
            \begin{maketikzfigure}[1][0.66]
                \draw (0, 6) node[vcc] {};
                \draw (0, 6) to [short] (-1, 6) to 
                [R, l=$R_1$] (-1, 4) to
                [american inductor, l=$L_1$] (-1, 2) to
                [C, l=$C_1$] (-1, 0) to
                [short] (0, 0) 
                node[ground] {};

                \draw (0, 6) to [short] (1, 6) to
                [C, l=$C_2$] (1, 0) to 
                [short] (0, 0);
            \end{maketikzfigure}
    \end{twocolumns}
\end{frame}

\begin{frame}{Comment résonne un crystal}
    \begin{twocolumns}[0.33]
        \leftcol
            \begin{maketikzfigure}
                \draw (0, 6) node[vcc] {};
                \draw (0, 6) to [short] (-1, 6) to 
                [R, l=$R_1$, color=UDSviolet] (-1, 4) to
                [american inductor, l=$L_1$, color=orange] (-1, 2) to
                [C, l=$C_1$, color=blue] (-1, 0) to
                [short] (0, 0) 
                node[ground] {};

                \draw (0, 6) to [short] (1, 6) to
                [C, l=$C_2$, color=red] (1, 0) to 
                [short] (0, 0);
            \end{maketikzfigure}
        \rightcol

            \begin{maketikzfigure}
            \begin{axis}[
              width=12cm, height=6.5cm,
              axis lines=left,
              xlabel={$f$}, ylabel={$Z$},
              xmin=0, xmax=1, ymin=0, ymax=1,
              xtick=\empty, ytick=\empty,
              clip=false
            ]

            \addplot[ultra thick, accent, smooth]
              coordinates {
                (0.02, 0.92) (0.10, 0.60) (0.22, 0.22)             % falling, capacitive
                (0.30, 0.30) (0.38, 0.55) (0.46, 0.80)             % rising to parallel resonance
                (0.52, 0.88) (0.62, 0.62) (0.78, 0.48) (0.98,0.36) % roll-off
              };

            \def\xFs{0.26}
            \def\xFr{0.50}

            \draw[dashed] (axis cs:\xFs,0) -- (axis cs:\xFs, 1);
            \draw[dashed] (axis cs:\xFr,0) -- (axis cs:\xFr, 1);

            \draw[color=UDSviolet, thick] (axis cs:0, 0.20) -- (axis cs:1, 0.20);

            \node[above] at (axis cs:\xFs, 1) {$f_s$};
            \node[above] at (axis cs:\xFr, 1) {$f_r$};

            \node[anchor=south west] at (axis cs:0.02, 1) {\small Capacitif};
            \node[anchor=south]      at (axis cs:0.38, 1) {\small Inductif};
            \node[anchor=south east] at (axis cs:0.98, 1) {\small Capacitif};

            \node[below] at (axis cs:\xFs,0) {\scriptsize Résonance Série};
            \node[below] at (axis cs:\xFr,-0.1) {\scriptsize Résonance Parallèle};

            \end{axis}
                
            \end{maketikzfigure}
    \end{twocolumns}
\end{frame}

\begin{frame}{Crystal sur un microcontrôleur}
    \ctikzset{bipoles/capacitor/height=0.5}
    \ctikzset{bipoles/capacitor/width=0.2}
    \ctikzset{logic ports/scale=0.5}
    \begin{maketikzfigure}

        \draw (4, 0) node [dipchip,
            num pins = 8,
            hide numbers,
            external pins width=0.3,
            external pad fraction=4,
            ](IC){};

            \node [right, font=\tiny] at (IC.bpin 1) {OSC2};
            \node [right, font=\tiny] at (IC.bpin 4) {OSC1};


            \draw[thick] (IC.pin 1) to [short] ($(IC.pin 1) - (1.5, 0)$) to
            [piezoelectric, color=accent] ($(IC.pin 4) - (1.5, 0)$) to
            [short] (IC.pin 4);

            \only<1, 3>{
                \draw[thick] ($(IC.pin 1) - (1.5, 0)$) to
                [C, l=$C_1$] ($(IC.pin 1) - (3.5, 0)$);
                \draw[thick] ($(IC.pin 4) - (1.5, 0)$) to
                [C, l=$C_2$] ($(IC.pin 4) - (3.5, 0)$);
            }
            \only<2, 4->{
                \draw[thick] ($(IC.pin 1) - (1.5, 0)$) to
                [C, l=$C_1$, color=accent2] ($(IC.pin 1) - (3.5, 0)$);
                \draw[thick] ($(IC.pin 4) - (1.5, 0)$) to
                [C, l=$C_2$, color=accent2] ($(IC.pin 4) - (3.5, 0)$);
            } 

            \draw[thick] ($(IC.pin 1) - (3.5, 0)$) to [short] ($(IC.pin 4) - (3.5, 0)$)
            to [short]   ($(IC.pin 4) - (3.5, 1)$) node[ground]{};


            \only<3->{
                \draw[dashed] (IC.bpin 1) to [short] ($(IC.pin 1) + (1.5, 0)$) to [open]
                ($(IC.pin 4) + (1.5, 0)$) to [short] (IC.bpin 4);

                \draw ($(IC.pin 1) + (1.5, 0)$) to [inline not, color=accent2] ($(IC.pin 4) + (1.5, 0)$);
            }

            \only<4->{
                \draw[->, color = accent]    ($(IC.pin 1) + (2, 0)$)    -- ($(IC.pin 4) + (2, 0)$)    node[midway, below, sloped] {$+\ang{180}$};
                \draw[<-, color = UDSviolet] ($(IC.pin 1) - (0.75, 0)$) -- ($(IC.pin 4) - (0.75, 0)$) node[midway, below, sloped] {$+\ang{180}$};
            }
    \end{maketikzfigure}
\end{frame}

\begin{frame}{Condensateur de charge}
    \begin{twocolumns}
        \leftcol
        \only<1>{
        \begin{itemize}
            \item Les circuits de lecture CMOS d'oscillateur rajoutent une phase de $\ang{180}$
            \item Phase doit être compensée pour atteindre $\ang{360}$
            \item Condensateurs viennent compléter le modèle RLCC du crystal
            \item Crystaux calibrés pour des valeurs de condensateurs $C_L$
        \end{itemize}
        }

        \only<2->{
            \begin{tabular}{l | c}
                \multicolumn{2}{c}{\textbf{\textbf{ESC-16-20S-4X}}}\\
                \hline
                \textbf{Frequency}             & $\SI{16}{\mega\hertz}$\\
                \textbf{Frequency Tolerance}   & $\pm 30\,ppm$\\
                \textbf{Frequency Stability}   & $\pm 50\,ppm$\\
                \textbf{Shunt Capacitance}     & $\SI{7}{\pico\farad}$\\
                \textbf{Load Capacitance}      & $\SI{20}{\pico\farad}$\\
                \textbf{Operating Temperature} & $\SI{-10}{\celsius} à \SI{70}{\celsius}$\\
                \textbf{Storage Temperature}   & $\SI{-55}{\celsius} à \SI{125}{\celsius}$\\
                \textbf{ESR Max}               & $\SI{40}{\ohm}$\\
            \end{tabular}
        }

        \rightcol
        \only<2>{
            \makefigure[1][0.25]{level-5/hc49}
        }
        \begin{align*}
            C_1 &= C_2 = 2 \cdot (C_L - C_{stray})\\
            C_L &= \dfrac{C_1 \cdot C_2}{C_1 + C_2} + C_{stray}\\
            \vspace{18pt}
            C_{stray} \sim \SI{5}{\pico\farad}\\
        \end{align*}
        \only<3->{
            \begin{align*}
                C_1 &= C_2 = 2 \cdot (\SI{20}{\pico\farad} - \SI{10}{\pico\farad})\\
                C_1 &= C_2 = \SI{20}{\pico\farad}\\
            \end{align*}
        }
    \end{twocolumns}
\end{frame}

\begin{frame}{Types d'oscillateurs}
    \begin{tabular}{c | c | c | c}
        \textbf{XO} & \textbf{VCXO} & \textbf{TCXO} & \textbf{OCXO}\\
        \hline
        \textit{Crystal Oscillator} & 
        \textit{\textbf{V}oltage-\textbf{C}ontrolled} &
        \textit{\textbf{T}emperature-\textbf{C}ontrolled} &
        \textit{\textbf{O}ven-\textbf{C}ontrolled}\\
    \end{tabular}
\end{frame}