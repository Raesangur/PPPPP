%!TEX root = ../presentation.tex



% Insert a figure with optional scaling and caption.
% 1: Width as a fraction of \textwidth   (default: 1)
% 2: Height as a fraction of \textheight (default: 0.8)
% 3: Caption text (optional)
% 4: Filename of the image (relative to 'pictures/' directory, no extension needed)
%
% Example:
% \makefigure[0.9][0.5][A sample image]{example-image}
\NewDocumentCommand{\makefigure}{O{1} O{0.8} o m}{%
    \begin{figure}%
        \centering%
        \includegraphics%
            [width=#1\textwidth, height=#2\textheight, keepaspectratio, page=1]%
            {pictures/#4}%
        \IfValueTF{#3}{%
            \caption*{#3}%
        }{}%
    \end{figure}%
}


% Insert a figure with border and with optional scaling and caption.
% 1: Width as a fraction of \textwidth   (default: 1)
% 2: Height as a fraction of \textheight (default: 0.8)
% 3: Caption text (optional)
% 4: Filename of the image (relative to 'pictures/' directory, no extension needed)
%
% Example:
% \makefigureborder[0.75][0.7][A sample image]{example-image}
\NewDocumentCommand{\makefigureborder}{O{1} O{0.8} o m}{%
    \begin{figure}%
        \centering%
        \tcbox[colframe=accent, colback=background]{
            \includegraphics%
                [width=#1\textwidth, height=#2\textheight, keepaspectratio, page=1]%
                {pictures/#4}%
            \IfValueTF{#3}{%
                \caption*{#3}%
            }{}%
        }
    \end{figure}%
}


% Create a TikZ circuit figure with adjustable size.
% 1: Width as a fraction of \textwidth   (default: 1)
% 2: Height as a fraction of \textheight (default: 0.8)
%
% Example:
% \begin{maketikzfigure}[0.7][0.5]
%     \draw (0,0) to[battery] (0,2);
% \end{maketikzfigure}
\NewDocumentEnvironment{maketikzfigure}{O{1} O{0.8}}{%
    \vspace{-16pt}%
    \begin{center}%
        \begin{adjustbox}{width=#1\textwidth, height=#2\textheight, keepaspectratio}%
            \begin{circuitikz}[american voltages]%
}
{%
            \end{circuitikz}%
        \end{adjustbox}%
    \end{center}%
}




% Begin a two-column layout with adjustable left column width.
% 1: Width of the left column as a fraction of \textwidth (default: 0.5)
%    The right column will take the remaining space after the left column
%
% Example:
% \begin{twocolumns}[0.6]
%   \leftcol
%     Left side content.
%   \rightcol
%     Right side content.
% \end{twocolumns}
\NewDocumentEnvironment{twocolumns}{O{0.5}}{%  
  \def\leftcolwidth{#1\textwidth}%
  \def\rightcolwidth{\dimexpr \textwidth - #1\textwidth\relax}%
  
  \begin{columns}%
}{%
  \end{column}%
  \end{columns}%
}

\NewDocumentCommand{\leftcol}{}{%
  \begin{column}{\leftcolwidth}%
}

\NewDocumentCommand{\rightcol}{}{%
  \end{column}%
  \begin{column}{\rightcolwidth}%
}


% Color an icon
% 1: Color name   (optional, default: accent)
% 2: Icon command (e.g., \faCheck)
%
% Example:
% \icon{\faCheck}
% \icon[red]{\faTimes}
\newcommand{\icon}    [2][accent] {\textcolor{#1}{#2}}

% Display an icon at the start of a list item with customizable color and spacing.
% 1: Color name       (optional, default: accent)
% 2: Horizontal space (optional, default: -12pt)
% 3: Icon command     (e.g., \faCheck)
%
% Example:
% \item[] \itemicon           {\faCheck}
% \item[] \itemicon[gray]     {\faCircle}
% \item[] \itemicon[red][-6pt]{\faTimes}
\NewDocumentCommand{\itemicon}{O{accent} O{-12pt} m}{\hspace{#2}\icon[#1]{#3}}



% Create a two-column list using a tabular environment.
% 1: Font size or formatting command (optional, default: \normalsize)
% 2: Row spacing via \arraystretch   (optional, default: 1.25)
% 3: Tabular column format           (optional, default: c l)
%
% Example:
% \begin{makelist}[\small][1.5]
%   \icon{\faCheck} & Item one \\
%   \icon{\faTimes} & Item two \\
% \end{makelist}
\NewDocumentEnvironment{makelist}{O{\normalsize} O{1.25} O{c l}}{%
    #1%
    \renewcommand{\arraystretch}{#2}%
    \begin{tabular}{#3}%
}{%
    \end{tabular}%
    \renewcommand{\arraystretch}{1}%
}


%Roman Numerals
\makeatletter
\newcommand*{\rom}[1]{\expandafter\@slowromancap\romannumeral #1@}
\makeatother
