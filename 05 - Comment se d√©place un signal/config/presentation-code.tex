%!TEX root = ../presentation.tex


% ----------------- Vectorial Quantities Shortcut CODE -----------------

% To help write complex vectorial equation
%
% Example:
% I have a big \eflux

\newcommand{\ddt}[1]{\frac{\delta {#1}}{\delta t}}
\newcommand{\nabv}{\vec{\nabla}}

\newcommand{\efield}{\vec{\mathcal{E}}}
\newcommand{\eflux}{\vec{\mathcal{D}}}
\newcommand{\mfield}{\vec{\mathcal{H}}}
\newcommand{\mflux}{\vec{\mathcal{B}}}

\newcommand{\ecurrent}{\vec{\mathcal{J}}}
\newcommand{\Icurrent}{\vec{\mathcal{I}}}
% ----------------- INLINE CODE -----------------

% Display monospace text in a grey box. Similar to ` text in markdown.
% 1: Color of the text in the textbox - default: [black]
% 2: Text to display
%
% Note: This command sometimes continues in the margins of the page, 
% putting the punctuation as the 3rd argument prevents the following punctuation
% to be alone at the start of the next line.
%
% Example:
% This is an \inline{example} without a specified color.
% This is another \href{https://www.example.com}{\inline[blue]{example}}
\newcommand{\inline}[2][black]{ %
  \hspace{-8pt}%
  \begingroup%
  \raggedright%
  {
    \mbox{%
      \raggedright\tcbox[on line,%
                         boxsep=4pt, left=-1pt,right=-1pt,top=-4pt,bottom=-4.5pt,%
                         opacityframe=0, colback=gray!50,%
                         fontupper={\strut},%
                         enhanced, breakable]%
      {%
        \raggedright\lstinline[basicstyle=\ttfamily\small\color{#1},%
                               breaklines=true, breakatwhitespace=true,%
                               moredelim={[s][\ttfamily]{_}{_}}]%
      {#2}%
      }%
    }%
  }
  \endgroup%
  \hspace*{-8pt}~%
}


% --------------- REGULAR LISTING ---------------

% Create a lstlisting environment with custom syntax highlighting.
% 1: Syntax highlighting language - default: [logbook]
% 
% Note: The syntax highlighting argument is currently unused, as no other styles than logbook have been defined.
%
% Example:
% \begin{makelisting}{python}
% if __name__ == "__main__":
%     print("Example")
% \end{makelisting}
\lstnewenvironment{makelisting}[2][]
{
  \vspace{-8pt}
  \lstset{style=logbook #1}
}
{
  \vspace{-12pt}
}

% Create two aligned columns with proper spacing.
% Used to compare two listings or figures easily.
%
% Example:
% \begin{makecompare}
%   This is the first example, on the left.
%   \newcol
%   This second example is on the right!
% \end{makecompare}
\newenvironment{makecompare}
{
  \vspace{-1.25\cringlineskip}
  \begin{multicols}{2}
}
{
  \end{multicols}
  \vspace{-2\cringlineskip}
}


 % ----------------- CODE BLOCK -----------------
% https://tex.stackexchange.com/a/468526
\newtcbinputlisting[auto counter, list inside = lol, list type = {lstlisting}]{\makecode}[3][logbook]{
  breakable,
  listing file = {code/#3},
  listing options={style = logbook},
  listing only,
  boxrule = 1pt,
  title = {\textbf{Code \thetcbcounter:} \textbf{#2} \hfill \textbf{#3}},
  label = code:#3
}


% ---------------- CODE FORMATS -----------------
\lstdefinestyle{logbook}{
  escapeinside={<@}{@>},
  language=C,
  aboveskip=0.5cm,
  breakatwhitespace=false,
  breaklines=true,
  numbers=left,
  numbersep=8pt,
  numberfirstline = false,
  linewidth=\textwidth,
  stepnumber=1,
  frame=lines,
  framesep=0pt,
  framerule=0pt,
  framextopmargin=3pt,
  framexbottommargin=3pt,
  framexleftmargin=0.4cm,
  xleftmargin={0.75cm},
  rulecolor=\color{Black},
  rulesep=.4pt,
  %backgroundcolor=\color{background},
  basicstyle=\small\ttfamily,
  identifierstyle=\color{RoyalBlue},
  commentstyle=\color{ForestGreen}\itshape,
  keywordstyle=\color{Plum}\bfseries,
  numberstyle=\small\ttfamily,
  stringstyle=\ttfamily\color{RedOrange},
  showstringspaces=false,
  showspaces=false,
  keepspaces=true,
  showtabs=false,
  tabsize=4,
  captionpos=t,
}
