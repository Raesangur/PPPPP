%!TEX root = ../presentation.tex

\section[Level 2]{Current Paths [30min-50min]}


\subsection[2min-Pascal]{Signal Source \rom{1}}
\begin{frame}{Comparaison des sources}
    \begin{twocolumns}
        \leftcol
        \begin{center}
            \textbf{Source de tension}
        \end{center}
        \rightcol
        \begin{center}
            \textbf{Source de courant}
        \end{center}
    \end{twocolumns}

    \begin{twocolumns}[0.55]
        \leftcol
        \begin{itemize}
            \item Modèle plus traditionnel
            \item Fournit une tension fixe peu importe le courant demandé
            \item Tension cause un courant au travers d'une load
        \end{itemize}

        \rightcol
        \begin{itemize}
            \item Fournit un courant fixe peu importe la tension requise pour driver ce courant demandé
            \item Courant au travers d'une load crée une tension
        \end{itemize}
        
    \end{twocolumns}

    \vspace{-8pt}
    \begin{twocolumns}
        \leftcol
        \begin{center}
            $I = \dfrac{V}{R}$
        \end{center}
        \rightcol
        \begin{center}
            $V = RI$
        \end{center}
    \end{twocolumns}

    \vfill

    \begin{twocolumns}
        \leftcol
        \begin{maketikzfigure}[1][0.25]
            \draw (0, 0) node[ground]{}
                to [V, l=$V_S$, invert] ++(0, 2.5)
                to [short] ++(2.5, 0)
                to [R, l_=$R_{LOAD}$,i={$I = \frac{V_S}{R_{LOAD}}$}] ++(0, -2.5)
                to node[ground]{} ++(0, 0);
        \end{maketikzfigure}
        \rightcol
        \begin{maketikzfigure}[1][0.25]
            \draw (0, 0) node[ground]{}
                to [I, l=$I_S$] ++(0, 2.5)
                to [short, i=$I_S$] ++(2.5, 0)
                to [R, l_=$R_{LOAD}$,v^={$V = I_S \cdot R_{LOAD}$}] ++(0, -2.5)
                to node[ground]{} ++(0, 0);
        \end{maketikzfigure}
    \end{twocolumns}
\end{frame}

\begin{frame}{Exemples de sources}
    \begin{itemize}
    \only<1->{
        \item[] \itemicon{\faBatteryThreeQuarters} \textbf{Batteries}
        \begin{itemize}
            \item[] Réaction chimique qui fournit un potentiel de tension fixe aux bornes de la cellule.
        \end{itemize}
    }
    \only<2->{
        \item[] \itemicon{\faSun} \textbf{Effect Photovoltaïque}

        \begin{itemize}
            \item[] Excitation d'électrons frappés par des photons dans un matériau semiconducteur, causant leur déplacement.
        \end{itemize}
    }
    \only<3->{
        \item[] \itemicon{\faThermometerHalf} \textbf{Effect Seebeck}
        \begin{itemize}
            \item[] Effet thermoélectrique qui produit un différentiel de tension entre deux métaux différents qui ont des énergies thermiques différentes.
        \end{itemize}
    }
    \only<4->{
        \item[] \itemicon{\faVolumeUp} \textbf{Effect Piézoélectrique}
        \begin{itemize}
            \item[] Charge électrique provenant d'un matériau dont la matrice crystalline est déformée mécaniquementé.
        \end{itemize}
    }
    \only<5->{
        \item[] \itemicon{\faBolt} \textbf{Transistor Bipolaire --- Miroir de Courant}
        \begin{itemize}
            \item[] Deux effets qui répliquent un courant (avec un multiplicateur) en utilisant des transistors.
        \end{itemize}
    }
    \only<5->{
        \item[] \itemicon{\faSync} \textbf{Induction Électromagnétique}
        \begin{itemize}
            \item[] Le sujet de cette présentation.
        \end{itemize}
    }
    \end{itemize}
\end{frame}

\begin{frame}{Sources dans la vraie vie}
    \begin{itemize}
        \item Une vraie source de courant n'existe pas
        \item Une vraie source de tension n'existe pas
        \bigskip
        \item Tout ce qui existe sont des sources de puissances
        \item Les "sources" utilisent du feedback pour réguler courant et/ou tension
        \bigskip
        \item Il existe des sources de résistance aussi
    \end{itemize}
\end{frame}

\subsection[3min - Max]{Harmonics \rom{1}}
\begin{frame}{Plan}
    \begin{makelist}[\small][1.5]
        \icon[red]{\faTimes} & Transformé de fourier\\
        \icon[red]{\faTimes} & Addition de Signaux \\
        \icon[red]{\faTimes} & Taylor \\
        \icon[red]{\faTimes} & Harmonique paires/impaires
    \end{makelist}
\end{frame}

\begin{frame}{Square Wave}
    \makefigure[1.0][0.7][]{level-2/harmonic-meme}
\end{frame}

\subsection[5min-Pascal]{Propagation Speed \rom{1}}
\begin{frame}{Vitesse de propagation}
    \begin{twocolumns}
        \leftcol
        \begin{makelist}[\small][1.5]
            \icon{\faLongArrowAltRight} & Tout signal électromagnétique a une $v$\\
            \icon{\faSun} & Vitesse de la lumière: $c = \SI{299792458}{\meter\per\second}$\\
            \icon{\faHockeyPuck} & Vitesse d'un signal dépend de la constante diélectrique\\
        \end{makelist}

        \begin{center}
            $v = \dfrac{c}{\sqrt{\varepsilon_r}}$
        \end{center}

        \rightcol
        \maketable{speed-of-light}
    \end{twocolumns}
\end{frame}

\begin{frame}{Phase Velocity}
    \begin{twocolumns}[0.33]
        \leftcol
        \begin{makelist}[\small][1.5]
            \icon{\faStumbleupon} Dépend de la fréquence du signal
        \end{makelist}

        \begin{center}
            $v_p = \dfrac{\lambda}{T}$\\
            \vspace{6pt}
            $n = \dfrac{c}{v_p}$
        \end{center}

        \rightcol
        \maketable{phase-velocity}
    \end{twocolumns}
\end{frame}


\subsection[5min-Pascal]{Ground \rom{1}}

\begin{frame}{Différentes catégories de GND}
    \begin{itemize}
        \item Parfois plusieurs GND dans un même circuit
        \item Les GND agissent comme une \textit{référence}
        \item \textit{Parfois} le GND est connecté à la terre / au chassis
        \item Équipotentialité
    \end{itemize}
    \vfill

    \begin{columns}
        \begin{column}{0.33\textwidth}
            \begin{center}
            \begin{maketikzfigure}[0.33][0.2]
            \draw (0, 0) node[sground]{};
            \end{maketikzfigure}
            Ground
            \end{center}
        \end{column}
        \begin{column}{0.33\textwidth}
            \begin{center}
            \begin{maketikzfigure}[0.33][0.2]
            \draw (0, 0) node[ground]{};
            \end{maketikzfigure}
            Earth
            \end{center}
        \end{column}
        \begin{column}{0.33\textwidth}
            \begin{center}
            \begin{maketikzfigure}[0.33][0.2]
            \draw (0, 0) node[cground]{};
            \end{maketikzfigure}
            Chassis
            \end{center}
        \end{column}
    \end{columns}
\end{frame}

\begin{frame}{Dans un circuit}
    \vfill
    \begin{maketikzfigure}[0.8][0.66][european voltages]
        \ctikzset{voltage/bump b=2.5} % Adjust arrow bumpiness for voltage

        \only<1-3>{
            \draw [thick]
                (0,0) to [short] (10,0);
        }

        \draw [thick]
            (0,0) to [american voltage source, l=$V$, invert] (0,5);


        \only<1>{
        \draw [thick]
            (0, 5) to [short] (10,5);
        }
        \only<2>{
        \draw [thick]
            (0, 5) to [short] (2.5, 5)
                   to [R, l=$R_{parasite}$, v=$i \cdot R_{parasite}$, color=accent] (7.5, 5)
                   to [short] (10, 5);
        }
        \only<3->{
        \draw [thick]
            (0, 5) to [short] (2.5, 5)
                   to [R, l=$R_{parasite}$] (7.5, 5)
                   to [short] (10, 5);
        }

        \only<4->{
        \draw [thick]
            (0, 0) to [short] (2.5, 0)
                   to [R, l=$R_{parasite}$, color=accent2] (7.5, 0)
                   to [short] (10, 0);
        }

        \only<3>{
            \node[thick, color=accent, font=\boldmath] at (0,  5.5) {$V_A$};
            \node[thick, color=accent, font=\boldmath] at (10, 5.5) {$V_B$};
        }
        \only<4->{
            \node[thick] at (0,  5.5) {\textbf{$V_A$}};
            \node[thick] at (10, 5.5) {\textbf{$V_B$}};
        }

        \only<1>{
        \draw [thick]
            (10, 5)
            to [R, l_=$R$, i_={$i = \frac{V}{R}$}] (10,0);
        }
        \only<2>{
        \draw [thick]
            (10, 5)
            to [R, l_=$R$, i_={$i = \frac{V}{R + R_{parasite}}$}] (10,0);
        }
        \only<3->{
        \draw [thick]
            (10, 5)
            to [R, l_=$R$, i_=$i$] (10,0);
        }
    \end{maketikzfigure}
    \vfill
\end{frame}

\begin{frame}{Diapositive la plus importante}
    \only<2->{
    \begin{center}
        \textcolor{UDSgreenFierte}{\faLongArrowAltRight}
        \textbf{Tous les grounds ne sont pas égaux!}
        \textcolor{UDSgreenFierte}{\faLongArrowAltLeft}
    \end{center}
    }
\end{frame}


\begin{frame}{Le Ground n'est pas un sink magique}
    \begin{makelist}[\small][1.5]
        \icon[red]{\faTimes} & Tout le courant doit s'en aller quelque part\\
        \icon[red]{\faTimes} & Le ground est utile comme \textit{référence de tension}\\
        \icon[red]{\faTimes} & Le ground n'est pas nécessairement un \textit{sink}
    \end{makelist}

    \vfill

    \begin{maketikzfigure}[0.8][0.4]
        \draw [thick]
            (0,0) to [short] (10,0);

        \draw [thick]
            (0,0) to [american voltage source, l=$V$] (0,5);

        \draw [thick]
            (10, 5) to [short, i={$i = \frac{V}{R}$}] (0,5);

        \draw [thick]
            (10, 5) to [R, l_=$R$] (10,0);
    \end{maketikzfigure}
\end{frame}


\subsection[5min-Max]{Induction}
\begin{frame}{Plan}
    \begin{makelist}[\small][1.5]
        \icon[red]{\faTimes} & Comment les courants sont induits \\
        \icon[red]{\faTimes} & Self-Induction dans une boucle\\
        \icon[red]{\faTimes} & Regle de la main droite\\
        \icon[red]{\faTimes} & Item 3
    \end{makelist}
\end{frame}

\subsection[5min-Pascal]{Current loops}
\begin{frame}{Plan}
    \begin{makelist}[\small][1.5]
        \icon[red]{\faTimes} & GND Loop avec cable(Ou on place ca apres la section noise?)\\
        \icon[red]{\faTimes} & Frequency dependant loop\\
        \icon[red]{\faTimes} & Item 3
    \end{makelist}
\end{frame}

\begin{frame}{Quote}
    \begin{makelist}[\small][1.5]
        \icon[red]{\faTimes} & L'Électricité prend le chemin le plus court\\
        \icon[green]{\faCheck} & L'Électricité prend le chemin avec la plus petite Resistance\\ 
    \end{makelist}
\end{frame}

\subsection[3min-Max]{Radiation \rom{1}}
\begin{frame}{Plan}
    \begin{makelist}[\small][1.5]
        \icon[red]{\faTimes} & Inductor field\\
        \icon[red]{\faTimes} & Sun Coronal Mass ejection Analogy\\
        \icon[red]{\faTimes} & Simple Travelling wave\\
        \icon[red]{\faTimes} & Wavelength\\
        \icon[red]{\faTimes} & Induction is actually radiation
    \end{makelist}
\end{frame}

\begin{frame}{Let's talk about Auroras}
\begin{twocolumns}[0.3]
    \leftcol
        \makefigure[1.0][0.7][Auroras]{level-2/auroras}
    \rightcol
        \makefigure[1.0][0.7][Sun Flare]{level-2/CME-blast}
    \end{twocolumns}
\end{frame}

\begin{frame}{Coronal Mass Ejection}
    \begin{twocolumns}[0.5]
        \leftcol
            \makefigure[1.0][0.7][Auroras]{level-2/coronal-mass-ejection}
        \rightcol
            \makefigure[1.0][0.7][Sun Flare]{level-2/coronal-mass-ejection-step}
    \end{twocolumns}
\end{frame}

\begin{frame}{How it relates to an Inductor}
    \begin{twocolumns}[0.5]
        \leftcol
            \makefigure[1.0][0.7][Inductor]{level-1/inductor}
        \rightcol
            \makefigure[1.0][0.7][Inductor Magnetic field]{level-1/inductor-magnetic-field}
    \end{twocolumns}
\end{frame}

\begin{frame}
    \begin{twocolumns}[0.5]
        \leftcol
            \makefigure[1.0][0.7][Magnetic field loop]{level-2/coronal-mass-ejection-loop}
        \rightcol
            \makefigure[1.0][0.7][Inductor Magnetic field]{level-1/inductor-magnetic-field}
    \end{twocolumns}
\end{frame}

\begin{frame}
    \begin{twocolumns}[0.5]
        \leftcol
            \makefigure[1.0][0.7][Magnetic Field Ejection]{level-2/coronal-mass-ejection-step}
        \rightcol
            \makefigure[1.0][0.7][Inductor Magnetic field]{level-1/inductor-current}
    \end{twocolumns}
\end{frame}

\subsection[5min-Pascal]{Fil d'une année lumière de long }
% Parasitiques
\begin{frame}{Plan}
    \begin{makelist}[\small][1.5]
        \icon[red]{\faTimes} & Item 1\\
        \icon[red]{\faTimes} & Item 2\\
        \icon[red]{\faTimes} & Item 3
    \end{makelist}
\end{frame}