%!TEX root = ../presentation.tex 



\begin{comment}
\section{Présentation de la présentation}

\subsection{Fonctionnement du projet}

\begin{frame}{Architecture du projet}
    \begin{twocolumns}
        \leftcol
        \begin{itemize}
            \item /chapters
            \begin{itemize}
                \item /chapters/chapter-1.tex
                \item /chapters/chapter-2.tex
                \item \dots
                \item /chapters/current.tex
            \end{itemize}
            \item /config
            \begin{itemize}
                \item /config/presentation-beamer.tex
                \item /config/presentation-code.tex
                \item /config/presentation-format.tex
                \item /config/presentation-objects.tex
                \item /config/presentation-packages.tex
                \item /config/presentation-sections.tex
            \end{itemize}
        \end{itemize}
        \rightcol
        \begin{itemize}
            \item /pictures
            \begin{itemize}
                \item /pictures/background
                \item /pictures/logo
                \item /pictures/raw
                \item /pictures/\dots .pdf
                \item /pictures/\dots .png
            \end{itemize}
            \item /tikz
            \item presentation-config.tex
            \item presentation.tex
            \item presentation.pdf
            \item Makefile
        \end{itemize}
    \end{twocolumns}
\end{frame}



\begin{frame}{Building}
    \begin{itemize}
        \item Template buildé local avec TeXLive
        \item Buildé avec un \inline{makefile}
        \bigskip
        \only<2->{
        \item \inline{make clean}
        \only<3->{
        \begin{itemize}
            \item Supprime tous les fichiers non-source du projet
        \end{itemize}
        }
        \item \inline{make format}
        \only<3->{
        \begin{itemize}
            \item Recrée le fichier de format précompilé
            \item À refaire à chaque modification aux fichiers dans \inline{config/}
            \item Permet d'accélérer grandement le build du projet
        \end{itemize}
        }
        \item \inline{make build}
        \only<3->{
        \begin{itemize}
            \item Commande principale, build le pdf
        \end{itemize}
        }
        \item \inline{make simple}
        \only<3->{
        \begin{itemize}
            \item Build le pdf mais ne gère pas la bibliographie ou des affaires récursives
            \item page numbers might be wrong
        \end{itemize}
        }
        }
    \end{itemize}
\end{frame}


\subsection{Outils et commandes}
\begin{frame}{Frame avec deux colonnes}
    \begin{twocolumns}
        \leftcol
        \begin{itemize}
            \item On utilise un environnement \inline{/twocolumns}
            \begin{itemize}
                \item \inline{/leftcol} Commence la colonne gauche
                \item \inline{/rightcol} Commence la colonne droite
                \item Il faut toujours les deux!
            \end{itemize}
        \end{itemize}
        \rightcol
        \begin{itemize}
            \item L'espacement des colonnes se fait avec un paramètre optionnel
            \begin{itemize}
                \item \inline{/begin\{twocolumns\}[0.66]}
            \end{itemize}
        \end{itemize}
    \end{twocolumns}
\end{frame}

\begin{frame}{Figures}
    \begin{twocolumns}[0.4]
        \leftcol
        \begin{itemize}
            \item Toujours avoir de la transparence
            \item en \inline{.pdf} ou \inline{.png}
            \item Si une image est modifiée, l'image source devrait être dans \inline{/pictures/raw/}
            \begin{itemize}
                \item Transparence
                \item Cropping
                \item Flèches \& shits
            \end{itemize}
        \end{itemize}
        \makefigure[0.5]{mcp3561-ic}
        \rightcol
        \makefigure[1][0.75][Figure avec une caption]{s7-app1-p5-oscillator}
    \end{twocolumns}
\end{frame}

\begin{frame}{Animations}
    \begin{itemize}
        \item<1-> Toujours hardcodé
        \item<2-> Jamais de \inline{/pause}
        \item<3>  Cet item disparaît à la prochaine slide
        \item<4-> Préférablement utiliser des \inline{/only<>}
        \only<4>{
            \item Cet item disparaît de façon propre à la prochaine slide
        }
        \only<6->{
            \bigskip
        }
        \item<5->[] \itemicon{\faCheck} On peut utiliser des icônes de \inline{fontawesome5}
        \item<6->[] \itemicon{\faCircle} Les icônes remplacent des bulletpoints avec \inline{/itemicon}
        \item<6-> Icônes normales avec \inline{/icon} \icon{\faChartLine}
        \item<6-> Icônes de couleur avec \inline{/icon[red]} \icon[red]{\faCity}
        \bigskip
        \item<7-> Les colonnes et icônes sont genre \inline{config/presentation-objects.tex}
        \item<7-> Avec les figures!
    \end{itemize}
\end{frame}

\begin{frame}{Bullet points}
    \centering
    \begin{makelist}[\normalsize][1.5]
        \icon{\faCheck}          & Item one \\
        \icon{\faTimes}          & Item two \\
        \icon{\faList}           & Liste d'item avec des icônes \\
        \icon{\faClipboardCheck} & Mieux qu'un \inline{/itemize} \\
        \icon[red]{\faClock}     & Un peu plus long à faire, ça peut être une passe finale \\
    \end{makelist}
\end{frame}

\end{comment}