%!TEX root = ../presentation.tex 

\section[Level 1]{Surface Ripple [20min]}
\begin{frame}{Introduction}
 Introduction des mathematiques et équation fondamentales à l'electromagnetisme
\end{frame}


\subsection[10min - Max]{EM Fields \rom{1}}
\begin{frame}{Plan}
    \begin{makelist}[\small][1.5]
        \icon{\faCheck} & Champ Vectoriel\\
        \icon[red]{\faTimes} & Divergente, Rotationnelle\\
        \icon[red]{\faTimes} & Regle de la main droite \\
        \icon[red]{\faTimes} & Equation de Maxwell
    \end{makelist}
\end{frame}

\begin{frame}{Champ Vectoriel}
\end{frame}

\begin{frame}{Divergente}

\end{frame}

\begin{frame}{Curl}
\begin{twocolumns}[0.5]
    \leftcol
        \makefigure[1.0][0.8][Cyclone Analogy]{level-1/cyclone}
    \rightcol
        \makefigure[1.0][0.6][Rotating Ball Analogy]{level-1/curl-ball}
 \end{twocolumns}
\end{frame}

\subsection[1min - Max]{Superposition \rom{1}}
\begin{frame}{Plan}
    \begin{makelist}[\small][1.5]
        \icon[red]{\faTimes} & Équation Linéaire\\
        \icon[red]{\faTimes} & Addition de Signaux
    \end{makelist}
\end{frame}

\begin{frame}{Green Theorem}
    \begin{twocolumns}[0.5]
        \leftcol
            \makefigure[1.0][0.7][Reddit moment]{level-1/green-theorem-meme}
        \rightcol
            \makefigure[1.0][0.7][Inductor Magnetic field]{level-1/inductor-magnetic-field}
    \end{twocolumns}
\end{frame}

\subsection[4min - Max]{Charge Movement}
\begin{frame}{Plan}
    \begin{makelist}[\small][1.5]
        \icon[red]{\faTimes} & Comment les Electrons bougent\\
        \icon[red]{\faTimes} & Propriété materiaux
    \end{makelist}
\end{frame}


\subsection[3min - Max]{Passive Components \rom{1}}
\begin{frame}{Plan}
    \begin{makelist}[\small][1.5]
        \icon[red]{\faTimes} & Resistance\\
        \icon[red]{\faTimes} & Condensateur \\
        \icon[red]{\faTimes} & Inducteur
    \end{makelist}
\end{frame}