%!TEX root = ../presentation.tex 

\section[Level 1]{Surface Ripple [20min]}
\begin{frame}{Introduction}
 Introduction des mathematiques et équation fondamentales à l'electromagnetisme
\end{frame}


\subsection[10min - Max]{EM Fields \rom{1}}
\begin{frame}{Plan}
    \begin{makelist}[\small][1.5]
        \icon{\faCheck} & Champ Vectoriel\\
        \icon[red]{\faTimes} & Divergente, Rotationnelle\\
        \icon[red]{\faTimes} & Regle de la main droite \\
        \icon[red]{\faTimes} & Equation de Maxwell
    \end{makelist}
\end{frame}

\begin{frame}{Champs Vectoriel}
        \begin{twocolumns}[0.4]
        \leftcol
            \makefigure[1.0][0.8][]{level-1/vector-field-1}
        \rightcol
           \makefigure[1.0][0.8][]{level-1/vector-field-2}
    \end{twocolumns}
\end{frame}

\begin{frame}{Dérivée Temporelle vs. Spatiale}
    \begin{twocolumns}[0.5]
        \leftcol
            Dérivée Temporelle
        \rightcol
           Dérivée Spatiale:
            \begin{equation}
                \text{Divergence} = \nabv \cdot \vec{V}
            \end{equation}
            \begin{equation}
                \text{Curl} = \nabv \times \vec{V}
            \end{equation}
    \end{twocolumns}
\end{frame}

\begin{frame}{Divergence}
    \begin{equation}
        \nabv \cdot \vec{V} = \frac{\partial \vec{V_x}}{\partial x} + \frac{\partial \vec{V_y}}{\partial y} + \frac{\partial \vec{V_z}}{\partial z}
    \end{equation}
    \makefigure[1.0][0.6][]{level-1/divergence}
\end{frame}

\begin{frame}{Gauss's Law}
    \begin{twocolumns}[0.5]
    \leftcol
        \centering{\underline{Gauss's Law}}
        \vspace{-8pt}
        \begin{equation}
            \nabv \cdot \efield = \frac{\rho}{\epsilon_0}
            \hspace{1em} \Longleftrightarrow \hspace{1.0em}
            \nabv \cdot \eflux = \rho v
        \end{equation}
    \rightcol
        \makefigure[1.0][0.5][]{level-1/gauss-law-1}
    \end{twocolumns}
\end{frame}
 
\begin{frame}{E Field on a PCB}
    \vspace{-20pt}
    \makefigure[1.0][1.0][]{level-1/pcb-E-fields-before}
\end{frame}

\begin{frame}{E Field on a PCB}
    \vspace{-20pt}
    \makefigure[1.0][1.0][]{level-1/pcb-E-fields}
\end{frame}

\begin{frame}{Gauss's Law for Magnetism}
    \begin{twocolumns}[0.35]
        \leftcol
            \centering{\underline{Gauss's Law For Magnetism}}
            \vspace{-8pt}
            \begin{equation}
                \nabv \cdot \mflux = 0
            \end{equation}\\
            \vspace{10pt}
            \makefigure[1.0][0.4][]{level-1/magnetic-meme}
        \rightcol
            \makefigure[1.0][0.5][]{level-1/magnetic-meme-2}
    \end{twocolumns}
\end{frame}


\begin{frame}{Curl}
    \begin{equation}
        \nabv \times \vec{V} = \left(\frac{\partial\vec{V_z}}{\partial y}-\frac{\partial\vec{V_y}}{\partial z} \right) \check{x} +
        \left(\frac{\partial\vec{V_x}}{\partial z}-\frac{\partial\vec{V_z}}{\partial x} \right) \check{y} +
        \left(\frac{\partial\vec{V_y}}{\partial x}-\frac{\partial\vec{V_x}}{\partial y} \right) \check{z}
    \end{equation}
    \vspace{-15pt}
\begin{twocolumns}[0.5]
    \leftcol
        \makefigure[1.0][0.5][Cyclone Analogy]{level-1/cyclone}
    \rightcol
        \makefigure[1.0][0.5][Rotating Ball Analogy]{level-1/curl-ball}
 \end{twocolumns}
\end{frame}

\begin{frame}{Curl}
    \begin{makelist}[\small][1.5]
        \icon[red]{\faExclamationTriangle} & \textbf{ATTENTION:} La Rotationnelle n'a pas de définition en 2D! Il faut au moin 3 dimensions!
    \end{makelist}
\end{frame}

\begin{frame}{Faraday's Law of Induction}
    \begin{twocolumns}[0.35]
    \leftcol
        \centering{\underline{Faraday's Law of Induction}}
        \vspace{-10pt}
        \begin{equation}
            \nabv \times \efield = -\ddt{\mflux}
        \end{equation}
    \rightcol
        \makefigure[1.0][0.8][]{level-1/generator}
 \end{twocolumns}

\end{frame}

\begin{frame}{Ampère-Maxwell's Law}
    \centering{\underline{Ampère-Maxwell's Law}}
    \vspace{-10pt}
    \begin{equation}
        \nabv \times \mfield = \ddt{\eflux} + \ecurrent
        \hspace{1em} \Longleftrightarrow \hspace{1.0em}
        \nabv \times \mflux = \mu_0 \epsilon_0 \ddt{\efield} + \mu_0 \ecurrent
    \end{equation}
\end{frame}

\begin{frame}{E Field on a PCB}
    \vspace{-20pt}
    \makefigure[1.0][1.0][]{level-1/pcb-H-fields-before}
\end{frame}

\begin{frame}{E Field on a PCB}
    \vspace{-20pt}
    \makefigure[1.0][1.0][]{level-1/pcb-H-fields}
\end{frame}

\begin{frame}{Maxwell Equation}
    \begin{twocolumns}[0.3]
    \leftcol
        \makefigure[1.0][0.8][James Clark Maxwell]{level-1/james-maxwell}
    \rightcol
        \centering{\underline{Gauss's Law}}
        \vspace{-8pt}
        \begin{equation}
                \nabv \cdot \efield = \frac{\rho}{\epsilon_0}
                \hspace{1em} \Longleftrightarrow \hspace{1.0em}
                \nabv \cdot \eflux = \rho v
        \end{equation}
        \centering{\underline{Gauss's Law For Magnetism}}
        \vspace{-8pt}
        \begin{equation}
            \nabv \cdot \mflux = 0
        \end{equation}
        \centering{\underline{Faraday's Law of Induction}}
        \vspace{-10pt}
        \begin{equation}
            \nabv \times \efield = -\ddt{\mflux}
        \end{equation}
        \centering{\underline{Ampère-Maxwell's Law}}
        \vspace{-10pt}
        \begin{equation}
            \nabv \times \mfield = \ddt{\eflux} + \ecurrent
            \hspace{1em} \Longleftrightarrow \hspace{1.0em}
            \nabv \times \mflux = \mu_0 \epsilon_0 \ddt{\efield} + \mu_0 \ecurrent
        \end{equation}
 \end{twocolumns}
\end{frame}

\begin{frame}{EM Field on a PCB}
    \vspace{-20pt}
    \makefigure[1.0][1.0][]{level-1/pcb-fields}
\end{frame}

\subsection[1min - Max]{Superposition \rom{1}}
\begin{frame}{Plan}
    \begin{makelist}[\small][1.5]
        \icon[red]{\faTimes} & Équation Linéaire\\
        \icon[red]{\faTimes} & Addition de Signaux
    \end{makelist}
\end{frame}

\begin{frame}{Green Theorem}
    \begin{twocolumns}[0.5]
        \leftcol
            \makefigure[1.0][0.7][]{level-1/green-theorem-meme}
        \rightcol
            \makefigure[1.0][0.7][Inductor Magnetic field]{level-1/inductor-magnetic-field}
    \end{twocolumns}
    \vspace{-15pt}
    Quiz: Dans quel sens vont les electrons dans la bobine?
\end{frame}

\subsection[2min - Max]{Voltage}
\begin{frame}{Plan}
    Analogie avec energie potentiel gravitationnel 
\end{frame}

\subsection[2min - Max]{Current}
\begin{frame}{Plan}
    Inclure conductivité $\sigma$ 
\end{frame}

\begin{frame}{Ohm's Law in Vector form}
    \begin{equation}
        \ecurrent = \sigma \efield
    \end{equation}
\end{frame}


\subsection[2min - Max]{Matériaux \rom{1}}
\begin{frame}{Plan}
    \begin{makelist}[\small][1.5]
        \icon[red]{\faTimes} & Retour sur $\mu$ \& $\epsilon$\\
        \icon[red]{\faTimes} & Difference entre $\epsilon$, $\epsilon_0$ et $\epsilon_r$\\
        \icon[red]{\faTimes} & Indices de refraction Reel
    \end{makelist}
\end{frame}

\begin{frame}{Permittivitée vs. Permeabilitée}
    \begin{twocolumns}[0.5]
        \leftcol
            Permittivity
        \rightcol
            Permeability
    \end{twocolumns}
\end{frame}

\begin{frame}{$\epsilon$ \& $\mu$}
    \begin{twocolumns}[0.5]
        \leftcol
            \begin{equation}
                \epsilon = \text{Electric Permittivity [F/m]}
            \end{equation}
            \begin{equation}
                \mu = \text{Magnetic Permeabililty [H/m]}
            \end{equation}
        \rightcol
            \makefigure[1.0][0.7][Imperméable]{level-1/impermeable}
    \end{twocolumns}
\end{frame}

\begin{frame}{$\epsilon_0$ \& $\mu_0$}
    \begin{twocolumns}[0.5]
        \leftcol
            Electric Free Space Permittivity [F/m]:
            \begin{equation}
                \epsilon_0 \approx 8.854\times 10^{-12}
            \end{equation}
            Magnetic Free Space Permeabililty [H/m]:
            \begin{equation}
                \mu_0 \approx 4\pi \times 10^{-7}
            \end{equation}
        \rightcol
            \makefigure[1.0][0.7][Imperméable]{level-1/impermeable}
    \end{twocolumns}
\end{frame}

\begin{frame}{$\epsilon_r$ \& $\mu_r$}
    \begin{twocolumns}[0.5]
        \leftcol
            Electric Relative Permittivity [F/m]:
            \begin{equation}
                \epsilon_r =\frac{\epsilon}{\epsilon_0}
            \end{equation}
            Magnetic Relative Permeabililty [H/m]:
            \begin{equation}
                \mu_r =\frac{\mu}{\mu_0}
            \end{equation}
        \rightcol
            \makefigure[1.0][0.7][Imperméable]{level-1/impermeable}
    \end{twocolumns}
\end{frame}

\begin{frame}{$\epsilon_0$ \& $\mu_0$}
    \begin{twocolumns}[0.5]
        \leftcol
            Electric Relative Permittivity [F/m]:
            \begin{equation}
                \epsilon_0 =\frac{\epsilon}{\epsilon_0}
            \end{equation}
            Magnetic Relative Permeabililty [H/m]:
            \begin{equation}
                \mu_r =\frac{\mu}{\mu_0}
            \end{equation}
        \rightcol
            \makefigure[1.0][0.7][Imperméable]{level-1/impermeable}
    \end{twocolumns}
\end{frame}

\begin{frame}{Indice de Refraction}
    \begin{twocolumns}[0.5]
        \leftcol
            Indice de refraction pour materiaux \textbf{sans pertes}:
            \begin{equation}
                n =\frac{\sqrt{\epsilon \mu}}{\sqrt{\epsilon_0 \mu_0}} = \sqrt{\epsilon_r \mu_r}
            \end{equation}
        \rightcol
            \makefigure[1.0][0.7][Radio Telescope]{level-1/rf-telescope}
    \end{twocolumns}
\end{frame}


\begin{comment}
\subsection[4min - Max]{Charge Movement}
\begin{frame}{Plan}
    \begin{makelist}[\small][1.5]
        \icon[red]{\faTimes} & Comment les Electrons bougent\\
        \icon[red]{\faTimes} & Propriété materiaux
    \end{makelist}
\end{frame}

\begin{frame}{EM Properties of metals}
    \maketable{conductivity}
\end{frame}


\subsection[3min - Max]{Passive Components \rom{1}}
\begin{frame}{Plan}
    \begin{makelist}[\small][1.5]
        \icon[red]{\faTimes} & Resistance\\
        \icon[red]{\faTimes} & Condensateur \\
        \icon[red]{\faTimes} & Inducteur
    \end{makelist}
\end{frame}
\end{comment}