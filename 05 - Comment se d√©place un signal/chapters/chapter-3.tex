%!TEX root = ../presentation.tex

\section[Level 3]{Impedance \& Reflection [20min - 1h10]}
\subsection[5min-Pascal]{Signal Source \rom{2}}
\begin{frame}{Plan}
    \begin{makelist}[\small][1.5]
        \icon[red]{\faTimes} & Type of source\\
        \icon[red]{\faTimes} & High/Low Impedance\\
        \icon[red]{\faTimes} & GPIO output circuit
    \end{makelist}
\end{frame}

\subsection[5min-Max]{Materiaux \rom{2}}
\begin{comment}
\begin{frame}{Plan}
    \begin{makelist}[\small][1.5]
        \icon[red]{\faTimes} & Complex $\epsilon$ \& $\mu$ \\
        \icon[red]{\faTimes} & Complex Refractive index\\
        \icon[red]{\faTimes} & Losses in dB\\
    \end{makelist}
\end{frame}
\end{comment}
\begin{frame}{Rappel}
    \begin{twocolumns}[0.4]
        \leftcol
            Electric Permittivity [F/m]:
            \begin{equation*}
                \varepsilon =\varepsilon_r \varepsilon_0
            \end{equation*}
            Magnetic Permeabililty [H/m]:
            \begin{equation*}
                \mu =\mu_r \mu_0
            \end{equation*}
            Refractive Index:
            \begin{equation*}
                n =\frac{\sqrt{\varepsilon \mu}}{\sqrt{\varepsilon_0 \mu_0}} = \sqrt{\varepsilon_r \mu_r}
            \end{equation*}
        \rightcol
            Nous avons initialement considéré ces valeurs comme étant des nombre réel $\mathbb{R}$.\\
            \vspace{10pt}
            Cela fonctionne tres bien pour definir des matériaux \textbf{sans pertes}.\\
            \vspace{10pt}
            Par contre, dans la vrai vie, ce sont des nombres complexe $\mathbb{C}$. Cela permet de définir des matériaux \textbf{avec pertes}
    \end{twocolumns}
\end{frame}

\begin{frame}{Matériaux Complexes}
        \begin{twocolumns}[0.6]
        \leftcol
            Indice de Refraction Complexe:
            \begin{equation*}
                \tilde{n} = n + jk
            \end{equation*}
            Permittivitée Complexe:
            \begin{equation*}
                \varepsilon = \epsilon_1 + j \epsilon_2 = (n+jk)^2
            \end{equation*}
            \vspace{5pt}
            \begin{equation*}
                \epsilon_1 = n^2 - k^2
            \end{equation*}
            \begin{equation*}
                \epsilon_2 = 2nk
            \end{equation*}
        \rightcol
           Indice de Refraction:
            \begin{equation*}
                n = \sqrt{\frac{\sqrt{\epsilon_1^2 + \epsilon_2^2}+\epsilon_1}{2}}
            \end{equation*}
           Coefficient d'Extinction:
            \begin{equation*}
                k = \sqrt{\frac{\sqrt{\epsilon_1^2 + \epsilon_2^2}-\epsilon_1}{2}}
            \end{equation*}
        \end{twocolumns}
\end{frame}

\begin{frame}{Coefficient d'Extinction}
    mettre graphique sinusoidale ammortie
    \begin{equation}
        T = e^{-kl}
    \end{equation}
    a valider
\end{frame}

\subsection[5min-Pascal]{Impédances \rom{1}}
\begin{frame}{Plan}
    \begin{makelist}[\small][1.5]
        \icon[red]{\faTimes} & PPPPP2\\
        \icon[red]{\faTimes} & Impedance dans le plan complexe\\
        \icon[red]{\faTimes} & Rappel qu'on ignore la conductance G.\\ %On va la resortir plus tard dans la presentation
        \icon[red]{\faTimes} & Propagation Constant (p.56)
    \end{makelist}
\end{frame}

\begin{frame}{Quote}
    \begin{makelist}[\small][1.5]
        \icon[red]{\faTimes} & L'Électricité prend le chemin avec la plus petite \textbf{Resistance}\\
        \icon[green]{\faCheck} & L'Électricité prend le chemin avec la plus petite \textbf{Impedance}\\
    \end{makelist}
\end{frame}

\subsection[5min-Pascal]{Réflection}
\begin{frame}{Plan}
    \begin{makelist}[\small][1.5]
        \icon[red]{\faTimes} & Bounce Diagram\\
        \icon[red]{\faTimes} & Impedance Mismatch\\
        \icon[red]{\faTimes} & Item 3
    \end{makelist}
\end{frame}
%Bounce Diagrams

\subsection[5min-Pascal]{Transmission Line \rom{1}}
\begin{frame}{Plan}
    \begin{makelist}[\small][1.5]
        \icon[red]{\faTimes} & Equation de base\\
        \icon[red]{\faTimes} & Pertes en dB (exponential decay)\\
    \end{makelist}
\end{frame}
