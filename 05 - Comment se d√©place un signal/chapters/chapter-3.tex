%!TEX root = ../presentation.tex

\section[Level 3]{Impedance \& Reflection [20min - 1h10]}
\subsection[5min-Pascal]{Signal Source \rom{2}}
\begin{frame}{Plan}
    \begin{makelist}[\small][1.5]
        \icon[red]{\faTimes} & Type of source\\
        \icon[red]{\faTimes} & High/Low Impedance\\
        \icon[red]{\faTimes} & GPIO output circuit
    \end{makelist}
\end{frame}

\subsection[5min-Pascal]{Impédances \rom{1}}
\begin{frame}{Plan}
    \begin{makelist}[\small][1.5]
        \icon[red]{\faTimes} & PPPPP2\\
        \icon[red]{\faTimes} & Impedance dans le plan complexe\\
        \icon[red]{\faTimes} & Rappel qu'on ignore la conductance G. %On va la resortir plus tard dans la presentation
    \end{makelist}
\end{frame}


\subsection[5min-Pascal]{Réflection}
\begin{frame}{Plan}
    \begin{makelist}[\small][1.5]
        \icon[red]{\faTimes} & Bounce Diagram\\
        \icon[red]{\faTimes} & Impedance Mismatch\\
        \icon[red]{\faTimes} & Item 3
    \end{makelist}
\end{frame}
%Bounce Diagrams

\subsection[5min-Pascal]{Transmission Line \rom{1}}
\begin{frame}{Plan}
    \begin{makelist}[\small][1.5]
        \icon[red]{\faTimes} & Equation de base\\
        \icon[red]{\faTimes} & Pertes en dB (exponential decay)\\
    \end{makelist}
\end{frame}