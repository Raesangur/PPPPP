%!TEX root = ../presentation.tex 

\section[Bonus Level 12]{Non-linearity Valley [14min-3h49]}
\subsection[5min-Max]{Matériaux \rom{6}}
\begin{frame}{Plan}
    \begin{makelist}[\small][1.5]
        \icon[red]{\faTimes} & Higher-Order Effects\\
    \end{makelist}
\end{frame}

\subsection[5min-Pascal]{Passive Component \rom{3}}
\begin{frame}{Plan}
    \begin{makelist}[\small][1.5]
        \icon[red]{\faTimes} & Nonlinear passive component models\\
        \icon[red]{\faTimes} & 
    \end{makelist}
\end{frame}

\subsection[3min-Max]{Superposition \rom{3}}
\begin{frame}{Plan}
    \begin{makelist}[\small][1.5]
        \icon[red]{\faTimes} & Superposition breaks\\
        \icon[red]{\faTimes} & Item 2\\
        \icon[red]{\faTimes} & Item 3
    \end{makelist}
\end{frame}


\subsection[5min-Max]{Harmonics \rom{3}}
\begin{frame}{Plan}
    \begin{makelist}[\small][1.5]
        \icon[red]{\faTimes} & How non-linearity create harmonics\\
        \icon[red]{\faTimes} & Item 2\\
        \icon[red]{\faTimes} & Item 3
    \end{makelist}
\end{frame}

\begin{frame}{Effect Of Non-Linearity}
   \makefigure[1.0][1.0][]{level-12/non-linear}
\end{frame}


\subsection[3min-Max]{Intermodulation}
\begin{frame}{Plan}
    \begin{makelist}[\small][1.5]
        \icon[red]{\faTimes} & Item 1\\
        \icon[red]{\faTimes} & Item 2\\
        \icon[red]{\faTimes} & Item 3
    \end{makelist}
\end{frame}

\subsection[3min-Max]{Crossmodulation}
\begin{frame}{Plan}
    \begin{makelist}[\small][1.5]
        \icon[red]{\faTimes} & Item 1\\
        \icon[red]{\faTimes} & Item 2\\
        \icon[red]{\faTimes} & Item 3
    \end{makelist}
\end{frame}
% À réévaluer
