%!TEX root = ../presentation.tex 

\section[Bonus Level 9]{Advanced Building Blocks [17min-3h06]}
\subsection[2min-Max]{Signal Source \rom{5}}
\begin{frame}{Plan}
    \begin{makelist}[\small][1.5]
        \icon[red]{\faTimes} & PLL\\
        \icon[red]{\faTimes} & n-Synth\\
        \icon[red]{\faTimes} & Item 3
    \end{makelist}
\end{frame}

\subsection[2min-Max]{Materiaux \rom{5}}
\begin{frame}{Plan}
    \begin{makelist}[\small][1.5]
        \icon[red]{\faTimes} & Electric \& Magnetic Susceptibility\\
        \icon[red]{\faTimes} & Tensor\\
        \icon[red]{\faTimes} & Isotropism vs. Anisotropism\\
    \end{makelist}
\end{frame}

\begin{frame}{Susceptibility}
    La susceptibilitée est une autre valeur pour définir les propriétés de matériaux.
    \begin{equation}
        \begin{aligned}
            \suse &=  \varepsilon_r - 1 \hspace{2em} \text{Electric Susceptibility}\\
            \susm &=  \mu_r - 1 \hspace{2em} \text{Magnetic Susceptibility}
        \end{aligned}
    \end{equation}
    \vspace{20pt}
    Vacuum Susceptibility is always 0
\end{frame}

\subsection[10min-Pascal]{Stackup \rom{2}}
\pascalbackground
\begin{frame}{Plan}
    \begin{makelist}[\small][1.5]
        \icon[red]{\faTimes} & Rogers\\
        \icon[red]{\faTimes} & Substrate weave pattern\\
        \icon[red]{\faTimes} & Avantages / désavantages de certains matériaux
    \end{makelist}
\end{frame}

%https://www.nwengineeringllc.com/article/what-is-the-fiber-weave-effect-in-a-pcb-substrate.php
\begin{frame}{FR4 Fabrication \& Defect}
    \textbf{Je t'ai mis des images cool que j'ai trouvé pour te starter sur les weaves patterns}
\end{frame}

\begin{frame}{FR4 Fabrication \& Defect}
    \makefigure[1.0][1.0][]{level-9/fr4-stackup}
\end{frame}

\begin{frame}{FR4 Fabrication \& Defect}
    \makefigure[1.0][1.0][]{level-9/fr4-cross}
\end{frame}

\begin{frame}{FR4 Fabrication \& Defect}
    \makefigure[0.80][0.80][]{level-9/fr4-names}
\end{frame}

\begin{frame}{FR4 Fabrication \& Defect}
    \makefigure[1.0][1.0][]{level-9/fr4-defects}
\end{frame}

\begin{frame}{FR4 Weave patterns}
    \makefigure[0.8][0.8][]{level-9/fr4-weave-type}
\end{frame}

\begin{frame}{FR4 Weave patterns}
    \makefigure[0.80][0.80][]{level-9/fr4-weave}
\end{frame}

\begin{frame}{FR4 Fiber Weave effect}
    \makefigure[0.80][0.80][]{level-9/fr4-weave-effect}
\end{frame}


\subsection[2min-Max]{Stubs}
\maxbackground
\begin{frame}{Plan}
    \begin{makelist}[\small][1.5]
        \icon[red]{\faTimes} & Item 1\\
        \icon[red]{\faTimes} & Item 2\\
        \icon[red]{\faTimes} & Item 3
    \end{makelist}
\end{frame}

\subsection[2min-Max]{Resonator}
\maxbackground
\begin{frame}{Plan}
    \begin{makelist}[\small][1.5]
        \icon[red]{\faTimes} & Item 1\\
        \icon[red]{\faTimes} & Item 2\\
        \icon[red]{\faTimes} & Item 3
    \end{makelist}
\end{frame}

\subsection[4min-Max]{Antennes}
\maxbackground
\begin{frame}{Plan}
    \begin{makelist}[\small][1.5]
        \icon[red]{\faTimes} & Item 1\\
        \icon[red]{\faTimes} & Item 2\\
        \icon[red]{\faTimes} & Item 3
    \end{makelist}
\end{frame}
