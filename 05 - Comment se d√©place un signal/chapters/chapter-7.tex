%!TEX root = ../presentation.tex 

\section[Level 7]{Field lines and Fringes [20min-2h27]}

\subsection[5min-Pascal]{Ground Planes \rom{2}}
\begin{frame}{Plan}
    \begin{makelist}[\small][1.5]
        \icon[red]{\faTimes} & Ground Bounce\\
        \icon[red]{\faTimes} & Vias pour changer de layers\\
        \icon[red]{\faTimes} & item 3
    \end{makelist}
\end{frame}

\subsection[5min-Pascal]{Waveguide \rom{2}}
\begin{frame}{Plan}
    \begin{makelist}[\small][1.5]
        \icon[red]{\faTimes} & stripline Field\\
        \icon[red]{\faTimes} & Microstrip Field\\
        \icon[red]{\faTimes} & CPWG, GCPWG field\\
        \icon[red]{\faTimes} & Vias
    \end{makelist}
\end{frame}

\begin{frame}{When it's a waveguide}    
    \maketable{model-threshold}
\end{frame}

%Transmission line 3?

\subsection[10min-Max-Pascal]{Matériaux \rom{3}}
\begin{frame}{Plan}
    \begin{makelist}[\small][1.5]
        \icon[red]{\faTimes} & Susceptibility\\
        \icon[red]{\faTimes} & Polarization and Magnetisation\\
        \icon[red]{\faTimes} & Effective Complex Permittivity\\
        \icon[red]{\faTimes} & Frequency dependance\\
    \end{makelist}
\end{frame}

\begin{frame}{Rappel}
    \begin{twocolumns}[0.4]
        \leftcol
            Electric Permittivity [F/m]:
            \begin{equation*}
                \varepsilon =\varepsilon_r \varepsilon_0
            \end{equation*}
            Refractive Index:
            \begin{equation*}
                n =\frac{\sqrt{\varepsilon \mu}}{\sqrt{\varepsilon_0 \mu_0}} = \sqrt{\varepsilon_r \mu_r}
            \end{equation*}
            Permittivitée Complexe:
            \begin{equation*}
                \varepsilon = \epsilon_1 + j \epsilon_2 = (n+jk)^2
            \end{equation*}
        \rightcol
            Dans les sections précédentes, nous avons vu pourquoi les propriété des matériaux sont définies avec \textbf{nombres complexe} $\mathbb{C}$.
    \end{twocolumns}
\end{frame}

\begin{frame}{Rappel}
    \centering
    \begin{twocolumns}[0.4]
        \leftcol
            \centering
            \underline{Intensitée du Champ Électrique}
            \begin{equation*}
                \efield \rightarrow [\SI{}{\volt\per\meter}]
            \end{equation*}
            \underline{Densitée de Flux Électrique}
            \begin{equation*}
                \eflux \rightarrow [\SI{}{\coulomb\per\meter\squared}]
            \end{equation*}
            \underline{Relation}
            \begin{equation*}
                \eflux = \varepsilon \efield
            \end{equation*}
        \rightcol
            \centering
            \underline{Intensitée de Champ Magnétique}
            \begin{equation*}
                \mfield \rightarrow [\SI{}{\ampere\per\meter}]
            \end{equation*}
            \underline{Densitée de Flux Magnétique}
            \begin{equation*}
                \mflux \rightarrow [\SI{}{\weber\per\meter\squared} \;\; ou \;\; \SI{}{\tesla}]
            \end{equation*}
            \underline{Relation}
            \begin{equation*}
                \mflux = \mu \mfield
            \end{equation*}
    \end{twocolumns}
\end{frame}


\begin{frame}{Susceptibility}
    La susceptibilitée est une autre valeur pour définir les propriétés de matériaux.
    \begin{equation}
        \begin{aligned}
            \suse &=  \varepsilon_r - 1 \hspace{2em} \text{Electric Susceptibility}\\
            \susm &=  \mu_r - 1 \hspace{2em} \text{Magnetic Susceptibility}
        \end{aligned}
    \end{equation}
    \vspace{20pt}
    Vacuum Susceptibility is always 0
\end{frame}

\begin{frame}{Polarization Density}
    \begin{twocolumns}[0.6]
        \leftcol
        La densitée de polarisation représente la \textbf{contribution des dipôles électriques} du matériau qui \textbf{s'alignent} en réponse à un champs $\efield$
        \begin{equation*}
            \eflux = \varepsilon \efield \hspace{1em} \rightarrow \hspace{1em} \eflux = \varepsilon_0 \efield + \pola
        \end{equation*}
        \vspace{10pt}
        \begin{equation*}
            \begin{aligned}
                \pola &=  \left(\varepsilon - \varepsilon_0 \right) \efield\\
                &= \varepsilon_0 \suse \efield
            \end{aligned}
        \end{equation*}
        \rightcol
    \makefigure[1.0][0.6][]{level-7/polarization}
    \end{twocolumns}
    \centering
    $\pola$ représente également l'énergie électrique que le matériau accumule.
\end{frame}

\begin{frame}{Effective Complex Permittivity}
    \begin{twocolumns}[0.5]
     \leftcol
         Conversion to phasor form:
        \begin{equation*}
            \begin{aligned}
                \nabv \times \mfield &= \ddt{\eflux} + \ecurrent \\
                &= j\omega \eflux + \sigma \efield\\
                &= j\omega \varepsilon \efield + \sigma \efield\\
            \end{aligned}
        \end{equation*}
            Developpement:
        \begin{equation*}
            \begin{aligned}
                j\omega \varepsilon \efield + \sigma \efield &= j\omega \left(\epsilon_1 - j \epsilon_2\right)\efield + \sigma \efield \\
                &= j\omega \epsilon_1 \efield +  \left( \sigma + \omega \epsilon_2 \right)\efield \\
                &= j\omega \left(\epsilon_1 - j \epsilon_2 - j \frac{\sigma}{\omega}\right)\efield\\
            \end{aligned}
        \end{equation*}
    \rightcol
        \begin{equation*}
            \begin{aligned}
                \omega \varepsilon_1 &: \text{Ability to store electric energy}\\
                \omega \varepsilon_2 &: \text{Dielectric damping loss}\\
                \sigma &: \text{Conductivity loss}
            \end{aligned}
        \end{equation*}
    \end{twocolumns}
\end{frame}

\begin{frame}{Effective Complex Permittivity}
    \centering
    \underline{Effective Complex Relative Permittivity}
    \begin{equation}
        \varepsilon_e = \left(\epsilon_1 - j\epsilon_2 - j\frac{\sigma}{\omega}\right)\frac{1}{\varepsilon_0}
    \end{equation}
    This term describe the properties of \textbf{non-homogeneous} materials. It can be seen as an average of all the individual relative permittivities.

\end{frame}

\subsection[5min-Pascal]{EM Fields \rom{2}}
\begin{frame}{Plan}
    \begin{makelist}[\small][1.5]
        \icon[red]{\faTimes} & Skin effect\\
        \icon[red]{\faTimes} & EMI\\
        \icon[red]{\faTimes} & Item 3
    \end{makelist}
\end{frame}

\begin{frame}{Skin Depth of Different Materials}
    %\begin{center}
    %    \textbf{1oz PCB = $\SI{35}{\micro\meter}$}
    %\end{center}

    \maketable{skin-depth}
\end{frame}


