%!TEX root = ../presentation.tex

\section[Level 4]{Noise [27min - 1h37]}
\subsection[5min-Max]{Decibel Review}
\begin{frame}{Plan}
    \begin{makelist}[\small][1.5]
        \icon[red]{\faTimes} & Pourquoi c'est important\\
        \icon[red]{\faTimes} & Analogie des dB avec le stock market\\
        \icon[red]{\faTimes} & Item 3
    \end{makelist}
\end{frame}

\begin{frame}{Signal Power}
    \centering{\underline{Signal Power in dB}}
    \begin{equation}
            P_{Signal|\unit{dB}}=10 \log{\frac{P_{\text{Signal}|\unit{W}}}{1\unit{W}}}
    \end{equation}
    \vspace{10pt}\\
    \centering{\underline{Signal Power in dBm}}
    \begin{equation}
            P_{Signal|\unit{dBm}}=10 \log{\frac{P_{\text{Signal}|\unit{mW}}}{1\unit{mW}}}
    \end{equation}
\end{frame}

\begin{frame}{System Gain}
    \centering{\underline{Voltage Gain}}
    \begin{equation}
            A_{V|\unit{dB}}=20 \log{\frac{V_{out}}{V_{in}}}
            \hspace{2.0em} \Longleftrightarrow \hspace{2.0em}
            \frac{P_{out}}{P_{in}} = 10^{\frac{A_{P}}{20}} 
    \end{equation}
    \vspace{10pt}\\
    \centering{\underline{Power Gain}}
    \begin{equation}
            A_{V|\unit{dB}}=10 \log{\frac{P_{out}}{P_{in}}}
            \hspace{2.0em} \Longleftrightarrow \hspace{2.0em}
            \frac{P_{out}}{P_{in}} = 10^{\frac{A_{P}}{10}} 
    \end{equation}
\end{frame}


\subsection[4min-Max]{Signal Source \rom{3}}
\begin{frame}{Plan}
    \begin{makelist}[\small][1.5]
        \icon[red]{\faTimes} & Random Noise Source\\
        \icon[red]{\faTimes} & Noise Power\\
        \icon[red]{\faTimes} & Source of noise in a circuit
    \end{makelist}
\end{frame}

\begin{frame}{Noise Source}
    \begin{makelist}[\small][1.5]
        \icon[red]{\faTimes} & Thermal Noise\\
        \icon[red]{\faTimes} & Environmental Noise \\
        \icon[red]{\faTimes} & Flicker Noise
    \end{makelist}
\end{frame}


\subsection[2min-Max]{Noise Spectrum}
\begin{frame}{Plan}
    \begin{makelist}[\small][1.5]
        \icon[red]{\faTimes} & Frequency dependant noise power\\
        \icon[red]{\faTimes} & Demo avec type de bruit (red, white, brown, etc..)
    \end{makelist}
\end{frame}

\begin{frame}{Average Power of a Signal}
    \begin{equation}
        P_{n}=\lim_{T \rightarrow \infty}\frac{1}{T}\int_{0}^{T}n^{2}(t) \,dt
   \end{equation}
    \vspace{10pt}\\
    \begin{equation}
        \int_{0}^{\infty}S_{x}(f) \, df = \lim_{T \rightarrow \infty}\frac{1}{T}\int_{0}^{T}x^{2}(t) \, dt
   \end{equation}
\end{frame}

\subsection[3min-Max]{Harmonics \rom{2}}
\begin{frame}{Plan}
    \begin{makelist}[\small][1.5]
        \icon[red]{\faTimes} & Gauss representation in frequency domain of a sine wave\\
        \icon[red]{\faTimes} & Sinc function\\
        \icon[red]{\faTimes} & Item 3
    \end{makelist}
\end{frame}
%
%Sinc function

\subsection[5min-Max]{Signal to Noise Ratio (SNR)}
\begin{frame}{Plan}
    \begin{makelist}[\small][1.5]
        \icon[red]{\faTimes} & Why it matters\\
        \icon[red]{\faTimes} & How can you tell the SNR you need\\
        \icon[red]{\faTimes} & Shannon-Hartley Theorem\\
        \icon[red]{\faTimes} & Application: DAC,ADC\\
        \icon[red]{\faTimes} & Application: Example for Voyager 1 Detection \href{https://www.seti.org/detecting-voyager-1-ata}{Link}
    \end{makelist}
\end{frame}

\begin{frame}{Definition}
    \begin{equation}
        SNR = \frac{P_{\text{Signal}|\unit{W}}}{P_{\text{Noise}|\unit{W}}}=P_{\text{Signal}|\unit{dB}} - P_{\text{Noise}|\unit{dB}}
    \end{equation}
\end{frame}

\begin{frame}{Shannon-Hartley}
    \begin{equation}
        C = BW \cdot \log_{2}\left( 1+\frac{P_{\text{Signal}|\unit{W}}}{P_{\text{Noise}|\unit{W}}} \right)
    \end{equation}
    \begin{makelist}[\small][1.5]
        \icon[blue]{\faCloudversify} & C: Canal Capacity [Bit/s]\\
        \icon[blue]{\faChartArea} & BW: Canal Bandwidth [Hz]
    \end{makelist}
\end{frame}
% 
% 
\begin{frame}{Determining SNR}
    Mettre tableau avec resolution et SNR
\end{frame}

\subsection[5min-Pascal]{Jitter}
\begin{frame}{Plan}
    \begin{makelist}[\small][1.5]
        \icon[red]{\faTimes} & Item 1\\
        \icon[red]{\faTimes} & Item 2\\
        \icon[red]{\faTimes} & Item 3
    \end{makelist}
\end{frame}


\subsection[5min-Pascal]{Eye diagram}
\begin{frame}{Plan}
    \begin{makelist}[\small][1.5]
        \icon[red]{\faTimes} & Item 1\\
        \icon[red]{\faTimes} & Item 2\\
        \icon[red]{\faTimes} & Item 3
    \end{makelist}
\end{frame}
