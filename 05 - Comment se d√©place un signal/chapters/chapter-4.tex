%!TEX root = ../presentation.tex

\section[Level 4]{Noise [27min - 1h37]}

\introbackground
\begin{frame}[plain, label=intro-level-4]
    \centering
    \Large
    \textcolor{white}{\textbf{Sujets Abordés dans la Section:}}\\
    \vspace{24pt}
    \begin{tabular}{c l}
        \textcolor{UDSgreenFierte}{\faEye}
            & \textcolor{white}{Introduction sur les \textbf{Decibels} \& \textbf{GPIO}}\\
            [0.3em]
        \textcolor{UDSgreenFierte}{\faEye}
            & \textcolor{white}{Explication de différent type de \textbf{Bruit Parasites} dans les circuits}\\
            [0.3em]
        \textcolor{UDSgreenFierte}{\faEye}
            & \textcolor{white}{Quantifier l'impact des \textbf{Effets Parasites}}\\
            [0.3em]
        \textcolor{UDSgreenFierte}{\faHubspot}
            & \textcolor{white}{Comment lire un \textbf{Eye Diagram}}\\
            [0.3em]
    \end{tabular}
\end{frame}
\defaultbackground

\subsection[5min-Max]{Decibel Review}
\maxbackground
\begin{frame}{Plan}
    \begin{makelist}[\small][1.5]
        \icon[red]{\faTimes} & Pourquoi c'est important\\
        \icon[red]{\faTimes} & Analogie des dB avec le stock market\\
        \icon[red]{\faTimes} & Item 3
    \end{makelist}
\end{frame}

\begin{frame}{Signal Power}
    \centering{\underline{Signal Power in dB}}
    \begin{equation}
            P_{Signal|\unit{dB}}=10 \log{\frac{P_{\text{Signal}|\unit{W}}}{1\unit{W}}}
    \end{equation}
    \vspace{10pt}\\
    \centering{\underline{Signal Power in dBm}}
    \begin{equation}
            P_{Signal|\unit{dBm}}=10 \log{\frac{P_{\text{Signal}|\unit{mW}}}{1\unit{mW}}}
    \end{equation}
\end{frame}

\begin{frame}{System Gain}
    \centering{\underline{Voltage Gain}}
    \begin{equation}
            A_{V|\unit{dB}}=20 \log{\frac{V_{out}}{V_{in}}}
            \hspace{2.0em} \Longleftrightarrow \hspace{2.0em}
            \frac{V_{out}}{V_{in}} = 10^{\frac{A_{v}}{20}} 
    \end{equation}
    \vspace{10pt}\\
    \centering{\underline{Power Gain}}
    \begin{equation}
            A_{V|\unit{dB}}=10 \log{\frac{P_{out}}{P_{in}}}
            \hspace{2.0em} \Longleftrightarrow \hspace{2.0em}
            \frac{P_{out}}{P_{in}} = 10^{\frac{A_{P}}{10}} 
    \end{equation}
\end{frame}


\subsection[4min-Max]{Signal Source \rom{3}}
\maxbackground
\begin{frame}{Plan}
    \begin{makelist}[\small][1.5]
        \icon[red]{\faTimes} & Random Noise Source\\
        \icon[red]{\faTimes} & Noise Power\\
        \icon[red]{\faTimes} & Source of noise in a circuit
    \end{makelist}
\end{frame}

\begin{frame}{Noise Source}
    \begin{makelist}[\small][1.5]
        \icon[red]{\faTimes} & Thermal Noise\\
        \icon[red]{\faTimes} & Environmental Noise \\
        \icon[red]{\faTimes} & Flicker Noise
    \end{makelist}
\end{frame}

\begin{frame}{Bruit Thermique - Resistance}
    \begin{twocolumns}[0.5]
        \leftcol
        Johnson-Nyquist Noise Equation:
        \begin{equation}
            V_{rms}= \sqrt{4k_BTR\Delta f}
        \end{equation}
        $k_b = 1.38\times 10^{-23}$: Boltzmann Constant [J]\\
        $T$: Temperature [K]\\
        $\Delta f$: Frequency Range\\
        $R$: Resistance [Ohm]\\
        \rightcol
        \makefigure[0.8][0.8][]{level-4/johnson-circuit}
    \end{twocolumns}
\end{frame}

\begin{frame}{Bruit Thermique - Thermique}
    %\vspace{20pt}
    \centering
    \underline{Valeurs pour 300K:}
    \begin{equation}
        V_{rms}= 0.13\cdot\sqrt{R\Delta f} \hspace{2.0em} [\SI{}{\nano\volt}]
    \end{equation}
    \makefigure[1.0][1.0]{level-4/johnson-nyquist-band}
\end{frame}

\begin{frame}{Bruit Thermique - Resistance}
    \makefigure[0.8][0.8][]{level-4/RC-noise_simpler}
\end{frame}

\begin{frame}{Bruit Thermique - Autres}
    \begin{twocolumns}[0.5]
        \leftcol
        \centering
        \underline{Bruit Condensateur:}
        \begin{equation}
            V_{rms}= \sqrt{\frac{k_BT}{C}}
        \end{equation}
        \rightcol
        \centering
        \underline{Bruit Inducteur:}
        \begin{equation}
            \overline{I_{n}}= \sqrt{\frac{k_BT}{L}}
        \end{equation}
    \end{twocolumns}
\end{frame}

\begin{frame}{Bruit Thermique - Applications}
    \maketable{thermal-noise}
\end{frame}

%Radio Noise
%https://www.itu.int/rec/R-REC-P.372/en
\begin{frame}{Bruits Environnementaux}
    \begin{makelist}[\small][1.5]
        \icon{\faBolt} & \textbf{Atmospheriques:} Éclairs, Effets Corona, etc... ($\approx -120 \text{ to} -80 \SI{}{\dbm\per\hertz}$, $<20\SI{}{\mega\hertz}$)\\
        \icon{\faHouzz} & \textbf{Industriel:} Lignes HV, Lampes Fluorescentes, etc... ($\approx -110 \text{ to} -80 \SI{}{\dbm\per\hertz}$, $20-300\SI{}{\mega\hertz}$)\\
        \icon{\faSun} & \textbf{Solaire:} Vents solaires\\
        \icon{\faMeteor} & \textbf{Cosmique:} Effets collectif des étoiles (8MHz-1.5GHz), CMB, Quasar\\
        & Pulsar, Chaos Gods ($\approx -204 \text{ to} -164 \SI{}{\dbm\per\hertz}$, $20-120\SI{}{\mega\hertz}$)\\
        \icon{\faFrown} & \textbf{Psychologique:} Toi qui fait pas tes lectures\\
    \end{makelist}
\end{frame}

\begin{frame}{Autres source de bruit sur un PCB}
    \maketable{pcb-noise}
\end{frame}

\subsection[2min-Max]{Noise Spectrum}
\maxbackground
\begin{frame}{Plan}
    \begin{makelist}[\small][1.5]
        \icon[red]{\faTimes} & Frequency dependant noise power\\
        \icon[red]{\faTimes} & Demo avec type de bruit (red, white, brown, etc..)
    \end{makelist}
\end{frame}

\begin{frame}{Average Power of a Signal}
    \begin{equation}
        P_{n}=\lim_{T \rightarrow \infty}\frac{1}{T}\int_{0}^{T}n^{2}(t) \,dt
   \end{equation}
    \vspace{10pt}\\
    \begin{equation}
        \int_{0}^{\infty}S_{x}(f) \, df = \lim_{T \rightarrow \infty}\frac{1}{T}\int_{0}^{T}x^{2}(t) \, dt
   \end{equation}
\end{frame}

\subsection[3min-Max]{Harmonics \rom{2}}
\maxbackground
\begin{frame}{Plan}
    \begin{makelist}[\small][1.5]
        \icon[red]{\faTimes} & Gauss representation in frequency domain of a sine wave\\
        \icon[red]{\faTimes} & Sinc function\\
        \icon[red]{\faTimes} & Item 3
    \end{makelist}
\end{frame}
%
%Sinc function

\subsection[5min-Max]{Signal to Noise Ratio (SNR)}
\maxbackground
\begin{frame}{Plan}
    \begin{makelist}[\small][1.5]
        \icon[red]{\faTimes} & Why it matters\\
        \icon[red]{\faTimes} & How can you tell the SNR you need\\
        \icon[red]{\faTimes} & Shannon-Hartley Theorem\\
        \icon[red]{\faTimes} & Application: DAC,ADC\\
        \icon[red]{\faTimes} & Application: Example for Voyager 1 Detection \href{https://www.seti.org/detecting-voyager-1-ata}{Link}
    \end{makelist}
\end{frame}

\begin{frame}{Definition}
    \begin{equation}
        SNR = \frac{P_{\text{Signal}|\unit{W}}}{P_{\text{Noise}|\unit{W}}}=P_{\text{Signal}|\unit{dB}} - P_{\text{Noise}|\unit{dB}}
    \end{equation}
\end{frame}

\begin{frame}{Shannon-Hartley}
    \begin{equation}
        C = BW \cdot \log_{2}\left( 1+\frac{P_{\text{Signal}|\unit{W}}}{P_{\text{Noise}|\unit{W}}} \right)
    \end{equation}
    \begin{makelist}[\small][1.5]
        \icon{\faCloudversify} & C: Canal Capacity [Bit/s]\\
        \icon{\faChartArea} & BW: Canal Bandwidth [Hz]
    \end{makelist}
\end{frame}
% 
% 
\begin{frame}{Determining SNR}
    Mettre tableau avec resolution et SNR
\end{frame}

\begin{frame}{Common PCB Protocols}
    \maketable{protocol-snr}
\end{frame}

\begin{frame}{(BER) Bit Error Rate}
    \maketable{ber}
\end{frame}

\subsection[5min-Pascal]{Jitter}
\pascalbackground
\begin{frame}{Plan}
    \begin{makelist}[\small][1.5]
        \icon[red]{\faTimes} & Item 1\\
        \icon[red]{\faTimes} & Item 2\\
        \icon[red]{\faTimes} & Item 3
    \end{makelist}
\end{frame}


\subsection[5min-Pascal]{Eye diagram \rom{1}}
\pascalbackground
\begin{frame}{Plan}
    \begin{makelist}[\small][1.5]
        \icon[red]{\faTimes} & Item 1\\
        \icon[red]{\faTimes} & Item 2\\
        \icon[red]{\faTimes} & Item 3
    \end{makelist}
\end{frame}
