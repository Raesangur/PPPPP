%!TEX root = ../presentation.tex 


\section[Level 8]{Field lines and Fringes [26min-2h49]}

\subsection[6min-Max]{Matériaux \rom{4}}
\maxbackground

\begin{frame}{Rappel - Impédance Caractéristique}
    \begin{columns}
        \begin{column}{0.33\textwidth}
            \begin{center}
            $Z_0 = \sqrt{\dfrac{R + j \omega L}{G + j \omega C}}$\\
            \vspace{20pt}
            $G \rightarrow 0$\\
            $R \lll \omega L$\\
            \vspace{20pt}
            $Z_0 = \sqrt{\dfrac{j \omega L}{j \omega C}}$\\
            \vspace{5pt}
            $Z_0 = \sqrt{\dfrac{L}{C}}$\\
            \end{center}
        \end{column}
        \begin{column}{0.66\textwidth}
            \resizebox{\textwidth}{!}{
            \begin{circuitikz}[american voltages]
                \draw [thick]
                (0,0) to [short, *-] (10,0)
                to [R, l_=$R_{LOAD}$] (10,5)
                (0,0) to [open, v<=$V$] (0,5)
                to [short, *- ,i=$i$] (2,5)
                to [R, l_=$R$] (4, 5)
                to [short] (7, 5)
                to [american inductor, l_=$L$] (9, 5)
                to [short] (10,5)
                (0, 5) to [open] (4.5, 5)
                to [C, l_=$C$] (4.5, 0)
                (0, 5) to [open] (6, 5)
                to [R, l_=$G$] (6, 0)
                ;
            \end{circuitikz}
            }
        \end{column}
    \end{columns}
\end{frame}

\begin{frame}{Conduction G}
    \begin{columns}
        \begin{column}{0.33\textwidth}
            \begin{center}
            $Z_0 = \sqrt{\dfrac{R + j \omega L}{G + j \omega C}}$\\
            \end{center}
            \small{
            Le parametre G possède plusieurs noms:
            \begin{itemize}
                \item Conduction\\
                \item Relaxation Losses\\
                \item Substrate Vibration Losses\\
                \item Dielectric Losses
            \end{itemize}}
        \end{column}

        \begin{column}{0.66\textwidth}
            \resizebox{\textwidth}{!}{
            \begin{circuitikz}[american voltages]
                \draw [thick]
                (0,0) to [short, *-] (10,0)
                to [R, l_=$R_{LOAD}$] (10,5)
                (0,0) to [open, v<=$V$] (0,5)
                to [short, *- ,i=$i$] (2,5)
                to [R, l_=$R$] (4, 5)
                to [short] (7, 5)
                to [american inductor, l_=$L$] (9, 5)
                to [short] (10,5)
                (0, 5) to [open] (4.5, 5)
                to [C, l_=$C$] (4.5, 0)
                (0, 5) to [open] (6, 5)
                to [R, l_=$G$] (6, 0)
                ;
            \end{circuitikz}
            }
        \end{column}
    \end{columns}
\end{frame}

\begin{frame}{FR4 Fabrication \& Defect}
    \makefigure[0.8][0.8][]{level-9/fr4-cross}
\end{frame}

\begin{frame}{Conduction on different laminates}
%https://resources.altium.com/p/increasingly-important-role-loss-tangents-pcb-laminates
    \maketable{laminate-tand}
\end{frame}

%Transmission line 4?
\subsection[6min-Max]{Impedance \rom{3}}
\maxbackground


\begin{frame}{Relation frequentielle}
    \begin{columns}
        \begin{column}{0.33\textwidth}
            \begin{center}
            $Z_0 = \sqrt{\dfrac{R + j \omega L}{G + j \omega C}}$\\
            \end{center}
        \end{column}
        \begin{column}{0.66\textwidth}
            En réalité, les paramètres sont très dépendant de la frequence du signal dans les GHz.
        \end{column}
    \end{columns}
    \vspace{30pt}
    \makefigure[0.8][0.7][]{level-8/impedance-frequency}
\end{frame}

\begin{frame}{Skin effect}
    skin effect diagram, skin effect increase resistance
\end{frame}

\begin{frame}{Skin Depth of Different Materials}
    %\begin{center}
    %    \textbf{1oz PCB = $\SI{35}{\micro\meter}$}
    %\end{center}

    \maketable{skin-depth}
\end{frame}

\begin{frame}{Indice effectif}
    Mettre equation Ee
\end{frame}

\begin{frame}{Impédance dans les GHz}
    \makefigure[0.9][0.9][Gold Microstrip on Alumina, w = 240 $\mu$m, h = 500 $\mu$m]{level-8/impedance-hf}
\end{frame}




\subsection[5min-Pascal]{Passive Component \rom{2}}
\pascalbackground
\begin{frame}{Plan}
    \begin{makelist}[\small][1.5]
        \icon[red]{\faTimes} & Frequency-dependant passives\\
        \icon[red]{\faTimes} & Item 3
    \end{makelist}
\end{frame}


\subsection[10min-Max]{Waveguide \rom{1}}
\pascalbackground
\begin{frame}{Plan}
    \begin{makelist}[\small][1.5]
        \icon[red]{\faTimes} & Qu'arrive t'il quand on monte encore plus la frequence\\
        \icon[red]{\faTimes} & Figure Guide d'onde\\
        \icon[red]{\faTimes} & Mode TE et TM\\
        \icon[red]{\faTimes} & Propagation
    \end{makelist}
\end{frame}

\begin{frame}{When it's a waveguide}    
    \maketable{model-threshold}
\end{frame}