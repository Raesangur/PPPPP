%!TEX root = ../presentation.tex 


\section[Level 8]{Dielectric Depths [26min-2h49]}

%Transmission line 4?
\subsection[6min-Max-Pascal]{Matériaux \rom{4}}
\begin{frame}{Plan}
    \begin{makelist}[\small][1.5]
        \icon[red]{\faTimes} & Relaxation Loss\\
        \icon[red]{\faTimes} & Conduction G\\
        \icon[red]{\faTimes} & Substrate vibration\\
        \icon[red]{\faTimes} & Loss Tangent
    \end{makelist}
\end{frame}

\subsection[5min-Pascal]{Passive Component \rom{2}}
\begin{frame}{Plan}
    \begin{makelist}[\small][1.5]
        \icon[red]{\faTimes} & Frequency-dependant passives\\
        \icon[red]{\faTimes} & Item 3
    \end{makelist}
\end{frame}



\subsection[10min-Pascal]{Stackup \rom{1}}
\begin{frame}{Plan}
    \begin{makelist}[\small][1.5]
                \icon[red]{\faTimes} & Conversion\\
        \icon[red]{\faTimes} & Lecture datasheet diélectrique\\
        \icon[red]{\faTimes} & Optimisation Stackup\\
        \icon[red]{\faTimes} & Propriété FR4, copper, or, plomb
    \end{makelist}
\end{frame}

\subsection[2min-Max]{Vitesse de Propagation \rom{2}}
\begin{frame}{Plan}
    \begin{makelist}[\small][1.5]
        \icon[red]{\faTimes} & Dispersion\\
        \icon[red]{\faTimes} & Item 2\\
        \icon[red]{\faTimes} & Item 3
    \end{makelist}
\end{frame}

\subsection{Fin premiere Partie}
\begin{frame}{fin}
    Fin officiel du PPMPMPPMPMPPMPPPMPPPMPPPMP05. Le reste c'est pour les crinqué
\end{frame}