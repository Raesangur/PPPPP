%!TEX root = ../presentation.tex 

\section[Intro]{Introduction}

\pascalbackground
\begin{frame}{What will be covered}
    \makefigure[1.0][0.8][]{introduction/knowledge-map}
\end{frame}

\begin{frame}{Objectifs de la Presentation}
    \begin{makelist}[\small][1.5]
        \icon{\faCheck} & Vulgariser les outils mathématiques derrière la propagation de signaux EM\\
        \icon{\faCheck} & Comprendre les limites de ses outils.\\
        \icon{\faCheck} & Enseigner les mots-clés necsessaire à aborder un problème RF\\
        \icon{\faCheck} & Parler de concepts que l'UdS ne montre pas en APP\\
        \icon{\faCheck} & Montrer des analogies pour aider a visualiser les champs EM sur un PCB\\
        \icon{\faCheck} & Devenir une ressource dans le future\\
        \icon{\faCheck} & Une exploration de sujets connexes aux PCB\\
        \icon{\faCheck} & Avoir du fun. Et créer une discussion sur de nombreux sujets.
    \end{makelist}
\end{frame}

\begin{frame}{Non-Objectifs de la Presentation}
    \begin{makelist}[\small][1.5]
        \icon[red]{\faTimes} & Montrer les mathématiques de façon rigoureuse\\
        \icon[red]{\faTimes} & Un guide exhaustif sur la propagation de signal EM\\
        \icon[red]{\faTimes} & \\
        \icon[red]{\faTimes} & 
    \end{makelist}
\end{frame}

\begin{frame}{Nos Qualifications}
   \makefigure[1.0][0.8][]{introduction/qualification-meme}
\end{frame}

\begin{frame}{Complexitée de PCB}
    \maketable{pcb-complexity}
    \makefigure[0.8][0.8][]{introduction/pcbs}
\end{frame}

\begin{frame}{Quote de Julien Guay}
    "Si tu veux une job en PCB, il faut que tu saches comment faire du 10 GHz" - Julien Guay, Finissant Promo 66
\end{frame}

