%!TEX root = ../presentation.tex

\section[Level 6]{Basic Building Blocks [12min-2h07]}
\introbackground
\begin{frame}[plain, label=intro-level-6]
    \centering
    \Large
    \textcolor{white}{\textbf{Sujets Abordés dans la Section:}}\\
    \vspace{24pt}
    \begin{tabular}{c l}
        \textcolor{UDSgreenFierte}{\faEye}
            & \textcolor{white}{Avantages/Inconveniant de different types de \textbf{Lignes de Transmissions Planaires}}\\
            [0.3em]
        \textcolor{UDSgreenFierte}{\faHubspot}
            & \textcolor{white}{Comprendre l'effets de \textbf{Discontinuitées} sur les traces}\\
            [0.3em]
        \textcolor{UDSgreenFierte}{\faEye}
            & \textcolor{white}{Fonctionnement \textbf{d'Oscillateur \& Cristaux}}\\
            [0.3em]
    \end{tabular}
\end{frame}
\defaultbackground




\subsection[2min-Max]{Vitesse de Propagation \rom{2}}
\maxbackground
\begin{frame}{Plan}
    \begin{makelist}[\small][1.5]
        \icon[red]{\faTimes} & Dispersion\\
        \icon[red]{\faTimes} & Item 2\\
        \icon[red]{\faTimes} & Item 3
    \end{makelist}
\end{frame}




\subsection[5min-Max]{Coaxial Cable}
\maxbackground
\begin{frame}{Plan}{Microwave connector type}
    \makefigure[0.85][0.85][\href{https://www.microwaves101.com/encyclopedias/microwave-coaxial-connectors}{[Source]}]{level-6/microwave-connector}
\end{frame}




\subsection[5min-Max]{Planar Transmission Lines \rom{1}}
\maxbackground
\begin{frame}{Plan}
    \textbf{Je pense qu'on devrait rennomer ca Structure I ou Planar Transmission Line. Waveguide c'est plus un modele.}\\
    \begin{makelist}[\small][1.5]
        \icon[red]{\faTimes} & Transmission-Frequency plot\\
        \icon[red]{\faTimes} & Explain Strip line\\
        \icon[red]{\faTimes} & Show other structures\\
        \icon[red]{\faTimes} & CPWG, Microstrip, GCPWG
    \end{makelist}
\end{frame}

\begin{frame}{Microstrip}
    \begin{twocolumns}[0.5]
        \leftcol
            \vspace{-30pt}
            \makefigure[1.0][1.0][]{level-6/microstrip}
        \rightcol
            \begin{makelist}[\small][1.5]
                \icon{\faCheck} & Simple à fabriquer\\
                \icon{\faCheck} & Facile à modéliser\\
                \icon{\faCheck} & Compact\\
                \icon[red]{\faTimes} & Grande pertes par radiation >30GHz\\
                \icon[red]{\faTimes} & Beaucoup de Dispersion\\
            \end{makelist}
    \end{twocolumns}
\end{frame}

\begin{frame}{Stripline}
    \begin{twocolumns}[0.5]
        \leftcol
            \vspace{-30pt}
            \makefigure[1.0][1.0][]{level-6/stripline}
        \rightcol
            \begin{makelist}[\small][1.5]
                \icon{\faCheck} & Tres peu de radiation\\
                \icon[red]{\faTimes} & Nescessite 2 GND plane\\
            \end{makelist}
    \end{twocolumns}
\end{frame}

\begin{frame}{Coplanar Waveguide}
    \begin{twocolumns}[0.5]
        \leftcol
            \vspace{-30pt}
            \makefigure[1.0][1.0][]{level-6/cpwg}
        \rightcol
            \begin{makelist}[\small][1.5]
                \icon{\faCheck} & Peu de Dispersion\\
                \icon{\faCheck} & Attenuation des Radiations\\
                \icon[red]{\faTimes} & Nescessite plus d'espace\\
                \icon[red]{\faTimes} & Via Stitching ($\approx \frac{\lambda}{10}$)\\
            \end{makelist}
    \end{twocolumns}
\end{frame}


\subsection[5min-Pascal]{Signal Source \rom{4}}
\pascalbackground
\begin{frame}{Plan}
    \begin{makelist}[\small][1.5]
        \icon[red]{\faTimes} & Crystals\\
        \icon[red]{\faTimes} & Oscillators\\
        \icon[red]{\faTimes} & Item 3
    \end{makelist}
\end{frame}

\subsection[5min-Max]{Discontinuities}
\maxbackground
\begin{frame}{Open Microstrip Stub}
    \begin{twocolumns}[0.5]
        \leftcol
            \vspace{-30pt}
            \makefigure[1.0][1.0][Open microstrip Stub]{level-6/open-stub}
        \rightcol
            fuck
    \end{twocolumns}
\end{frame}

\begin{frame}{Step}
    \begin{twocolumns}[0.5]
        \leftcol
            \vspace{-30pt}
            \makefigure[1.0][1.0][Step]{level-6/step}
        \rightcol
            fuck
    \end{twocolumns}
\end{frame}

\begin{frame}{Notch}
    \begin{twocolumns}[0.5]
        \leftcol
            \vspace{-30pt}
            \makefigure[1.0][1.0][Notch]{level-6/notch}
        \rightcol
            fuck
    \end{twocolumns}
\end{frame}

\begin{frame}{Gap}
    \begin{twocolumns}[0.5]
        \leftcol
            \vspace{-30pt}
            \makefigure[1.0][1.0][Gap]{level-6/gap}
        \rightcol
            fuck
    \end{twocolumns}
\end{frame}

\begin{frame}{Crossover}
    \begin{twocolumns}[0.5]
        \leftcol
            \vspace{-30pt}
            \makefigure[1.0][1.0][Crossover]{level-6/crossover}
        \rightcol
            fuck
    \end{twocolumns}
\end{frame}

\begin{frame}{Bend}
    \begin{twocolumns}[0.5]
        \leftcol
            \vspace{-30pt}
            \makefigure[1.0][1.0][Bend]{level-6/bend}
        \rightcol
            fuck
    \end{twocolumns}
\end{frame}

%\subsection{Impedance Transformer}
\begin{frame}{Tapered Line}
    \begin{twocolumns}[0.5]
        \leftcol
            \vspace{-30pt}
            \makefigure[1.0][1.0][Transformer]{level-6/transformer}
        \rightcol
            fuck
    \end{twocolumns}
\end{frame}

\subsection[2min-Max]{Couplers}
\maxbackground
\begin{frame}{Plan}
    \begin{makelist}[\small][1.5]
        \icon[red]{\faTimes} & Coupling\\
        \icon[red]{\faTimes} & Transfer - S-parameters\\
        \icon[red]{\faTimes} & Item 3
    \end{makelist}
\end{frame}

\begin{frame}{Coupling Coefficient}
    \textbf{Mettre le gif?}
\end{frame}

\begin{frame}{Coupling Coefficient}
    \makefigure[0.8][0.8][]{level-6/coupling-distance-20reso-light}
\end{frame}

\begin{frame}{Coupling Length}
    \makefigure[0.9][0.9][]{level-6/coupling_heatmap_3.0-light}
\end{frame}

\begin{frame}{Crosstalk calculation}
    \makefigure[0.8][0.8][]{level-6/coupling-crosstalk-light.png}
\end{frame}

\begin{frame}{Application}
    Mettre exemple de circuit RF avec mesure signal de retour d'antenne
    %\makefigure[0.9][0.9][]{level-6/coupling-crosstalk-light.png}
\end{frame}

\subsection[2min-Max]{Stubs}
\maxbackground
\begin{frame}{Plan}
    \begin{makelist}[\small][1.5]
        \icon[red]{\faTimes} & Item 1\\
        \icon[red]{\faTimes} & Item 2\\
        \icon[red]{\faTimes} & Item 3
    \end{makelist}
\end{frame}

\begin{frame}{Microstrip Stub}
    \begin{twocolumns}[0.5]
        \leftcol
            \vspace{-30pt}
            \makefigure[1.0][1.0][Open Microstrip Stub]{level-6/open-stub}
        \rightcol
            fuck
    \end{twocolumns}
\end{frame}

\begin{frame}{Planar Radial Stub}
    \begin{twocolumns}[0.5]
        \leftcol
            \vspace{-30pt}
            \makefigure[1.0][1.0][Planar Radial Stub]{level-6/radial-stub}
        \rightcol
            fuck
    \end{twocolumns}
\end{frame}

\begin{frame}{Butterfly Radial Stub}
    \begin{twocolumns}[0.5]
        \leftcol
            \vspace{-30pt}
            \makefigure[1.0][1.0][Butterfly radial Stub]{level-6/butterfly-stub}
        \rightcol
            fuck
    \end{twocolumns}
\end{frame}

\begin{frame}{Via stubs}
    \makefigure[1.0][1.0][]{level-6/via-stubs}
\end{frame}

\begin{frame}{Via stubs}
    \vspace{-40pt}
    \makefigure[1.0][1.0][]{level-6/via-stubs-reflection}
\end{frame}








